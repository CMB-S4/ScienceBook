Neutrinos are the least explored corner of the Standard Model of particle physics.  The 2015 Nobel Prize recognized the discovery of neutrino oscillations, which shows that they have mass. However, the overall scale of the masses and the full suite of mixing parameters are still not measured.  Cosmology offers a unique view of neutrinos; they were produced in large numbers in the high temperatures of the early universe and left a distinctive imprint in the cosmic microwave background and on the large-scale structure of the universe. Therefore, CMB-S4 and large-scale structure surveys together will have the power to detect properties of neutrinos that supplement those probed by large terrestrial experiments such as short- and long-baseline as well as neutrino-less double beta decay experiments.


Specifically, while long baseline experiments are sensitive to the differences in the masses of the different types of neutrinos, CMB-S4  will probe the sum of all the neutrino masses. The current lower limit on the sum of neutrino masses imposed by oscillation experiments is \mbox{$\sum m_\nu=58\,\rm{meV}$}. CMB-S4, in conjunction with upcoming baryon acoustic oscillation surveys, will measure this sum with high significance. Once determined, the sum of neutrino masses will inform the prospects for future neutrino-less double beta decay experiments that aim to determine whether neutrinos are their own anti-particle. Furthermore, an upper limit below \mbox{$\sum m_\nu=105\,\rm{meV}$} would disfavor the inverted mass hierarchy. Finally, CMB-S4 is particularly sensitive to the possible existence of additional neutrinos that interact even more weakly than the neutrinos in the Standard Model. These so-called {\it sterile} neutrinos are also being vigorously pursued with short baseline experiments around the world. So the combination of CMB-S4, large scale structure surveys, and terrestrial probes adds up to a comprehensive assault on the three-neutrino paradigm.

