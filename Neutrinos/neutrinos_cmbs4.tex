%%%%%% CMB-S4 Neutrinos Chapter  %%%%%%%%%%%%%%%%
 
\chapter{Neutrino Physics from the Cosmic Microwave Background}
\renewcommand*\thesection{\arabic{section}}

\def\beq{\begin{equation}}
\def\eeq{\end{equation}}

\def\bea{\begin{eqnarray}}
\def\eea{\end{eqnarray}}

\def\Neff{N_{\rm eff}}
\def\gtrsim{\raise-.75ex\hbox{$\buildrel>\over\sim$}}
%%%%%%%%%%%%%%%%%%%%%%%%%%%%%%%%%%%%%%%%%%%%%%%%%%%%%%%%%%%
%%%%%%%%%%%%%%%%%%%%%%%%%%%%%%%%%%%%%%%%%%%%%%%%%%%%%%%%%%%
%%%%%%%%%%%%%%%%%%%%%%%%%%%%%%%%%%%%%%%%%%%%%%%%%%%%%%%%%%%
%%%%%%%%%%%%%%%%%%%%%%%%%%%%%%%%%%%%%%%%%%%%%%%%%%%%%%%%%%%

\section{Introduction}

Direct interactions between neutrinos and observable matter effectively ceased about one second after the end of inflation.  Nevertheless, the total energy density carried by neutrinos was comparable to other matter sources through today.  As a result, the gravitational effect of the neutrinos is detectable both at the time of recombination and in the growth of structure at later times~\cite{Abazajian:2013oma}, leaving imprints in the temperature and polarization spectrum as well as in CMB lensing.

CMB-S4 can improve our understanding of neutrino physics in regimes of interest for both cosmology and particle physics.  Arguably the most important parameters of interest will be the sum of the neutrino masses $\sum m_\nu$ and the effective number of neutrino species, $\Neff$.  These two parameters have natural targets that are within reach of a CMB-S4 experiment:
\begin{itemize}
\item $ \sum m_\nu \gtrsim 58$ meV is the lower bound guaranteed by observations of solar and atmospheric neutrino oscillations.  A CMB experiment with $\sigma(\sum m_\nu) < 20$ meV would be guaranteed a detection of a least 3$\sigma$.
\item $\Delta \Neff > 0.027$ is predicted for any light particle that was in thermal equilibrium with the standard model.  A CMB experiment producing $\sigma(\Neff) \lesssim 0.01$ would be sensitive to all models in this very broad class of extensions of the Standard Model, which includes a wide range of axions and axion-like particles.  
\end{itemize}
Current CMB data already provides a robust detection of the cosmic neutrino background at $>10 \sigma$.  A CMB-S4 experiment will provide an order of magnitude improvement in sensitivity that opens a new window back to the time of neutrino decoupling and beyond.

Section 2 will review the motivation for studying neutrino masses with cosmological probes, and specifically with the CMB.  We will explain why cosmology is sensitive to $\sum m_\nu$ via different probes and how it is complementary to the experimental neutrino effort.  Section 3 will review the physics of $\Neff$ and its role as probe of the C$\nu$B and as sensitive tool for beyond the Standard model physics.  We will emphasize the unique impact $\Neff$ has on the CMB that makes it distinguishable from other extensions of $\Lambda$CDM.  In Section 4 we will discuss the implications for a variety of well motivated models, including sterile neutrinos and axions.   

\section{Neutrino Mass}

\subsection{Theory Review}

{\it Subsection to be completed by Marilena Loverde}



\subsection{Observational Signatures and Target}
%
Massive neutrinos contribute to the critical density as
\beq
\Omega_\nu h^2 \simeq \frac{\sum m_\nu}{93 \, {\rm eV}} \ .
\eeq
As discussed above, the signature of massive neutrinos manifests through
the energy density $\Omega_\nu$ making its transition from relativistic
radiation to non-relativistic matter as its temperature drops;
the transition occurs at around $z_\mathrm{nr} \sim
2000m_\nu/1\,\mathrm{eV}$.  During the epoc when
neutrinos are relativistic, they free-stream out of the over-dense regions
and washes out the matter fluctuations on small scales.  This effect
turns off roughly at scales larger than the horizon scale at
the redshift $z_\mathrm{nr}$.  Thus, comparison of the amplitudes of the
fluctuations at large and small
scales probes neutrino mass.

The amplitudes at large scales are precisely measured through
primordial CMB fluctuations in both TT and EE power spectra, except for the
uncertainty from the optical depth $\tau$, which we discuss later.
%
The large scale structure (LSS) measures the small scales; this is the
area where we expect significant improvement through CMB S4.

\subsubsection{CMB Power Spectra (TT, EE, Lensing Potential)}
Measurement of the lensing potential of LSS through gravitational
lensing of the CMB provides us the opportunity of precisely measuring
the small-scale amplitudes.  (Refer to lensing section?)
%
Enumerate current measurements.  Mention future prospects.?

Neutrino mass can also be probed through the early Integrated Sachs
Wolfe (ISW) effect in the primordial TT and EE power spectra, since
massive neutrinos changes the expansion history by affecting the
duration of the radiation dominant era.  This measurement is already
cosmic-variance limited and only marginal improvement is possible in
future.  

\subsubsection{SZ Cluster Abundance}
Abundance of galaxy clusters also gives us a measure of the 
amplitudes at small scales ($\simeq 0.1h\mathrm{Mpc}^{-1}$).
%
CMB datasets can be used to measure the abundance through the
Sunyaev-Zel'dvich (SZ) effect.
%
However, the interpretation of this dataset to derive the neutrino mass
 requires an accurate understanding of 
 the non-trivial mass-observable relation.
[Mention mass-observable relation study works in progress ...]


\subsubsection{Cross-correlations with External Datasets}
Redshift tomography?  Calibrating optical surveys to help them measuring
neutrino mass?  (Remember LSST is a DOE project and thus helping them
makes a good case.)


\subsection{Forecasts}

\subsection{Relation to Lab Experiments}

Relation between lab measurements of neutrino masses and the cosmological measurements.

{\it Subsection to be completed by Clarence Chang}

\section{Effective Number of Neutrinos}

\subsection{Theory Review}

{\it Subsection to be completed by Joel Meyers with a contribution from George Fuller}

The {\it effective} number of neutrinos defined to be
\beq
\Neff=\frac{8}{7} \left(\frac{11}{4}\right)^{4/3} \frac{\rho_{R}-\rho_\gamma}{\rho_\gamma}  \ ,
\eeq
where $\rho_{R}$ is the total energy density in radiation and $\rho_\gamma$ is the energy density in photons. 

{\it Status of calculations in the Standard Model} -- (Contribution expected from G. Fuller)


{\it Light Species}

\begin{figure}[t!]
\begin{center}
\includegraphics[width=0.65\textwidth]{Neutrinos/Neff.pdf}
\caption{Contribution to $\Neff$ from a massless field that was in thermal equilibrium with the Standard model at temperatures $T> T_{\rm freeze}$.  For $T_{\rm freeze} \gg m_{\rm top}$, these curves saturate with $\Delta \Neff > 0.027$.   The region in red shows the range of forecasts for $\sigma(\Neff)$ for plausible CMB-S4 configurations. }
\label{fig:limits}
\end{center}
\end{figure}

\subsection{Observational Signatures}

Cosmic neutrinos play to two important roles in the CMB that are measured by $\Neff$.  They contribute to the total energy in radiation which controls the expansion history and, indirectly, the damping tail.  The fluctuations of neutrinos and any other free streaming radiation also produces a constant shift in the of the acoustic peaks.  These two effects drive both current and future constraints on $\Neff$.  

The effect of neutrinos on the damping tail has dominated has historically driven the constraint on $\Neff$ in the CMB.  The largest effect is from the mean-free path of photons, which introduces a suppression $e^{-(k/k_d)^2}$ to short wavelengths, with~\cite{Zaldarriaga:1995gi}
\beq
k_d^{-2} =\int \frac{da}{a^3 \sigma_T n_e H} \frac{R^2+ \frac{16}{15}(1+R)}{6(1+R)^2} \ ,
\eeq
where $R$ is the ratio of the energy in baryons to photons, $n_e$ is the density of free electrons and $\sigma_T$ is the Thompson cross-section.  The damping scale is sensitive to the energy in all radiation through $H \propto \sqrt{\rho_{\rm radiation}}$ during radiation domination (which is applicable at high $\ell$, and is therefore sensitive to $\Neff$ or any form of dark radiation.  At this level, this also illustrates the degeneracy with $n_e$ which may be altered by the helium fraction, $Y_p$.

In reality, the effect on the damping tail is actually subdominant to the change to the scale of matter radiation equality and the location of the first acoustic peak~\cite{Hou:2011ec}.  As a result, the effect on neutrinos on the damping tail is more accurately represented by holding the first acoustic peak fixed.  This changes the sign of the effect on the damping tail, but the intuition for the origin of the effect (and degeneracy) remain applicable.

In addition to the effect on the Hubble expansion, perturbations in neutrinos effect the photon-baryon fluid through their gravitational influence.  The contributions from neutrinos are well described by a correction to the amplitude and the phase of the acoustic peaks in both temperature and polarization~\cite{Bashinsky:2003tk}.  The phase shift is a particularly compelling signature as it is not degenerate with other cosmological parameters~\cite{Bashinsky:2003tk,Baumann:2015rya}.  This effect is the result of the free-streaming nature of neutrinos that allows propagation speeds of effectively the speed of light (while the neutrinos are relativistic).  This effect is sensitive to any gravitationally coupled light thermal relics.

E-mode polarization will play an increasingly important role for several reasons.  First of all, the acoustic peaks are sharper in polarization which makes measurements of the peak locations more precise, and therefore aid the measurement of the phase shift.  The second reason is that polarization breaks a number of degeneracies that would also affect the damping tail~\cite{Baumann:2015rya}.

{\it Status of current observations} -- Planck has provided a strong constraint on $\Neff = 3.15 \pm 0.23$ when combining both temperature and polarization data.  The addition of polarization data has both improved the constraint on $\Neff$ and reduced the impact of the degeneracy with $Y_p$.  Recently, the phase shift from neutrinos has also been established direction in the Planck temperature data~\cite{Follin:2015hya}.  This provides the most direct evidence for presence of free-steaming radiation in the early universe, consistent with the cosmic neutrino background.


\subsection{Forecasts}

\subsection{Thermal History and Big Bang Nucleosynthesis}

{\it Subsection to be completed by Joel Meyers}



\section{Sterile Neutrinos and Other Targets}
\subsection{Sterile Neutrinos}

{\it Subsection to be completed by Kevork Abazajian}

A number of recent neutrino oscillation experiments have anomalies
that are possible indications of four or more mass eigenstates. The
first set of anomalies are in short baseline oscillation experiments:
first, in the Liquid Scintillator Neutirno Detector (LSND) experiment,
where electron antineutrinos were observed in a pure muon antineutrino
beam \cite{Athanassopoulos:1997pv}. The MiniBooNE Experiment also sees
an excess of electron neutrinos and antineutrinos in their muon
neutrino beam \cite{Aguilar-Arevalo:2013pmq}. Two-neutrino oscillation
interpretations of the results indicate mass splittings of $\Delta m^2
\approx 1\rm\ eV^2$ and mixing angles of $\sin^2 2\theta \approx
3\times 10^{-3}$ \cite{Aguilar-Arevalo:2013pmq}. Another anomaly
arises from re-evaluations of reactor antineutrino fluxes that
indicate a an increased flux of antineutrinos as well as a lower
neutron lifetime and commensurately increase the antineutrino events
from nuclear reactors by 6\%. This brought previous agreement of
reactor antineutrino experiments to have a $\sim$6\% deficit
\cite{Mention:2011rk,Huber:2011wv}. Another indication consistent with
sterile neutrinos are in radio-chemical gallium experiments for solar
neutrinos. In their callibrations, a 5-20\% deficit of the measured
count rate was found when intense sources of electron neutrinos from
electron capture nuclei were placed in proximity to the
detectors. Such a deficit could be produced by a $>1\rm\ eV$ sterile
neutrino with appreciable mixing with electron neutrinos
\cite{Bahcall:1994bq,Giunti:2010zu}. Some simultaneous fits to the
short baseline anomalies and reactor neutrino deficits, commensurate
with short baseline constraints, appear to prefer at least two extra
sterile neutrino states \cite{Conrad:2012qt,Kopp:2013vaa}, but see
Ref.~\cite{Giunti:2015mwa}. Because such neutrinos have relatively
large mixing angles, they would be thermalized in the early universe
with a standard thermal history, and affect primordial nucleosynthesis
\cite{Abazajian:2002bj}. 

In addition, there are combinations of CMB plus LSS
datasets that are in tension, particularly with a smaller amplitude of
fluctuations at small scale than that inferred in zero neutrino mass
modesl. This would be alleviated with the
presence of massive neutrinos, extra neutrinos, or both. In particular,
cluster abundance analyses \cite{Wyman:2013lza,Ade:2015fva} and weak lensing analyses
\cite{Battye:2013xqa} indicate a lower amplitude of
fluctuations than zero neutrino mass \cite{Giusarma:2014zza}0. Baryon Acoustic
Oscillation measures of expansion history are affected by the presence
of massive neutrinos, and nonzero neutrino mass may be indicated 
\cite{Beutler:2014yhv}, though 2015 Planck results show a lack of such
alleviation in cases with massive or extra neutrinos
\cite{Ade:2015xua}. 

There is a potential emergance of both laboratory and cosmological
indications of massive and, potentially, extra neutrinos. However, the
combined requirements of the specific masses to produce the short
baseline results, along with mixing angles that require thermalized
sterile neutrino states, are inconsistent at this point with
cosmological tension data sets
\cite{Joudaki:2012uk,Archidiacono:2013xxa}. The tension data sets are
not highly significant at this point ($\lesssim 3\sigma$), and there
are a significant set of proposals for short baseline oscillation
experiment follow up \cite{Abazajian:2012ys}. Future high-sensitivity
probes of neutrino mass and number such as CMB-S4 will be able to
definitively test for the presence of extra neutrino number and mass
consistent with sterile neutrinos.


\subsection{Axion-like Particles}

{\it Subsection to be completed by Dan Green}

A ubiquitous component of extensions of the Standard model is axions and/or axion-like particles (ALPs).  Axions have been introduced to solve the strong-CP problem~\cite{Peccei:1977hh}, the hierarchy problem~\cite{Graham:2015cka} and the naturalness of inflation~\cite{Freese:1990rb}.  Furthermore, they appear generically in string theory, in large numbers, leading to the qualitative phenomena describe as the string axiverse~\cite{Arvanitaki:2009fg}.

ALPS typically appear as (pseudo)-Goldstone bosons of some high energy global symmetry.  At low energy, the mass of the ALP is protected by an approximate shift symmetry of the general form $a \to a + c$ where $a$ is the axion and $c$ is a constant (for non-abelian Goldstone bosons, this transformation will include higher order terms in $a$).  We will define an ALP to be any such particle where all of the couplings of the axion to the Standard model respect such a symmetry.  This symmetry may be softly broken with an explicit mass term, although this is highly restricted in the case of the QCD axion.

Two couplings of particular interest for axion phenomenology are the coupling to gluons and photons, 
\beq
\frac{1}{4} g_{a \gamma \gamma} a \tilde F_{\mu \nu}F^{\mu\nu} \ , \qquad \qquad \frac{1}{4} g_{a g g} a \tilde G_{\mu \nu}G^{\mu\nu}  \ .
\eeq
These couplings typically appear as the consequence of chiral anomalies.  The coupling of the axion to gluons is what makes the solution to the strong-CP problem possible.  The coupling to photons is somewhat model dependent but typically arises in conjunction with the gluon coupling.  In addition or instead of these coupling, a variety of possible couplings to matter may also be included.

Two very common features of these models is that the axions as typically light (in many cases, $m \ll 1$ eV) and their interactions are suppressed by powers of $f_a$.  These two features make ALPS are particularly compelling target.  Because of the small masses, they will often behave as relativistic species in the CMB.  Furthermore, because their production rate will scale as $T^{2n +1} / f_a^{2n}$ for some $n \geq 1$, they are likely to be thermalized at high temperatures.  Given that $\Delta \Neff > 0.027$ under such circumstances, a CMB experiment with sensitivity at this level will be sensitive to a very wide range of ALP models.

{\it Status of current observations} -- Current constrains on ALPs arise from a combination of experimental~\cite{Graham:2015ouw}, astrophysical~\cite{Raffelt:2012kt} and cosmological~\cite{Marsh:2015xka} probes.  Current cosmological constraints are driven by several effects that depend on the mass of the axion.  For axion masses greater than 100 eV, stable thermal ALPs are easily excluded because they produce dark matter abundances inconsistent with observations.  By including the free streaming effects of thermal QCD-axions,  Planck data~\cite{DiValentino:2015wba} combined with local measurements provide the constrain $m_a < 0.525$ eV (95 \% CL).  At larger masses, ALPs become unstable and can be constrained by the change to $\Neff$ from energy injection as well as from spectral distortions and changes to BBN~\cite{Cadamuro:2011fd,Follin:2015hya}

{\it Implications for CMB Stange IV} -- Sensitive to $\Neff$ of order $\sigma(\Neff) \simeq 10^{-2}$ has the sensitive to probe then entire mass range of ALPs down to $m_a =0$ under the assumption that it thermalized in the early universe.  Interpreting such bounds in terms of the couplings of axions is more complicated~\cite{Brust:2013xpv} and can depend on assumptions about the reheating temperature.  For high (but plausible) reheat temperatures of $10^{10}$ GeV, CMB stage IV would be sensitive to $g_{a\gamma \gamma}, g_{a g g} \gtrsim 10^{-13} {\rm GeV}^{-1}$~\cite{}, which exceeds current constraints and future probes for a range possible axions masses (including the QCD axion).  

\subsection{Connections to BBN and Spectral Distortions}


\bibliography{cmbs4}

%%
%% Populate the .bib file with entries from SPIRES Bibtex (preferred)
%% or ADS Bibtex (if no SPIRES entry).
%%  SPIRES will also supply the CITATION line information; please include it.
%%


