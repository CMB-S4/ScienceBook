%%%%%% CMB-S4 Neutrinos Chapter  %%%%%%%%%%%%%%%%
 
\chapter{Neutrino Physics from the Cosmic Microwave Background}
\renewcommand*\thesection{\arabic{section}}

\def\beq{\begin{equation}}
\def\eeq{\end{equation}}

\def\bea{\begin{eqnarray}}
\def\eea{\end{eqnarray}}

\def\Neff{N_{\mathrm eff}}
\def\gtrsim{\raise-.75ex\hbox{$\buildrel>\over\sim$}}
%%%%%%%%%%%%%%%%%%%%%%%%%%%%%%%%%%%%%%%%%%%%%%%%%%%%%%%%%%%
%%%%%%%%%%%%%%%%%%%%%%%%%%%%%%%%%%%%%%%%%%%%%%%%%%%%%%%%%%%
%%%%%%%%%%%%%%%%%%%%%%%%%%%%%%%%%%%%%%%%%%%%%%%%%%%%%%%%%%%
%%%%%%%%%%%%%%%%%%%%%%%%%%%%%%%%%%%%%%%%%%%%%%%%%%%%%%%%%%%

\section{Introduction}

Direct interactions between neutrinos and observable matter effectively ceased about one second after the end of inflation.  Nevertheless, the total energy density carried by neutrinos was comparable to other matter sources through today.  As a result, the gravitational effect of the neutrinos is detectable both at the time of recombination and in the growth of structure at later times~\cite{Abazajian:2013oma}.


\section{Neutrino Mass}

\subsection{Theory Review}




\subsection{Observational Signatures and Target}

Massive neutrinos contribute to the crical density as
\beq
\Omega_\nu h^2 \simeq \frac{\sum m_\nu}{93 \, {\rm eV}} \ .
\eeq

\subsubsection{CMB Power Spectra (TT, EE, Lensing Potential)}

\subsubsection{SZ Cluster Abundance}

\subsubsection{Cross-correlations with External Datasets}


\subsection{Forecasts}

\subsection{Relation to Lab Experiments}

Relation between lab measurements of neutrino masses and the cosmological measurements.

\section{Effective Number of Neutrinos}

\subsection{Theory Review}

{\it Subsection to be completed by Joel Meyers with a contribution from George Fuller}

The {\it effective} number of neutrinos defined to be
\beq
\Neff=\frac{8}{7} \left(\frac{11}{4}\right)^{4/3} \frac{\rho_{R}-\rho_\gamma}{\rho_\gamma}  \ ,
\eeq
where $\rho_{R}$ is the total energy density in radiation and $\rho_\gamma$ is the energy density in photons. 

{\it Status of calculations in the Standard Model} -- (Contribution expected from G. Fuller)


\subsection{Observational Signatures and Target}

{\it Status of current observations} -- Planck has provided a strong constraint on $\Neff = 3.15 \pm 0.23$ when combining both temperature and polarization data.  

\subsection{Forecasts}

\subsection{Thermal History and Big Bang Nucleosynthesis}

{\it Subsection to be completed by Joel Meyers}

\section{Sterile Neutrinos}

\subsection{Theory and Motivation}

\subsection{Cosmological Implications}

\subsection{Relation to Lab Experiments}

\section{Other Targets}

\subsection{Axion-like Particles}

\subsection{Energy Injection and Particle Decays}

\subsection{Connections to BBN and Spectral Distortions}


\bibliography{cmbs4}

%%
%% Populate the .bib file with entries from SPIRES Bibtex (preferred)
%% or ADS Bibtex (if no SPIRES entry).
%%  SPIRES will also supply the CITATION line information; please include it.
%%


