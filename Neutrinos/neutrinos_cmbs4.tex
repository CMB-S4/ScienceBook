%%%%%% CMB-S4 Neutrinos Chapter  %%%%%%%%%%%%%%%%
 
\chapter{Neutrino Physics from the Cosmic Microwave Background}
\renewcommand*\thesection{\arabic{section}}

\def\beq{\begin{equation}}
\def\eeq{\end{equation}}

\def\bea{\begin{eqnarray}}
\def\eea{\end{eqnarray}}

\def\Neff{N_{\rm eff}}
\def\gtrsim{\raise-.75ex\hbox{$\buildrel>\over\sim$}}
%%%%%%%%%%%%%%%%%%%%%%%%%%%%%%%%%%%%%%%%%%%%%%%%%%%%%%%%%%%
%%%%%%%%%%%%%%%%%%%%%%%%%%%%%%%%%%%%%%%%%%%%%%%%%%%%%%%%%%%
%%%%%%%%%%%%%%%%%%%%%%%%%%%%%%%%%%%%%%%%%%%%%%%%%%%%%%%%%%%
%%%%%%%%%%%%%%%%%%%%%%%%%%%%%%%%%%%%%%%%%%%%%%%%%%%%%%%%%%%

\section{Introduction}

Direct interactions between neutrinos and observable matter effectively ceased about one second after the end of inflation.  Nevertheless, the total energy density carried by neutrinos was comparable to other matter sources through today.  As a result, the gravitational effect of the neutrinos is detectable both at the time of recombination and in the growth of structure at later times~\cite{Abazajian:2013oma}, leaving imprints in the temperature and polarization spectrum as well as in CMB lensing.

CMB-S4 can improve our understanding of neutrino physics in regimes of interest for both cosmology and particle physics.  Arguably the most important parameters of interest will be the sum of the neutrino masses $\sum m_\nu$ and the effective number of neutrino species, $\Neff$.  These two parameters have natural targets that are within reach of a CMB-S4 experiment:
\begin{itemize}
\item $ \sum m_\nu \gtrsim 58$ meV is the lower bound guaranteed by observations of solar and atmospheric neutrino oscillations.  A CMB experiment with $\sigma(\sum m_\nu) < 20$ meV would be guaranteed a detection of a least 3$\sigma$.
\item $\Delta \Neff > 0.027$ is predicted for any light particle that was in thermal equilibrium with the standard model.  A CMB experiment producing $\sigma(\Neff) \lesssim 0.01$ would be sensitive to all models in this very broad class of extensions of the Standard Model, which includes a wide range of axions and axion-like particles.  
\end{itemize}
Current CMB data already provides a robust detection of the cosmic neutrino background at $>10 \sigma$.  A CMB-S4 experiment will provide an order of magnitude improvement in sensitivity that opens a new window back to the time of neutrino decoupling and beyond.

Section 2 will review the motivation for studying neutrino masses with cosmological probes, and specifically with the CMB.  We will explain why cosmology is sensitive to $\sum m_\nu$ via different probes and how it is complementary to the experimental neutrino effort.  Section 3 will review the physics of $\Neff$ and its role as probe of the C$\nu$B and as sensitive tool for beyond the Standard model physics.  We will emphasize the unique impact $\Neff$ has on the CMB that makes it distinguishable from other extensions of $\Lambda$CDM.  In Section 4 we will discuss the implications for a variety of well motivated models, including sterile neutrinos and axions.   

\section{Neutrino Mass}

\subsection{Theory Review}




\subsection{Observational Signatures and Target}

Massive neutrinos contribute to the crical density as
\beq
\Omega_\nu h^2 \simeq \frac{\sum m_\nu}{93 \, {\rm eV}} \ .
\eeq

\subsubsection{CMB Power Spectra (TT, EE, Lensing Potential)}

\subsubsection{SZ Cluster Abundance}

\subsubsection{Cross-correlations with External Datasets}


\subsection{Forecasts}

\subsection{Relation to Lab Experiments}

Relation between lab measurements of neutrino masses and the cosmological measurements.

\section{Effective Number of Neutrinos}

\subsection{Theory Review}

{\it Subsection to be completed by Joel Meyers with a contribution from George Fuller}

The {\it effective} number of neutrinos defined to be
\beq
\Neff=\frac{8}{7} \left(\frac{11}{4}\right)^{4/3} \frac{\rho_{R}-\rho_\gamma}{\rho_\gamma}  \ ,
\eeq
where $\rho_{R}$ is the total energy density in radiation and $\rho_\gamma$ is the energy density in photons. 

{\it Status of calculations in the Standard Model} -- (Contribution expected from G. Fuller)


\subsection{Observational Signatures and Target}

{\it Status of current observations} -- Planck has provided a strong constraint on $\Neff = 3.15 \pm 0.23$ when combining both temperature and polarization data.  

\subsection{Forecasts}

\subsection{Thermal History and Big Bang Nucleosynthesis}

{\it Subsection to be completed by Joel Meyers}



\section{Sterile Neutrinos and Other Targets}
\subsection{Sterile Neutrinos}

\subsection{Axion-like Particles}

\subsection{Energy Injection and Particle Decays}

\subsection{Connections to BBN and Spectral Distortions}


\bibliography{cmbs4}

%%
%% Populate the .bib file with entries from SPIRES Bibtex (preferred)
%% or ADS Bibtex (if no SPIRES entry).
%%  SPIRES will also supply the CITATION line information; please include it.
%%


