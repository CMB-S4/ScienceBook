%%%%%% CMB-S4 Neutrinos Chapter  %%%%%%%%%%%%%%%%
 
\chapter{Neutrino Physics from the Cosmic Microwave Background}
\renewcommand*\thesection{\arabic{section}}

\def\beq{\begin{equation}}
\def\eeq{\end{equation}}

\def\bea{\begin{eqnarray}}
\def\eea{\end{eqnarray}}

\def\Neff{N_{\rm eff}}
\def\Nf{N_{\rm eff}}
\newcommand{\nucl}[3]{ \ensuremath{ \phantom{\ensuremath{^{#1}_{#2}}} \llap{\ensuremath{^{#1}}} \llap{\ensuremath{_{\rule{0pt}{.75em}#2}}} \mbox{#3} } }


\def\gtrsim{\raise-.75ex\hbox{$\buildrel>\over\sim$}}
%%%%%%%%%%%%%%%%%%%%%%%%%%%%%%%%%%%%%%%%%%%%%%%%%%%%%%%%%%%
%%%%%%%%%%%%%%%%%%%%%%%%%%%%%%%%%%%%%%%%%%%%%%%%%%%%%%%%%%%
%%%%%%%%%%%%%%%%%%%%%%%%%%%%%%%%%%%%%%%%%%%%%%%%%%%%%%%%%%%
%%%%%%%%%%%%%%%%%%%%%%%%%%%%%%%%%%%%%%%%%%%%%%%%%%%%%%%%%%%

\section{Introduction}

Direct interactions between neutrinos and observable matter effectively ceased about one second after the end of inflation.  Nevertheless, the total energy density carried by neutrinos was comparable to other matter sources through today.  As a result, the gravitational effect of the neutrinos is detectable both at the time of recombination and in the growth of structure at later times~\cite{Abazajian:2013oma}, leaving imprints in the temperature and polarization spectrum as well as in CMB lensing.

CMB-S4 can improve our understanding of neutrino physics in regimes of interest for both cosmology and particle physics.  Arguably the most important parameters of interest will be the sum of the neutrino masses $\sum m_\nu$ and the effective number of neutrino species, $\Neff$.  These two parameters have natural targets that are within reach of a CMB-S4 experiment:
\begin{itemize}
\item $ \sum m_\nu \gtrsim 58$ meV is the lower bound guaranteed by observations of solar and atmospheric neutrino oscillations.  A CMB experiment with $\sigma(\sum m_\nu) < 20$ meV would be guaranteed a detection of a least 3$\sigma$.
\item $\Delta \Neff > 0.027$ is predicted for any light particle that was in thermal equilibrium with the standard model.  A CMB experiment producing $\sigma(\Neff) \lesssim 0.01$ would be sensitive to all models in this very broad class of extensions of the Standard Model, which includes a wide range of axions and axion-like particles.  
\end{itemize}
Current CMB data already provides a robust detection of the cosmic neutrino background at $>10 \sigma$.  A CMB-S4 experiment will provide an order of magnitude improvement in sensitivity that opens a new window back to the time of neutrino decoupling and beyond.

Section 2 will review the motivation for studying neutrino masses with cosmological probes, and specifically with the CMB.  We will explain why cosmology is sensitive to $\sum m_\nu$ via different probes and how it is complementary to the experimental neutrino effort.  Section 3 will review the physics of $\Neff$ and its role as probe of the C$\nu$B and as sensitive tool for beyond the Standard model physics.  We will emphasize the unique impact $\Neff$ has on the CMB that makes it distinguishable from other extensions of $\Lambda$CDM.  In Section 4 we will discuss the implications for a variety of well motivated models, including sterile neutrinos and axions.   

\section{Neutrino Mass}

\subsection{Theory Review}

{\it Subsection to be completed by Marilena Loverde}



\subsection{Observational Signatures and Target}
%
Massive neutrinos contribute to the critical density as
\beq
\Omega_\nu h^2 \simeq \frac{\sum m_\nu}{93 \, {\rm eV}} \ .
\eeq
As discussed above, the signature of massive neutrinos manifests through
the energy density $\Omega_\nu$ making its transition from relativistic
radiation to non-relativistic matter as its temperature drops;
the transition occurs at around $z_\mathrm{nr} \sim
2000m_\nu/1\,\mathrm{eV}$.  During the epoc when
neutrinos are relativistic, they free-stream out of the over-dense regions
and washes out the matter fluctuations on small scales.  This effect
turns off roughly at scales larger than the horizon scale at
the redshift $z_\mathrm{nr}$.  Thus, comparison of the amplitudes of the
fluctuations at large and small
scales probes neutrino mass.

The amplitudes at large scales are precisely measured through
primordial CMB fluctuations in both TT and EE power spectra, except for the
uncertainty from the optical depth $\tau$, which we discuss later.
%
The large scale structure (LSS) measures the small scales; this is the
area where we expect significant improvement through CMB S4.

\subsubsection{CMB Power Spectra (TT, EE, Lensing Potential)}
Measurement of the lensing potential of LSS through gravitational
lensing of the CMB provides us the opportunity of precisely measuring
the small-scale amplitudes.  (Refer to lensing section?)
%
Enumerate current measurements.  Mention future prospects.?

Neutrino mass can also be probed through the early Integrated Sachs
Wolfe (ISW) effect in the primordial TT and EE power spectra, since
massive neutrinos changes the expansion history by affecting the
duration of the radiation dominant era.  This measurement is already
cosmic-variance limited and only marginal improvement is possible in
future.  

\subsubsection{SZ Cluster Abundance}
Abundance of galaxy clusters also gives us a measure of the 
amplitudes at small scales ($\simeq 0.1h\mathrm{Mpc}^{-1}$).
%
CMB datasets can be used to measure the abundance through the
Sunyaev-Zel'dvich (SZ) effect.
%
However, the interpretation of this dataset to derive the neutrino mass
 requires an accurate understanding of 
 the non-trivial mass-observable relation.
[Mention mass-observable relation study works in progress ...]


\subsubsection{Cross-correlations with External Datasets}
Redshift tomography?  Calibrating optical surveys to help them measuring
neutrino mass?  (Remember LSST is a DOE project and thus helping them
makes a good case.)


\subsection{Forecasts}

\subsection{Relation to Lab Experiments}

Relation between lab measurements of neutrino masses and the cosmological measurements.

{\it Subsection to be completed by Clarence Chang}

CMB-S4 will explore neutrino physics through precision measurements of the impact of the Cosmic Neutrino Background on the CMB.  There are two primary areas in which CMB-S4 provides interesting neutrino constraints.
 1) The first is using precision measurements of the radiation energy density in the early Universe to constrain neutrino physics beyond the 3-neutrino paradigm. The relativistic energy density in the early Universe is uniquely probed by the CMB and provides a critical constraint on any model for the neutrinos and their interactions. It is a highly complementary probe to terrestrial approaches to non-standard neutrino phenomena (e.g., sterile neutrinos) such as short baseline neutrino experiments. 
 2) The second is constraining the sum of the neutrino masses through precision measurements of the growth of large scale structure. The CMB-S4 mass sensitivity of 16 meV (1-sigma) is especially interesting as it corresponds to a 3-sigma measurement of the minimum predicted value. Fig.~\ref{fig:neutrino-noose} shows the complementarity of the projected CMB-S4 neutrino mass constraints to the neutrino oscillation experiments of the Intensity Frontier (Dune) as well as Katrin mass limits. The oscillation experiments measure the angles and phases of the neutrino mixing matrix and are insensitive to the absolute neutrino mass scale. CMB-S4 measures the sum of the neutrino masses in a manner that is independent of the neutrino oscillation parameters. Results from Intensity Frontier experiments together with CMB-S4 will provide precision measurements of the full neutrino mixing matrix (angles, phases, and masses).
 
 \begin{figure}[ht]
\centering \includegraphics[width=0.55\textwidth]{Neutrinos/numass_combine_dune}
\caption{Shown are the current constraints and forecast sensitivity of
  cosmology to the neutrino mass in relation to the neutrino mass
  hierarchy.  In the case of an ``inverted hierarchy,'' with an
  example case marked as a diamond in the upper curve, the CMB-S4 (with DESI BAO prior)
  cosmological constraints would have a very high-significance
  detection, with $1\sigma$ error shown as a blue band.  In the case
  of a normal neutrino mass hierarchy with an example case marked as
  diamond on the lower curve, CMB-S4 would detect the lowest
  $\sum m_\nu$ at $\gtrsim 3 \sigma$. Also shown is the
  sensitivity from the long baseline neutrino experiment (DUNE) as the
  pink shaded band, which should be sensitive to the neutrino
  hierarchy. Figure adapted from the Snowmass CF5 Neutrino planning document.
 }
\label{fig:neutrino-noose}
\end{figure}



\section{Effective Number of Neutrinos}

\subsection{Theory Review}

{\it Subsection to be completed by Joel Meyers with a contribution from George Fuller}

The angular power spectrum of the cosmic microwave background (CMB) at small angular scales is quite sensitive to the radiation content of the early universe, usually parametrized by a quantity $\Nf$ which will be defined more precisely below.  In the standard models of cosmology and particle physics, $\Nf$ is a measure of the energy density of the cosmic neutrino background.  More generally, however, $\Nf$ receives contributions from all forms of radiation apart from photons which are present in the early universe.  Due to its sensitivity to $\Nf$, the CMB can be used as a tool to probe physics of the standard model and beyond which are difficult to measure through other means.  Here we will give an overview of the theoretical issues related to $\Nf$ and outline the expectations from the standard model.

\subsection{Thermal History of the Early Universe} \label{ThermalHistory}
In this section, we will give a sketch of the thermal history of the standard hot big bang universe when the temperature of the plasma was falling from about $10^{11}$~K to about $10^8$~K following Section 3.1 of \cite{Weinberg:2008zzc}.  For other reviews see \cite{Dolgov:2002wy,Agashe:2014kda}.  During this era, there are two events of particular interest: neutrinos decoupled from the rest of the plasma, and a short time later electrons and positrons annihilated, heating the photons relative to the neutrinos.  Our task is to follow how these events impact the evolution of the energy densities of the photons and neutrinos.

For massless particles described by the Fermi-Dirac or Bose-Einstein distributions, the energy density is given by
\begin{equation}
	\rho(T) =
		\begin{cases}
			g\frac{\pi^2k_B^4}{30\hbar^3 c^3}T^4 \\
			\frac{7}{8}g\frac{\pi^2k_B^4}{30\hbar^3 c^3}
		\end{cases} \, ,
\end{equation}
where $g$ counts the number of distinct spin states.  The entropy density for massless particles is given by
\begin{equation}
	s(T) = \frac{4\rho(T)}{3T} \, .
\end{equation}
It is convenient to define a quantity $\mathcal{N}$ which counts the spin states for all particles and antiparticles, with an additional factor $\frac{7}{8}$ for fermions.  With this definition, the total energy density and entropy density of the universe during radiation domination are given by
\bea
	\rho(T) &=& \mathcal{N}\frac{\pi^2k_B^4}{30\hbar^3 c^3}T^4 \, , \nonumber \\
	s(T) &=& \frac{4}{3}\mathcal{N}\frac{\pi^2k_B^4}{30\hbar^3 c^3}T^3 \, .
\eea
In an expanding universe, the first law of thermodynamics implies that for particles in equilibrium, the comoving entropy density is conserved
\begin{equation}
	a^3s(T) = \mathrm{const} \, .
\end{equation}
One straightforward consequence of this conservation is that for radiation in free expansion, the temperature evolves as the inverse of the scale factor
\begin{equation}
	T\propto \frac{1}{a} \, .
\end{equation}
Let us now apply this to the physics of the early universe.

At a temperature of $10^{11}$~K ($k_BT\sim10$~MeV), the universe was filled with photons, electrons and positrons, and neutrinos and antineutrinos of three species, all in thermal equilibrium with negligible chemical potential, along with a much smaller density of baryons and dark matter both of which are unimportant for the present discussion.  As the temperature of the plasma dropped below about $10^{10}$~K (about 1 second after the end of inflation), the rate of collisions between neutrinos and electrons and positrons could no longer keep up with the expansion rate of the universe, and neutrinos began to fall out of equilibrium and begin a free expansion.  This is just above the temperature for which $m_e c^2 \sim k_B T$, and so for slightly lower temperatures electrons and positrons rapidly disappeared from equilibrium.  We will simplify the discussion by assuming that neutrinos decoupled instantaneously before electron-positron annihilation and comment below how a more detailed calculation modifies the results.  Non-zero neutrino masses can safely be neglected here as long as $m_\nu c^2\lesssim1$~keV which is guaranteed by current observational bounds.

>From this point on, we will distinguish the temperature of neutrinos $T_\nu$ from that of the photons $T_\gamma$.  Before neutrino decoupling, frequent interactions kept neutrinos and photons in equilibrium, ensuring they had a common falling temperature.  After the universe became transparent to neutrinos, the neutrinos kept their relativistic Fermi-Dirac distribution with a temperature which fell as the inverse of the scale factor.  The photons, on the other hand, were heated by the annihilation of the electrons and positrons.  Comoving entropy conservation allows us to compute the relative temperatures at later times.

After neutrino decoupling, but before electron positron annihilation, the thermal plasma contained two spin states of photons, plus two spin states each of electrons and positrons, which means that during this period,
\begin{equation}
	\mathcal{N}_{\mathrm{before}} = 2 + \frac{7}{8}(2+2) = \frac{11}{2} \, .
\end{equation}
After electron positron annihilation, only the two spin states of photons remained, and so
\begin{equation}
	\mathcal{N}_{\mathrm{after}} = 2 \, .
\end{equation}
Since $T_\nu\propto a^{-1}$ during this period, we can express the condition of comoving entropy conservation as follows
\begin{equation}
	\frac{\mathcal{N}_{\mathrm{before}}T_{\gamma,\mathrm{before}}^3}{T_{\nu,\mathrm{before}}^3} = \frac{\mathcal{N}_{\mathrm{after}}T_{\gamma,\mathrm{after}}^3}{T_{\nu,\mathrm{after}}^3} \, .
\end{equation}
Using the fact that $T_{\gamma,\mathrm{before}} = T_{\nu,\mathrm{before}}$, we find as a result
\begin{equation}
	\frac{T_{\gamma,\mathrm{after}}}{T_{\nu,\mathrm{after}}} = \left(\frac{11}{4}\right)^{1/3} \, .
\end{equation}
We find that in the instantaneous neutrino decoupling limit, the annihilation of electrons and positrons raised the photons relative to the neutrinos by a factor of $(11/4)^{1/3}\simeq1.401$.

After electron positron annihilation, assuming three species of light neutrinos and antineutrinos, each with one spin state, the radiation density of the universe is
\begin{equation}
	\rho_r = \frac{\pi^2k_B^4}{30\hbar^3 c^3}\left[2T_\gamma^4 + 3\frac{7}{8}T_\nu^4\right] = \frac{\pi^2k_B^4}{15\hbar^3 c^3}\left[1+3\,\frac{7}{8}\left(\frac{4}{11}\right)^{4/3}\right] T_\gamma^4 \, .
\end{equation}
It is conventional to define a quantity $\Nf$ which gives the radiation energy density in terms of the effective number of neutrino species as
\begin{equation}
	\rho_r = \frac{\pi^2k_B^4}{15\hbar^3 c^3}\left[1+\frac{7}{8}\left(\frac{4}{11}\right)^{4/3}\Nf\right] T_\gamma^4 \, .
\end{equation}
In the instantaneous neutrino decoupling approximation described above, we found $\Nf = 3$.  In the real universe, however, decoupling of neutrinos is not instantaneous, and the residual coupling of neutrinos at the time of electron positron annihilation increases $\Nf$ by a small amount in the standard model.

The current best estimate of $\Nf$ in the standard model is $\Nf = 3.046$ \cite{Mangano:2005cc}.  There is some theoretical uncertainty in this quantity due to the various numerical approximations that are made in the calculation\ldots
\begin{center}
{\color{blue}Placeholder for comments on uncertainty in $\Nf$ in standard model from George Fuller}
\end{center}



{\it Status of calculations in the Standard Model} -- (Contribution expected from G. Fuller)


{\it Light Species}

\begin{figure}[t!]
\begin{center}
\includegraphics[width=0.65\textwidth]{Neutrinos/Neff.pdf}
\caption{Contribution to $\Neff$ from a massless field that was in thermal equilibrium with the Standard model at temperatures $T> T_{\rm freeze}$.  For $T_{\rm freeze} \gg m_{\rm top}$, these curves saturate with $\Delta \Neff > 0.027$.   The region in red shows the range of forecasts for $\sigma(\Neff)$ for plausible CMB-S4 configurations. }
\label{fig:limits}
\end{center}
\end{figure}

\subsection{Dark Radiation}

As will be discussed in more detail below, the CMB angular power spectrum is sensitive to the value of $\Nf$ in our universe.  Measurement of the value of $\Nf$ provides a huge amount of insight into the early universe, and is in fact an observational probe of the conditions at very early times, well before recombination.  Even within the standard model, $\Nf$ provides an observational handle on the thermal history back to about one second after the end of inflation.  The true power of measuring $\Nf$, however, comes from the realization that it is sensitive not just to the neutrinos of the standard model, but it in fact receives contributions from all forms of radiation apart from photons present in the early universe and is thus a probe of new physics.

Collider experiments are known to provide a measurement of the number of neutrino species (or more precisely the number of species of fermions coupling to the $Z$ boson with mass below $m_Z/2$) and find very close agreement with three families of light active neutrinos \cite{ALEPH:2005ab}.  Cosmological measurements of $\Nf$ provide complimentary constraints and are sensitive to the total energy density of radiation whether constituted of active neutrinos or other light species.

If the measured value of $\Nf$ exceeds the standard model prediction, it would be an indication that there is additional radiation content in the early universe or that the thermal history is somehow modified.  Additional radiation which contributes to $\Nf$ is often referred to as dark radiation.  There is a huge number of possible sources for dark radiation, including sterile neutrinos \cite{Abazajian:2001nj,Strumia:2006db,Boyarsky:2009ix}, gravitational waves \cite{Boyle:2007zx,Stewart:2007fu,Meerburg:2015zua}, dark photons \cite{Ackerman:mha,Kaplan:2011yj,CyrRacine:2012fz}, and many more \cite{Cadamuro:2010cz,Weinberg:2013kea}.  It is also possible that the measured value of $\Nf$ could be found below the standard model prediction.  This can happen if for example photons are heated relative to neutrinos after decoupling \cite{Steigman:2013yua,Boehm:2013jpa}.

Furthermore, the CMB power spectrum has the ability to distinguish among certain types of dark radiation based on the behavior of its density perturbations  \cite{Chacko:2015noa,Baumann:2015rya}.  This point will be discussed further below. {\color{blue} (Check that Dan discusses free-streaming in effects on CMB section)}

Even a measurement of $\Nf$ which agrees with the standard model prediction to high precision would be very interesting due to the constraints it would place on physics beyond the standard model.  Any light thermal relic will contribute to $\Nf$ at a level that can be determined by the particle type and the decoupling temperature \cite{Brust:2013ova,Chacko:2015noa}.  The forecasted constraints on $\Nf$ from CMB-Stage IV {\color{blue} (Check for consistency with other sections in naming convention)} are at a level which could constrain all light thermal relics up to temperatures near the electroweak phase transition.  This broad reach to extremely high energies and very early times demonstrates the discovery potential for a precision measurement of $\Nf$ with the CMB.

\subsection{Observational Signatures}

Cosmic neutrinos play to two important roles in the CMB that are measured by $\Neff$.  They contribute to the total energy in radiation which controls the expansion history and, indirectly, the damping tail.  The fluctuations of neutrinos and any other free streaming radiation also produces a constant shift in the of the acoustic peaks.  These two effects drive both current and future constraints on $\Neff$.  

The effect of neutrinos on the damping tail has dominated has historically driven the constraint on $\Neff$ in the CMB.  The largest effect is from the mean-free path of photons, which introduces a suppression $e^{-(k/k_d)^2}$ to short wavelengths, with~\cite{Zaldarriaga:1995gi}
\beq
k_d^{-2} =\int \frac{da}{a^3 \sigma_T n_e H} \frac{R^2+ \frac{16}{15}(1+R)}{6(1+R)^2} \ ,
\eeq
where $R$ is the ratio of the energy in baryons to photons, $n_e$ is the density of free electrons and $\sigma_T$ is the Thompson cross-section.  The damping scale is sensitive to the energy in all radiation through $H \propto \sqrt{\rho_{\rm radiation}}$ during radiation domination (which is applicable at high $\ell$, and is therefore sensitive to $\Neff$ or any form of dark radiation.  At this level, this also illustrates the degeneracy with $n_e$ which may be altered by the helium fraction, $Y_p$.

In reality, the effect on the damping tail is actually subdominant to the change to the scale of matter radiation equality and the location of the first acoustic peak~\cite{Hou:2011ec}.  As a result, the effect on neutrinos on the damping tail is more accurately represented by holding the first acoustic peak fixed.  This changes the sign of the effect on the damping tail, but the intuition for the origin of the effect (and degeneracy) remain applicable.

In addition to the effect on the Hubble expansion, perturbations in neutrinos effect the photon-baryon fluid through their gravitational influence.  The contributions from neutrinos are well described by a correction to the amplitude and the phase of the acoustic peaks in both temperature and polarization~\cite{Bashinsky:2003tk}.  The phase shift is a particularly compelling signature as it is not degenerate with other cosmological parameters~\cite{Bashinsky:2003tk,Baumann:2015rya}.  This effect is the result of the free-streaming nature of neutrinos that allows propagation speeds of effectively the speed of light (while the neutrinos are relativistic).  This effect is sensitive to any gravitationally coupled light thermal relics.

E-mode polarization will play an increasingly important role for several reasons.  First of all, the acoustic peaks are sharper in polarization which makes measurements of the peak locations more precise, and therefore aid the measurement of the phase shift.  The second reason is that polarization breaks a number of degeneracies that would also affect the damping tail~\cite{Baumann:2015rya}.

{\it Status of current observations} -- Planck has provided a strong constraint on $\Neff = 3.15 \pm 0.23$ when combining both temperature and polarization data.  The addition of polarization data has both improved the constraint on $\Neff$ and reduced the impact of the degeneracy with $Y_p$.  Recently, the phase shift from neutrinos has also been established direction in the Planck temperature data~\cite{Follin:2015hya}.  This provides the most direct evidence for presence of free-steaming radiation in the early universe, consistent with the cosmic neutrino background.


\subsection{Forecasts}




\section{Sterile Neutrinos and Axions}
\subsection{Sterile Neutrinos}

{\it Subsection to be completed by Kevork Abazajian}

A number of recent neutrino oscillation experiments have anomalies
that are possible indications of four or more mass eigenstates. The
first set of anomalies are in short baseline oscillation experiments:
first, in the Liquid Scintillator Neutirno Detector (LSND) experiment,
where electron antineutrinos were observed in a pure muon antineutrino
beam \cite{Athanassopoulos:1997pv}. The MiniBooNE Experiment also sees
an excess of electron neutrinos and antineutrinos in their muon
neutrino beam \cite{Aguilar-Arevalo:2013pmq}. Two-neutrino oscillation
interpretations of the results indicate mass splittings of $\Delta m^2
\approx 1\rm\ eV^2$ and mixing angles of $\sin^2 2\theta \approx
3\times 10^{-3}$ \cite{Aguilar-Arevalo:2013pmq}. Another anomaly
arises from re-evaluations of reactor antineutrino fluxes that
indicate a an increased flux of antineutrinos as well as a lower
neutron lifetime and commensurately increase the antineutrino events
from nuclear reactors by 6\%. This brought previous agreement of
reactor antineutrino experiments to have a $\sim$6\% deficit
\cite{Mention:2011rk,Huber:2011wv}. Another indication consistent with
sterile neutrinos are in radio-chemical gallium experiments for solar
neutrinos. In their callibrations, a 5-20\% deficit of the measured
count rate was found when intense sources of electron neutrinos from
electron capture nuclei were placed in proximity to the
detectors. Such a deficit could be produced by a $>1\rm\ eV$ sterile
neutrino with appreciable mixing with electron neutrinos
\cite{Bahcall:1994bq,Giunti:2010zu}. Some simultaneous fits to the
short baseline anomalies and reactor neutrino deficits, commensurate
with short baseline constraints, appear to prefer at least two extra
sterile neutrino states \cite{Conrad:2012qt,Kopp:2013vaa}, but see
Ref.~\cite{Giunti:2015mwa}. Because such neutrinos have relatively
large mixing angles, they would be thermalized in the early universe
with a standard thermal history, and affect primordial nucleosynthesis
\cite{Abazajian:2002bj}. 

In addition, there are combinations of CMB plus LSS
datasets that are in tension, particularly with a smaller amplitude of
fluctuations at small scale than that inferred in zero neutrino mass
modesl. This would be alleviated with the
presence of massive neutrinos, extra neutrinos, or both. In particular,
cluster abundance analyses \cite{Wyman:2013lza,Ade:2015fva} and weak lensing analyses
\cite{Battye:2013xqa} indicate a lower amplitude of
fluctuations than zero neutrino mass \cite{Giusarma:2014zza}0. Baryon Acoustic
Oscillation measures of expansion history are affected by the presence
of massive neutrinos, and nonzero neutrino mass may be indicated 
\cite{Beutler:2014yhv}, though 2015 Planck results show a lack of such
alleviation in cases with massive or extra neutrinos
\cite{Ade:2015xua}. 

There is a potential emergance of both laboratory and cosmological
indications of massive and, potentially, extra neutrinos. However, the
combined requirements of the specific masses to produce the short
baseline results, along with mixing angles that require thermalized
sterile neutrino states, are inconsistent at this point with
cosmological tension data sets
\cite{Joudaki:2012uk,Archidiacono:2013xxa}. The tension data sets are
not highly significant at this point ($\lesssim 3\sigma$), and there
are a significant set of proposals for short baseline oscillation
experiment follow up \cite{Abazajian:2012ys}. Future high-sensitivity
probes of neutrino mass and number such as CMB-S4 will be able to
definitively test for the presence of extra neutrino number and mass
consistent with sterile neutrinos.


\subsection{Axion-like Particles}

{\it Subsection to be completed by Dan Green}

A ubiquitous component of extensions of the Standard model is axions and/or axion-like particles (ALPs).  Axions have been introduced to solve the strong-CP problem~\cite{Peccei:1977hh}, the hierarchy problem~\cite{Graham:2015cka} and the naturalness of inflation~\cite{Freese:1990rb}.  Furthermore, they appear generically in string theory, in large numbers, leading to the qualitative phenomena describe as the string axiverse~\cite{Arvanitaki:2009fg}.

ALPS typically appear as (pseudo)-Goldstone bosons of some high energy global symmetry.  At low energy, the mass of the ALP is protected by an approximate shift symmetry of the general form $a \to a + c$ where $a$ is the axion and $c$ is a constant (for non-abelian Goldstone bosons, this transformation will include higher order terms in $a$).  We will define an ALP to be any such particle where all of the couplings of the axion to the Standard model respect such a symmetry.  This symmetry may be softly broken with an explicit mass term, although this is highly restricted in the case of the QCD axion.

Two couplings of particular interest for axion phenomenology are the coupling to gluons and photons, 
\beq
\frac{1}{4} g_{a \gamma \gamma} a \tilde F_{\mu \nu}F^{\mu\nu} \ , \qquad \qquad \frac{1}{4} g_{a g g} a \tilde G_{\mu \nu}G^{\mu\nu}  \ .
\eeq
These couplings typically appear as the consequence of chiral anomalies.  The coupling of the axion to gluons is what makes the solution to the strong-CP problem possible.  The coupling to photons is somewhat model dependent but typically arises in conjunction with the gluon coupling.  In addition or instead of these coupling, a variety of possible couplings to matter may also be included.

Two very common features of these models is that the axions as typically light (in many cases, $m \ll 1$ eV) and their interactions are suppressed by powers of $f_a$.  These two features make ALPS are particularly compelling target.  Because of the small masses, they will often behave as relativistic species in the CMB.  Furthermore, because their production rate will scale as $T^{2n +1} / f_a^{2n}$ for some $n \geq 1$, they are likely to be thermalized at high temperatures.  Given that $\Delta \Neff > 0.027$ under such circumstances, a CMB experiment with sensitivity at this level will be sensitive to a very wide range of ALP models.

{\it Status of current observations} -- Current constrains on ALPs arise from a combination of experimental~\cite{Graham:2015ouw}, astrophysical~\cite{Raffelt:2012kt} and cosmological~\cite{Marsh:2015xka} probes.  Current cosmological constraints are driven by several effects that depend on the mass of the axion.  For axion masses greater than 100 eV, stable thermal ALPs are easily excluded because they produce dark matter abundances inconsistent with observations.  By including the free streaming effects of thermal QCD-axions,  Planck data~\cite{DiValentino:2015wba} combined with local measurements provide the constrain $m_a < 0.525$ eV (95 \% CL).  At larger masses, ALPs become unstable and can be constrained by the change to $\Neff$ from energy injection as well as from spectral distortions and changes to BBN~\cite{Cadamuro:2011fd,Follin:2015hya}

{\it Implications for CMB Stange IV} -- Sensitive to $\Neff$ of order $\sigma(\Neff) \simeq 10^{-2}$ has the sensitive to probe then entire mass range of ALPs down to $m_a =0$ under the assumption that it thermalized in the early universe.  Interpreting such bounds in terms of the couplings of axions is more complicated~\cite{Brust:2013xpv} and can depend on assumptions about the reheating temperature.  For high (but plausible) reheat temperatures of $10^{10}$ GeV, CMB stage IV would be sensitive to $g_{a\gamma \gamma}, g_{a g g} \gtrsim 10^{-13} {\rm GeV}^{-1}$~\cite{}, which exceeds current constraints and future probes for a range possible axions masses (including the QCD axion).  

\section{Complementarity of CMB and BBN}

\subsection{Introduction} \label{Introduction}

Primordial light element abundances have historically been an interesting observational test of hot big bang cosmology.  The process by which light elements form in the early universe known as big bang nucleosynthesis (BBN) was worked out theoretically in the early days of the development of the hot big bang model of cosmology \cite{Alpher:1948ve}.  It is a process which depends on all four fundamental forces, that unfolded during the first three minutes of our current phase of expansion, and which has long provided a useful constraint on physics beyond the standard model.  The current observational limits on primordial abundances overall show good agreement with the predictions of standard BBN.  Primordial light element abundances are sensitive to the radiation content of the universe as measured through $\Nf$, which also affects the angular power spectrum of the CMB.  Combining these two probes provides useful insight into the physics of the early universe which neither could achieve alone.

\subsection{Standard Big Bang Nucleosynthesis} \label{StandardBBN}
In this section, we will briefly review the physics of big bang nucleosynthesis in the standard model.  For more extensive reviews see for example \cite{Weinberg:2008zzc,Agashe:2014kda,Cyburt:2015mya}.

At temperatures above $k_BT\sim 1$~MeV, weak interactions kept neutrons and protons in thermal equilibrium, fixing their number densities to have the ratio $n_n/n_p = e^{-Q/k_BT}$, where $Q = 1.293$~MeV is the mass difference between neutrons and protons.  At lower temperatures, interactions which convert protons to neutrons could not keep up with the expansion rate, leaving free neutron beta decay as the only channel by which protons and neutrons interconverted.  The initial ratio of their number densities at the freeze-out temperature $k_BT_\mathrm{fr}\simeq0.8$~MeV was therefore
\begin{equation}
	n_n/p_n = e^{-Q/k_BT_\mathrm{fr}} \simeq 1/5 \, .
\end{equation}


After freeze-out, neutrons decayed until becoming bound into nuclei.  This process proceeded primarily through two-body processes, starting with the formation of deuterium.  The very small number density of baryons compared to that of photons (parametrized through $\eta\equiv n_b/n_\gamma$) delayed the start of these nuclear reactions until well after the temperature dropped below the binding energy of deuterium due to photo-dissociation of deuterium.  The condition of the onset of deuterium formation is set by requiring that the number of photons per baryon with energy above the binding energy of deuterium drops below unity
\begin{equation}
	\eta^{-1}e^{-|B_D|/k_BT_D} \simeq 1 \, .
\end{equation}
With the deuterium binding energy given by $|B_D|=2.23$~MeV, and $\eta\sim6\times10^{-10}$, we find that deuterium begins to form when the temperature drops below about $k_BT_D\simeq 0.1$~MeV.  By this time, due to free neutron decay, the neutron to proton ratio had dropped to about $n_n/n_p\simeq 1/7$.  Once deuterium was able to form, nearly all of the neutrons quickly became bound into the most energetically favorable light nucleus, which is $\nucl{4}{ }{He}$.  We can estimate the mass fraction of primordial $\nucl{4}{ }{He}$, $Y_p\equiv \frac{\rho\left(\nucl{4}{ }{He}\right)}{\rho_b}$ to be
\begin{equation}
	Y_p = \frac{2(n_n/n_p)}{1+n_n/n_p} \simeq 0.25 \, .
\end{equation}

In addition to $\nucl{4}{ }{He}$, BBN produces a small amount of $\nucl{ }{ }{D}$, $\nucl{3}{ }{He}$, $\nucl{6}{ }{Li}$, and $\nucl{7}{ }{Li}$ (and also $\nucl{7}{ }{Be}$ which subsequently decays by electron capture to $\nucl{7}{ }{Li}$).  While $Y_p$ is primarily sensitive to the neutron lifetime, the primordial abundances of the other light elements depend in complicated ways various nuclear rates and generally require numerical computation (see for example \cite{Wagoner:1966pv,Cyburt:2001pp,Pisanti:2007hk}).

Standard BBN is a one parameter model, depending only on the baryon to photon ratio $\eta$.  The theory predicts several abundances which can be used to fix $\eta$ and check the consistency of the theory, or alternatively, to constrain new physics.  Current observations agree quite well with the predictions of standard BBN, with the exception of $\nucl{7}{ }{Li}$.  It is unclear whether this disagreement points to a problem with the astrophysical determination of the primordial abundance or a problem with the standard theory.  The cosmological lithium problem remains unsolved \cite{Fields:2011zzb}.  From here on, however, we will ignore the lithium problem and focus on how measurements of the other abundances (primarily $\nucl{ }{ }{D}$ and $\nucl{4}{ }{He}$) can be used to constrain the physics of the early universe.




%-----------------------------------------------




\subsection{Beyond the Standard Model}\label{BSM}
Moving beyond standard BBN, measurements of primordial abundances have the ability to constrain many deviations from the standard thermal history and the standard model of particle physics.  Because BBN is sensitive to all fundamental forces, changes to any force can in principle impact light element abundances.  Of primary interest for our purpose is that BBN is sensitive to the expansion rate between about one second and a few minutes after the end of inflation.  The expansion rate is in turn determined by the radiation content of the universe during this period, and thus BBN is sensitive to $\Nf$.  

More specifically, the expansion rate determines the freeze-out temperature setting the initial ratio of neutrons to protons and the amount of time free neutrons have to decay.  Additional radiation compared to the standard model gives a higher expansion rate, which leads to a higher freeze-out temperature and less time for free neutron decay, leading to a larger primordial $\nucl{4}{ }{He}$ abundance.  The freeze-out temperature also depends weakly on the distribution function of electron neutrinos, though this is subdominant to the dependence $\Nf$ for small non-thermal distortions \cite{Serpico:2004gx}.

Historically, $Y_p$ had provided the best constraint on $\Nf$.  Recent advancements in the determination of primordial deuterium abundance have made constraints on $\Nf$ from deuterium competitive with those from $Y_p$ \cite{Cooke:2013cba}.  The precision with which primordial abundances constrain $\Nf$ is now comparable to that of constraints the CMB power spectrum, and there is no evidence for deviation from the standard model \cite{Ade:2015xua}.



%-----------------------------------------------




\subsection{Complementarity with the CMB}\label{Complementarity}
The CMB can be used to quite precisely constrain $\eta$ by measurement of the baryon fraction of the critical density, which is related to $\eta$ by
\begin{equation}
	\Omega_b h^2 \simeq \frac{\eta\times10^{10}}{274} \, .
\end{equation}
Using the value of $\eta$ determined from CMB measurements as an input for BBN makes standard BBN a theory without free parameters which agrees very well with all observations (apart from the aforementioned disagreement with the observed lithium abundance).  The CMB and BBN are sensitive to the baryon density measured at different times.  While BBN is sensitive to the baryon to photon ratio up to a few minutes after the end of inflation, the CMB is sensitive to the baryon density at much later times, closer to recombination about 380,000 years later.  Combining constraints from BBN and CMB on the baryon fraction therefore allows constraints on models where the photon or baryon density changes between these times.

The precision with which the CMB can constrain $\Nf$ will soon come to surpass the constraints from BBN, but the value of the latter will not be totally eclipsed.  BBN and the CMB probe the physics at different times, and so combining constraints can give insight into models where $\Nf$ changes in time.  If it were measured for example that $\Nf^{\mathrm{BBN}}<\Nf^{\mathrm{CMB}}$, this could be explained by the late decay of some unstable particles \cite{Fischler:2010xz,Menestrina:2011mz,Hooper:2011aj}.  Alternatively, if observations revealed that $\Nf^{\mathrm{BBN}}>\Nf^{\mathrm{CMB}}$, this might signal late photon heating \cite{Cadamuro:2010cz,Millea:2015qra}.

The power spectrum of the CMB is also directly sensitive to $Y_p$.  Since helium recombines earlier than hydrogen, the density of helium present at the time of recombination affects the free electron density, and thereby affects the damping tail of the CMB (though in a way which can be distinguished from the effects of $\Nf$) \cite{Bashinsky:2003tk,Hou:2011ec,Follin:2015hya,Baumann:2015rya}.  The degeneracy between $Y_p$ and $\Nf$ is more strongly broken with precise CMB polarization data.

CMB-Stage IV {\color{blue} (Check for consistent notation)} will provide constraints on $\Nf$ which are about an order of magnitude better than the current best constraints, and will also improve on the measurement of $Y_p$ by about a factor of two compared to the current best astrophysical measurements.  Combined with measurements of other primordial abundances, this will help to provide a very thorough check of our understanding of the early universe and provide the opportunity to discover physics beyond the standard model.






\bibliography{cmbs4}

%%
%% Populate the .bib file with entries from SPIRES Bibtex (preferred)
%% or ADS Bibtex (if no SPIRES entry).
%%  SPIRES will also supply the CITATION line information; please include it.
%%


