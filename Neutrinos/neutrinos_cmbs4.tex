%%%%%% CMB-S4 Neutrinos Chapter  %%%%%%%%%%%%%%%%
 
\chapter{Neutrino Physics from the Cosmic Microwave Background}
\renewcommand*\thesection{\arabic{section}}

\def\beq{\begin{equation}}
\def\eeq{\end{equation}}

\def\bea{\begin{eqnarray}}
\def\eea{\end{eqnarray}}

\def\Neff{N_{\mathrm eff}}
\def\gtrsim{\raise-.75ex\hbox{$\buildrel>\over\sim$}}
%%%%%%%%%%%%%%%%%%%%%%%%%%%%%%%%%%%%%%%%%%%%%%%%%%%%%%%%%%%
%%%%%%%%%%%%%%%%%%%%%%%%%%%%%%%%%%%%%%%%%%%%%%%%%%%%%%%%%%%
%%%%%%%%%%%%%%%%%%%%%%%%%%%%%%%%%%%%%%%%%%%%%%%%%%%%%%%%%%%
%%%%%%%%%%%%%%%%%%%%%%%%%%%%%%%%%%%%%%%%%%%%%%%%%%%%%%%%%%%

\section{Introduction}

Direct interactions between neutrinos and observable matter effectively ceased about one second after the end of inflation.  Nevertheless, the total energy density carried by neutrinos was comparable to other matter sources through today.  As a result, the gravitational effect of the neutrinos is detectable both at the time of recombination and in the growth of structure at later times~\cite{Abazajian:2013oma}.


\section{Neutrino Mass}

\subsection{Theory Review}



\subsection{Observational Signatures and Target}


\subsection{Forecasts}

\subsection{Relation to Lab Experiments}

Relation between lab measurements of neutrino masses and the cosmological measurements.

\section{Effective Number of Neutrinos}

\subsection{Theory Review}

The {\it effective} number of neutrinos defined to be
\beq
\Neff=\frac{8}{7} \left(\frac{11}{4}\right)^{4/3} \frac{\rho_{R}-\rho_\gamma}{\rho_\gamma}  \ ,
\eeq
where $\rho_{R}$ is the total energy density in radiation and $\rho_\gamma$ is the energy density in photons. With this definition, the Stanford thermal history and particle content produce $\Neff = 3.046$.

\subsection{Observational Signatures and Target}

\subsection{Forecasts}

\subsection{Thermal History and Big Bang Nucleosynthesis}

\section{Other Targets}

\subsection{Axion-like Particles}

\subsection{Decaying Massive Particles}


\bibliography{cmbs4}

%%
%% Populate the .bib file with entries from SPIRES Bibtex (preferred)
%% or ADS Bibtex (if no SPIRES entry).
%%  SPIRES will also supply the CITATION line information; please include it.
%%


