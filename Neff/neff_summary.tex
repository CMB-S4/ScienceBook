New light particles appear in many attempts to understand both the observed laws of physics and extensions to higher energies.  These light particles are often deeply tied to the underlying symmetries of nature and can play crucial roles in understanding some of the great outstanding problems in physics.  In most cases, these particles interact too weakly to be produced at an appreciable level in Earth-based experiments, making them experimentally elusive.  At the very high temperatures believed to be present in the early Universe, however, even extremely weakly coupled particles can be produced prolifically and can reach thermal equilibrium with the Standard Model particles. Light particles (masses less than $0.1$ eV) produced at early times survive until the time when the CMB is emitted and direct observations become possible.  Neutrinos are one example of such a relic found in the Standard Model.  Extensions of the Standard Model also include a wide variety of possible light relics including axions, sterile neutrinos, hidden photons, and gravitinos.  As a result, the search for light relics from the early Universe with CMB-S4 can shed light on some of the most important questions in fundamental physics, complementing existing collider searches and efforts to detect these light particles in the lab.  
 
Light relics contribute to the total energy density in radiation in the Universe during the radiation era and significantly alter the appearance of the CMB at small angular scales (high multipole number $\ell$). The energy density in radiation controls both the expansion rate of the Universe at that time and the fluctuations in the gravitational potential in which the baryons and photons evolve.  Through these effects, CMB-S4 can provide an exquisite measurement of the total energy density in light weakly-coupled particles, often parametrized by the quantity $\Neff$.  Any additional light particle that decoupled from thermal equilibrium with the Standard Model produces a change to the density equivalent to $\Delta \Neff \geq 0.027$ per effective degree of freedom of the particle.  This is a relatively large contribution to the radiation density that arises from the democratic population of all species during thermal equilibrium.  Realistic configurations of CMB-S4 can reach $\sigma(\Neff) \sim 0.02-0.03$, which will test the minimal contribution of any new spinless light relic at $1\sigma$ and at 2$\sigma$ for any particle with non-zero spin.  $\Neff$ is a unique measurement to cosmology, and it is likely that these thresholds can only be reached by observing the CMB with the angular resolution and sensitivity attainable by a CMB-S4 experiment.   

