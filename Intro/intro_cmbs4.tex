%%%%%% CMB-S4 Introduction %%%%%%%%%%%%%%%%
 
\chapter{Exhortations}
\label{chap:intro}

%%%%%%%%%%%%%%%%%%%%%%%%%%%%%%%%%%%%%%%%%%%%%%%



\bigskip

Fourteen billion years ago, in the first fraction of a second of our Universe's existence, the most extreme high-energy physics experiment took place. The ability to use the cosmic microwave background (CMB) to investigate this fantastic event, at energy scales as much as a trillion times higher than can be obtained at CERN, is at the very core of our quest to understand the fundamental nature of space and time and the physics that drive the evolution of the Universe. 

The CMB allows direct tests of models of the quantum mechanical origin of all we see in the Universe. Subtle correlations in its anisotropy imparted by the interplay of gravitational and quantum physics at high energies contain information on the unification of gravity and quantum physics. Separately, correlations induced on the background at later times encode details about the distribution of all the mass, ordinary and dark, in the Universe, as well as the properties of the neutrinos, including the number of neutrino species and types, and their still unknown masses. 

The purpose of this book is to set the scientific goals to be addressed by the next generation ground-based cosmic microwave background experiment, CMB-S4, consisting of dedicated telescopes at the South Pole, the high Chilean Atacama plateau and possibly a northern hemisphere site, all equipped with new superconducting cameras. CMB-S4 
is envisioned to be the definitive CMB experiment. It will enable a dramatic leap forward in cosmological studies by crossing critical thresholds in testing inflation, in the number and masses of the neutrinos, in finding possible new light relics, in  constraining the nature of dark energy, and in testing general relativity on large scales. 


We begin this chapter with a brief history and the current status of CMB measurements and cosmological results.  This is followed by a general overview of how the CMB-S4 science goals, as outlined in the executive summary, lead to general aspects of the instrument design. Based on these considerations, we present a strawman configuration that serves as an initial jumping-off point for exploring instrument configuration parameter space in the following science chapters.  Lastly, this chapter provides a brief overview of the path from the ongoing Stage-3 experiments to realizing CMB-S4. 

\section{Brief History and Current Status of CMB measurements}
\label{sec:background}

Since the discovery of the cosmic microwave background (CMB) 50 years ago, CMB measurements have led to spectacular insights into the fundamental workings of space and time, from the quantum mechanical origin of the Universe at extremely high energies through the growth of structure and the emergence of the dark energy that now dominates the energy density of the Universe. Studies of the CMB connect physics at the smallest scales and highest energies with the largest scales in the Universe, roughly 58 orders of magnitude in length scale. They connect physics at the earliest times to the structure that surrounds us now, over 52 magnitudes in time scale. 

The deep connections of CMB studies and particle physics predate the discovery of the background, going back to the 1940s when Alpher and Gamow were considering a hot, dense, early Universe as a possible site for nucleosynthesis. To produce the
amount of helium observed in the local Universe, they concluded there
had to be about $10^{10}$\ thermal photons for every nucleon. Alpher and Herman subsequently predicted that this background of photons would persist to the present day as a thermal bath at a few degrees Kelvin.

The continuing, remarkably successful, story of CMB studies is one driven by the close interplay of theory and phenomenology with increasingly sensitive and sophisticated experiments. The high degree of isotropy of the CMB across the sky, to nearly a part of one in a hundred thousand, led to the theory of inflation and cold dark matter in the 1980's. It was not until 1992 that instruments aboard the \cobe\ satellite led to the discovery of the anisotropy, and pinned the level of anisotropy for the following higher angular resolution measurements to characterize. In 2006 the \cobe\ measurements of the background anisotropy and its black-body spectrum were recognized with the second Nobel Prize for CMB research; the first was awarded in 1978 to Penzias and Wilson for the discovery of the CMB.  In the decade after the \cobe\ results, measurements with ground and balloon-based instruments revealed the acoustic peaks in the CMB angular power spectrum, which showed that the Universe was geometrically flat in accordance with predictions of inflation and provided strong support for contemporary claims for an accelerating Universe based on observations of type Ia supernovae (SNe), which were recognized with the 2011 Nobel Prize in physics. These early anisotropy measurements also provided an estimate of the universal baryon density and found it to be in excellent agreement with the level estimated at $t \sim 1$ second by big bang nucleosynthesis (BBN) calculations constrained to match the observed elemental abundances. The CMB measurements also clearly showed that dark matter was non-baryonic. The polarization anisotropy was discovered ten years after \cobe\ at the level predicted from temperature anisotropy measurements. The now standard $\Lambda$CDM cosmological model was firmly established.

Two CMB satellites have mapped the entire sky since \cobe, first \wmap\ with moderate angular resolution (as fine as 12 arcminutes), followed by \planck\ with resolution as fine as 5 arcminutes. Higher-resolution maps of smaller regions of the sky have been provided by ground-based experiments, most notably by the 10m South Pole Telescope (SPT) and the 6m Actacama Cosmology Telescope (ACT). The primary CMB temperature anisotropy is now well characterized through the damping tail, i.e., to multipoles $\ell \sim 3000$, and secondary anisotropies have been measured to multipoles up to  $10,000$.  Figure~\ref{fig:CurrentCMB} shows the current state of the temperature and polarization anisotropy measurements and the expected improvements with CMB-S4.


The $\Lambda$CDM model continues to hold up stunningly well, even as the precision of the CMB determined parameters has increased substantially. Inflationary constraints include limits on curvature constrained to be less than 3\% of the energy density,
departures from Gaussianity are bounded at the level of 1 part in $10^4$, 
and 
the predicted small departure from pure scale invariance of the primordial fluctuations is detected at 5-sigma confidence. Also of interest to particle physics, the effective number of light relativistic species (i.e., neutrinos and any yet identified ``dark radiation") is shown to be within 10\% of $\neff = 3.046$, the number predicted by BBN.  The sum of the masses of the neutrinos is found to be less than 0.6 eV. Dark matter is shown to be non-baryonic at $> 40$ sigma. Early dark energy models are highly constrained as are models of decaying dark matter. 

\begin{figure}[t]
\centering \includegraphics[width=0.9\textwidth]{Intro/CurrentCMB_withcmbs4.pdf}
\caption{Current measurements of the angular power spectrum of the CMB temperature and polarization anisotropy. The horizontal axis is scaled logarithmically in multipole $\ell$ left of the vertical dashed line ($\ell < 30$) and as $\ell^{0.6}$ at higher multipole.  Best-fit models of residual foregrounds plus primary CMB anisotropy power for TT datasets are also plotted. To illustrate the expected improvements with CMB-S4, the projections for a strawman instrumental configuration are shown in grey (binned with $\Delta\ell = 5$ for TT and EE spectra and $\Delta\ell = 30$ for BB) for a $\Lambda$CDM with $r =0$ cosmological model.}
\label{fig:CurrentCMB}
\end{figure}

There remains much science to extract from the CMB, including: 1) using CMB B-mode polarization to search for primordial gravitational waves to constrain the energy scale of inflation and to test alternative models, and to provide insights into quantum gravity; 2) obtaining sufficiently accurate and precise determinations of the effective number of light relativistic species (dark radiation) to search for new light relics, and to allow independent and rigorous tests of BBN and our understanding of the evolution of the Universe at $t = 1$\ sec; 3) a detection of the sum of the neutrino masses, even at the minimum mass allowed by oscillation experiments and in the normal hierarchy; 4) using secondary CMB anisotropy measurements to provide precision tests of dark energy through its impact on the growth of structure; and 5) testing general relativity and constraining alternate theories of gravity on large scales.

The best cosmological constraints come from analyzing the combination of primary and secondary CMB anisotropy measurements with other cosmological probes, such as baryon acoustic oscillations (BAO) and redshift space distortions, weak lensing, galaxy and galaxy cluster surveys, Lyman-alpha forest measurements, local determinations of the Hubble constant, observations of type Ia SNe, and others. The CMB primary anisotropy measurements provide highly complementary data for the combined analysis;  by providing a precision measurement of the Universe at $z = 1100$, the CMB data leads to tight predictions for measurements of the late time Universe for any adopted cosmological model---measurements of the Hubble constant, the BAO scale, and the normalization of the present day matter fluctuation spectrum being excellent examples. Secondary CMB measurements provide late-time probes directly from the CMB measurement, e.g., CMB lensing, the SZ effects and SZ cluster catalogs, which will provide critical constraints on the standard cosmological models and extensions to it. The cosmological reach of future cosmological surveys at all wavelengths will be greatly extended by their joint analyses with secondary CMB anisotropy measurements. 



\section{CMB-S4 Design Considerations}

The CMB-S4 science goals, as outlined in the executive summary, and detailed in the following chapters, lead to several general aspects of the instrument design. We briefly summarize the general design considerations below.

\subsection{Raw sensitivity considerations and detector count}

\begin{figure}[t]
\centering \includegraphics[width=0.6\textwidth]{Intro/expt_progress.pdf}
\caption{Plot illustrating the evolution of the raw sensitivity of CMB
  experiments, which scales as the total number of
  bolometers. Ground-based CMB experiments are classified into Stages
  with Stage II experiments having $O$(1000) detectors, Stage III
  experiments having $O$(10,000) detectors, and a Stage IV experiment
  (such as \cmbexp) having $O$(100,000) detectors. Figure from Snowmass  CF5
  Neutrino planning document.}
\label{fig:expt_progress-intro}
\end{figure}

The sensitivity of CMB measurements has increased enormously since Penzias and Wilson's discovery in 1965, following a Moore's Law like scaling, doubling every roughly 2.3 years. Fig.~\ref{fig:expt_progress-intro} shows the sensitivity of recent experiments, expectations for upcoming Stage-3 experiments, characterized by order 10,000 detectors on the sky, and the projection for a Stage 4 experiment with order 100,000 detectors. To obtain many of the CMB-S4 science goals requires of order $1~\mu$K arcminute sensitivity over roughly half of the sky, which for a four-year survey requires of order 500,000 CMB-sensitive detectors. 

To maintain the Moore's Law-like scaling requires a major leap forward, a phase change in the mode of operation of the ground based CMB program.  Two constraints drive the change:  1) CMB detectors are background-limited, so more pixels are needed on the sky to increase sensitivity; and 2) the pixel count for existing CMB telescopes are nearing saturation.  Even using multichroic pixels and wide field of view optics, these CMB telescopes are expected to field only tens of thousands of polarization detectors, far fewer than needed to meet the CMB-S4 science goals. 

CMB-S4 thus requires multiple telescopes, each with a maximally outfitted focal plane of pixels utilizing superconducting, background limited, CMB detectors. To achieve the large sky coverage and to take advantage of the best atmospheric conditions, the South Pole and the Chilean Atacama sites are baselined, with the possibility of adding a new northern site to increase sky coverage to the entire sky not contaminated by prohibitively strong Galactic emission.

\subsection{Degree angular scale (low $\ell$) sensitivity}

At the largest angular scales (low $\ell$)---the angular scales that must be measured well to pursue inflationary 
B modes---the CMB polarization anisotropy is highly contaminated by foregrounds. Galactic synchrotron dominates at low frequencies and galactic dust at high frequencies, as recently shown by the \planck\ and \planck/BICEP/KECK polarization results. Multi-band polarization measurements are required to distinguish the primordial polarized signals from the foregrounds. 

Adding to the complexity of low multipole CMB observations is the need to reject the considerable atmospheric noise contributions over the large scans needed to extract the low 
$\ell$ polarization. While the spatial and temporal fluctuations of the atmosphere are not expected to be polarized, any mismatches in the polarized beams or detector gains will lead to systematic contamination of the measured polarization by the much stronger unpolarized signal, referred to as T-P leakage. These issues can be mitigated by including additional modulations into the instrument design, such as bore-sight rotation or modulation of the entire optics with a polarization modulation scheme in front of the telescope. Implementing such modulations is easier for small telescopes, although they could in principle be implemented on large telescopes as well. The cost of a small aperture telescope is dominated by the detectors, making it feasible to deploy multiple telescopes each optimized for a single band, or perhaps multiple bands within the relatively narrow atmosphere windows.

It is therefore an attractive option for CMB-S4 to include dedicated small aperture telescopes for pursuing low-$\ell$ polarization. The default plan for CMB-S4 is to target the recombination bump, with E-mode and B-mode polarization down to $\ell \sim 20$. If Stage 3 experiments demonstrate that it is feasible to target the reionization bump at $\ell < 20$ from the ground, those techniques may be incorporated to extend the reach of CMB-S4. More likely, however, this is the $\ell$ range for which CMB-S4 will be designed to be complementary to balloon-based and satellite based measurements. 

\subsection{Subdegree angular scale (high $\ell$) sensitivity}

At the highest angular resolution (high $\ell$)---the angular scales needed for de-lensing the inflationary B modes, constraining \neff\ and $\Sigma m_\nu$,  investigating dark energy and performing gravity tests with secondary CMB  anisotropy---the CMB polarization anisotropy is much less affected by both foregrounds and atmospheric noise. In fact, it should be possible to measure the primary CMB anisotropy in E-mode polarization to multipoles significantly higher than is possible in TT, thereby extending the lever arm to measure the spectral index and running of the primordial scalar (density) fluctuations. CMB lensing benefits from $\ell_{max}$ of order 5000 and secondary CMB measurements are greatly improved with $\ell_{max}$ of order 10,000 and higher, requiring large-aperture telescopes with diameters of several meters. Owing to the steep scaling of telescope cost with aperture diameter, it is likely not cost-effective to consider separate large aperture telescopes each optimized for a single frequency band. 

CMB-S4 is therefore envisioned to include dedicated large-aperture, wide-field-of-view telescopes equipped with multiple band detector arrays.

\section{A strawman instrument configuration}
\label{sec:strawman}

The rough conceptual design outlined above clearly needs to be refined.  The first priorities are to determine the specific measurements needed to meet the requirements for each of the science goals---the purpose of this Science Book---and then to translate them into instrumentation design specifications. We need to determine:  the required resolution and sensitivity; the number of bands to mitigate foreground contamination, which is likely to be function of angular scale; the required sky coverage; the beam specifications; the scanning strategy and instrument stability; etc. 

Determining these specifications requires simulations, informed by the best available data and phenomenological models.  Only when we have these specifications in hand can we design the instrument and answer such basic questions as the number and sizes of the telescopes.  This will be, of course, an iterative process, involving detailed simulations and cost considerations. At this time the Science Book is a working document with this first edition focused primarily on defining the possible reach in the  key science areas, along with the simulations needed to refine the science case and set the specifications of the needed measurements. This will set the stage for defining the instrument.


On the other hand, we need a jumping-off point for exploring instrument configuration parameter space.  
Simple, back-of-the-envelope calculations make it clear that achieving the science goals outlined above requires a raw sensitivity equivalent to roughly 500,000 detectors operating for four years, though we may find that certain science goals push us to yet greater detector count. This order-of-magnitude level of sensitivity is appropriate for both measuring the tensor-to-scalar ratio $r$ and to the ``non-$r$'' science goals, but the other specifications for the instrument and survey (resolution, sky coverage, band placement) potentially pull in different directions for these two sets of goals. For this reason, we choose as a baseline for parameter forecasts two separate instrument configurations, one which we will optimize for $r$ constraints and one for non-$r$ science, with the detector effort split evenly between the two configurations. If the optimization exercise tells us that the two configurations are similar enough, then the two surveys can be re-merged. 

For the ``$r$'' survey, the strawman configuration consists of an array of small-aperture ($\sim 1$m) telescopes and a separate large-aperture telescope to measure and remove the lensing contamination on the patch of sky targeted by the small-aperture array. The $10^6$ detector years (250,000 detectors operating for four years) is split between the small- and large-aperture efforts in a way that optimizes the combination of noise and lensing residuals. The known foregrounds at 100 to 150~GHz, synchrotron and thermal dust, require the small-aperture effort to be split into at least three bands, but to guard against potential foreground complexity any realistic configuration would have many more. There are four accessible atmospheric windows in the frequency range at which the CMB peaks, centered at roughly 35, 90, 150, and 250~GHz. In the strawman configuration considered here, each of these windows is split into two bands. The total detector effort for the small-aperture telescopes is split between the eight bands to optimize the combination of noise and foreground residuals. The parameter space that can then be explored to discover what is necessary to reach the target sensitivity to $r$ include fraction of sky covered, band placement, and total detector count.

For the ``non-$r$'' survey, the strawman configuration consists of an array of medium-to-large-aperture telescopes,  with the full $10^6$ detector years (250,000 detectors operating for four years) dedicated to a small number of frequency bands near the peak of the CMB. The key instrumental parameters to investigate for the neutrino, light-relic, and dark energy science goals are angular resolution, sky coverage, and total detector effort. 

\section{The Road from Stage 3 to Stage 4}
\label{sec:context}


The Stage-2 and Stage-3 experiments are logical technical and scientific stepping stones to CMB-S4. Ongoing R\&D directed toward achieving the scaling up required to CMB-S4 is being pursued at several universities and national labs. 
Figure~\ref{fig:science_timeline-intro} shows the timeline of the expected increase in sensitivity and the corresponding improvement for a few of the key cosmological parameters for Stage-3, along with the threshold-crossing aspirational goals  targeted for CMB-S4. 


\begin{figure}[t]
\centering \includegraphics[width=0.8\textwidth]{Intro/Fig-FlowChart1_v1.pdf}
\vskip 10pt \caption{Schematic timeline showing the expected increase in sensitivity ($\mu$K$^2$) and the corresponding improvement for a few of the key cosmological parameters for Stage-3, along with the threshold-crossing aspirational goals targeted for CMB-S4.}
\label{fig:science_timeline-intro}
\end{figure}

Finally, in Fig.~\ref{fig:flowchart} we show how the scientific findings (yellow circles), the technical advances (blue circles) and satellite selections (green circles) would affect the science goals, survey strategy and possibly the design of CMB-S4.

\begin{figure}[ht]
\centering \includegraphics[trim=1in 0in 1.2in 0in, clip, width=0.8\textwidth,]{Intro/Fig-FlowChart2_v1.pdf}
\caption{Schematic chart showing how scientific findings (yellow circles), technical advances (blue circles) and satellite decisions by various agencies (green circles) would affect the science goals, the survey strategy and possibly the design of CMB-S4 (green boxes)}
\label{fig:flowchart}
\end{figure}





