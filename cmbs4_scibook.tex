%  This is the driver file for the CMB-S4 Science Book.

%%% it has been adapted from the Snowmass reports from the Cosmic Frontier  %%%%
%%% Working Groups to use bibtex. 19Sep15, T. Crawford (orig. 5/13/13 J. Feng) %%%%

%  D. Hitlin   9/23/03   derived from the BABAR Physics Book format

%  To use LATEX with this format, you must have the follwing files 
%  in the same directory as your text source and figure files
%  tcibook.cls
%  fancyhea.sty
%  work.sty
%  epsfig.sty
%  workshopsym.tex       This file provides macros for many common symbols
%                         Using these macros will provide uniformity of notation
%                         for the basic particle symbols, units, etc.
%
%  These provide the page size, type style, headings, etc.


\documentclass[titlepage]{tcibook}
\usepackage{fancyhea}
\usepackage{work}
\usepackage{bm}       %    enables bold math symbols  e.g.  \bm{\gamma}
\usepackage{graphicx}
\usepackage{hyperref}      % hypertext links %%ARXIV
\usepackage[usenames,dvipsnames]{color} %used for font color
\usepackage{amssymb}
\usepackage{amsmath}
\usepackage{chngcntr}
\usepackage{booktabs}
\usepackage{multirow}
\counterwithout{figure}{chapter}
%\usepackage[explicit]{titlesec}
\usepackage{titlesec}

\input workshopsymbols.tex      %   standard macros for common HEP terms
\newcommand{\cmbexp}{{CMB-S4}}
%    
% Abbreviations for ADS bibtex entries
%
\newcommand\mnras{{\it Mon. Not. Roy. Astron. Soc.}}
\newcommand\apj{{\it Ap.\ J.\ }}
\newcommand\apjl{{\it Ap.\ J.\ Lett.}}
\newcommand\araa{{\it Ann. Rev. Astron. Astroph.}}
\newcommand\physrep{{\it Phys. Rept.}}
\newcommand\apjs{{\it Ap.\ J.\ Suppl.\ }}
\newcommand\jcap{{\it JCAP }}
\newcommand\prd{{\it Phys.\ Rev.\ }{\bf D}\ }
\newcommand\aap{{\it A \& A\ }}

\setlength{\headheight}{14pt}

% subsubsubsections and subsubsections are numbered as well as chapters, sections and subsections.
\setcounter{secnumdepth}{3}

% macros
\newcommand{\neff}{\ensuremath{N_\mathrm{eff}}}
\newcommand{\fsky}{\ensuremath{f_\mathrm{sky}}}
\newcommand{\comment}[1]{{\color{blue}{Comment: #1}}}
\newcommand{\XX}{\mathit{XX}}
\newcommand{\TT}{\mathit{TT}}
\newcommand{\TE}{\mathit{TE}}
\newcommand{\EB}{\mathit{EB}}
\newcommand{\EE}{\mathit{EE}}
\newcommand{\BB}{\mathit{BB}}
\newcommand{\TTilde}[2]{\mathit{\tilde #1\!\tilde #2}}
\newcommand{\ttilde}[2]{\mathit{\tilde #1\tilde #2}}
\newcommand{\tWX}{\TTilde{W}{X}}
\newcommand{\tST}{\TTilde{S}{T}}
\newcommand{\tYZ}{\TTilde{Y}{Z}}
\newcommand{\tWW}{\TTilde{W}{W}}
\newcommand{\tXX}{\TTilde{X}{X}}
\newcommand{\tUV}{\TTilde{U}{V}}
\newcommand{\tEE}{\TTilde{E}{E}}
\newcommand{\tEB}{\TTilde{E}{B}}
\newcommand{\tBB}{\TTilde{B}{B}}
\newcommand{\expt}{\mathrm{expt}}
\newcommand{\eq}[1]{(\ref{eq:#1})} 
\newcommand{\eqq}[1]{Eq.~(\ref{eq:#1})} 
\definecolor{orange}{rgb}{1,0.3,0}
\newcommand\comments[1]{\textcolor{orange}{[#1]} }

\DeclareUrlCommand\email{\urlstyle{rm}}

\begin{document}

%\title{CMB-S4 Science Book \\ Working Draft}
%\author{CMB-S4 Collaboration}
%\maketitle


\def\bibname{References}

\bibliographystyle{utphys}  %%%% MODIFIED FOR CF %%%%
%\bibliographystyle{plain}

\raggedbottom

\pagenumbering{roman}

\parindent=0pt
\parskip=8pt
\setlength{\evensidemargin}{0pt}
\setlength{\oddsidemargin}{0pt}
\setlength{\marginparsep}{0.0in}
\setlength{\marginparwidth}{0.0in}
\marginparpush=0pt

% The content begins here

\renewcommand{\chapname}{chap:intro_}
\renewcommand{\chapterdir}{.}
\renewcommand{\arraystretch}{1.25}
\addtolength{\arraycolsep}{-3pt}

\pagenumbering{roman} 
\chapter*{CMB-S4 Science Book}
\vskip -9.5pt
\hbox to\headwidth{%
       \leaders\hrule height1.5pt\hfil}
\vskip-6.5pt
\hbox to\headwidth{%
       \leaders\hrule height3.5pt\hfil}

{\Large\bf
  \begin{center}
   CMB-S4 Collaboration\\
   Working Draft\\
   \bigskip
   \today
%  March 21, 2016
 \end{center}
}
\eject

\setcounter{page}{1}

\input ExecSum/exec-sum.tex

\begin{center}
  {\Large \bf Acknowledgements}
\end{center}
  
\tableofcontents

%%%% Author comment macros here

\def\as#1{[{\bf AS:} {\it #1}] }


%%%%%%%%%%%%%%%%%%%%%%%%%%%%%%%%%%%%%%%%%%%%%%%%%%%
%%%%%%%%%%%%%%%%%%%%%%%%%%%%%%%%%%%%%%%%%%%%%%%%%%%
%%%     All of your files should be in a subdirectory.  Here the
%%%     subdirectory is called CF0. The title of your
%%%     report should be wgreportCF0.tex in that subdirectory.  Input
%%%     that file here
%%%%%%%%%%%%%%%%%%%%%%%%%%%%%%%%%%%%%%%%%%%%%%%%%%%%
%%%%%%%%%%%%%%%%%%%%%%%%%%%%%%%%%%%%%%%%%%%%%%%%%%%

\eject
\pagenumbering{arabic} 
\setcounter{page}{1}
\input Intro/intro_cmbs4.tex
\input Inflation/inflation_cmbs4.tex 
\input Neutrinos/neutrinos_cmbs4.tex 
\input Neff/neff_cmbs4.tex
\input DarkMatter/darkmatter_cmbs4.tex
\input DarkEnergy/darkenergy_cmbs4.tex 
\input CMBLensing/cmblensing_cmbs4.tex 
\input Analysis/analysis_cmbs4.tex 
\input Instrumentation/instrumentation_cmbs4.tex 

\bibliography{cmbs4}

%\input conclusions_cmbs4.tex

%%%%%%%%%%%%%%%%%%%%%%%%%%%%%%%%%%%%%%%%%%%%%%%%%%
%%%%%%%%%%%%%%%%%%%%%%%%%%%%%%%%%%%%%%%%%%%%%%%%%%
%%%   Your subdirectory (here CF0) should include
%%%    the files:
%%%           wgreportCF0.tex
%%%           authorlistCF0.tex
%%%           bibCF0.bib
%%%         and all needed figures in pdf format
%%%%%%%%%%%%%%%%%%%%%%%%%%%%%%%%%%%%%%%%%%%%%%%%%%%%
%%%%%%%%%%%%%%%%%%%%%%%%%%%%%%%%%%%%%%%%%%%%%%%%%%%%

\end{document}


