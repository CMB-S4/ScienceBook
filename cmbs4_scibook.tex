%  This is the driver file for the CMB-S4 Science Book.

%%% it has been adapted from the Snowmass reports from the Cosmic Frontier  %%%%
%%% Working Groups to use bibtex. 19Sep15, T. Crawford (orig. 5/13/13 J. Feng) %%%%

%  D. Hitlin   9/23/03   derived from the BABAR Physics Book format

%  To use LATEX with this format, you must have the follwing files 
%  in the same directory as your text source and figure files
%  tcibook.cls
%  fancyhea.sty
%  work.sty
%  epsfig.sty
%  workshopsym.tex       This file provides macros for many common symbols
%                         Using these macros will provide uniformity of notation
%                         for the basic particle symbols, units, etc.
%
%  These provide the page size, type style, headings, etc.


\documentclass[titlepage]{tcibook}
\usepackage{fancyhea}
\usepackage{work}
\usepackage{bm}       %    enables bold math symbols  e.g.  \bm{\gamma}
\usepackage{graphicx}
\usepackage{hyperref}      % hypertext links %%ARXIV
\usepackage[usenames,dvipsnames]{color} %used for font color
\usepackage{amssymb}
\usepackage{amsmath}
\usepackage{chngcntr}
\usepackage{booktabs}
\usepackage{multirow}
\counterwithout{figure}{chapter}
%\usepackage[explicit]{titlesec}
\usepackage{titlesec}

\usepackage[utf8]{inputenc}

\input workshopsymbols.tex      %   standard macros for common HEP terms
\newcommand{\cmbexp}{{CMB-S4}}
%    
% Abbreviations for ADS bibtex entries
%
\newcommand\mnras{{\it Mon. Not. Roy. Astron. Soc.}}
\newcommand\apj{{\it Ap.\ J.\ }}
\newcommand\apjl{{\it Ap.\ J.\ Lett.}}
\newcommand\araa{{\it Ann. Rev. Astron. Astroph.}}
\newcommand\physrep{{\it Phys. Rept.}}
\newcommand\apjs{{\it Ap.\ J.\ Suppl.\ }}
\newcommand\jcap{{\it JCAP }}
\newcommand\prd{{\it Phys.\ Rev.\ }{\bf D}\ }
\newcommand\aap{{\it A \& A\ }}

\newcommand{\cobe}{{\sl COBE}}
\newcommand{\wmap}{{\sl WMAP}}
\newcommand{\planck}{{\sl Planck}}

\setlength{\headheight}{14pt}

% subsubsubsections and subsubsections are numbered as well as chapters, sections and subsections.
\setcounter{secnumdepth}{3}

% macros
\newcommand{\neff}{\ensuremath{N_\mathrm{eff}}}
\newcommand{\fsky}{\ensuremath{f_\mathrm{sky}}}
\newcommand{\comment}[1]{{\color{blue}{Comment: #1}}}
\newcommand{\XX}{\mathit{XX}}
\newcommand{\TT}{\mathit{TT}}
\newcommand{\TE}{\mathit{TE}}
\newcommand{\EB}{\mathit{EB}}
\newcommand{\EE}{\mathit{EE}}
\newcommand{\BB}{\mathit{BB}}
\newcommand{\TTilde}[2]{\mathit{\tilde #1\!\tilde #2}}
\newcommand{\ttilde}[2]{\mathit{\tilde #1\tilde #2}}
\newcommand{\tWX}{\TTilde{W}{X}}
\newcommand{\tST}{\TTilde{S}{T}}
\newcommand{\tYZ}{\TTilde{Y}{Z}}
\newcommand{\tWW}{\TTilde{W}{W}}
\newcommand{\tXX}{\TTilde{X}{X}}
\newcommand{\tUV}{\TTilde{U}{V}}
\newcommand{\tEE}{\TTilde{E}{E}}
\newcommand{\tEB}{\TTilde{E}{B}}
\newcommand{\tBB}{\TTilde{B}{B}}
\newcommand{\expt}{\mathrm{expt}}
\newcommand{\eq}[1]{(\ref{eq:#1})} 
\newcommand{\eqq}[1]{Eq.~(\ref{eq:#1})} 
\definecolor{orange}{rgb}{1,0.3,0}
\newcommand\comments[1]{\textcolor{orange}{[#1]} }


% cut and pasted from Neff chapter
\def\Neff{N_{\rm eff}}
\def\Nf{N_{\rm eff}}
\def\gs{g_{\star}}
\def\Mpl{M_{\rm P}}
%\def\gtrsim{\raise-.75ex\hbox{$\buildrel>\over\sim$}}
\def\lsim{\raise-.75ex\hbox{$\buildrel<\over\sim$}}


\DeclareUrlCommand\email{\urlstyle{rm}}

\begin{document}

%\title{CMB-S4 Science Book \\ Working Draft}
%\author{CMB-S4 Collaboration}
%\maketitle


\def\bibname{References}

\bibliographystyle{utphys}  %%%% MODIFIED FOR CF %%%%
%\bibliographystyle{plain}

\raggedbottom

\pagenumbering{roman}

\parindent=0pt
\parskip=8pt
\setlength{\evensidemargin}{0pt}
\setlength{\oddsidemargin}{0pt}
\setlength{\marginparsep}{0.0in}
\setlength{\marginparwidth}{0.0in}
\marginparpush=0pt

% The content begins here

\renewcommand{\chapname}{chap:intro_}
\renewcommand{\chapterdir}{.}
\renewcommand{\arraystretch}{1.25}
\addtolength{\arraycolsep}{-3pt}

\pagenumbering{roman} 
\chapter*{CMB-S4 Science Book\\ First Edition}
\vskip -9.5pt
\hbox to\headwidth{%
       \leaders\hrule height1.5pt\hfil}
\vskip-6.5pt
\hbox to\headwidth{%
       \leaders\hrule height3.5pt\hfil}


  \begin{center}
   {\Large\bf
      CMB-S4 Collaboration\\
      \bigskip
      August 1, 2016
   }
   \vskip 1 in
   {\Large\bf
   This advanced copy is being provided prior to posting with the author list on the public archive
% Draft date: \today
%  March 21, 2016
   }
 \end{center}
\eject

\setcounter{page}{1}

\input ExecSum/exec-sum.tex

\begin{center}
  {\Large \bf Acknowledgements}
\end{center}
\bigskip

This Science Book is the product of a large community of scientists spanning the globe that are united in their enthusiasm for CMB research and their understanding of the critical need of proceeding with the next generation ground-based CMB program, CMB-S4, to realize the full potential of CMB studies to expand our understanding of the fundamental nature of space and time and the evolution of the Universe.

The process to define the next generation ground-based CMB program and the understanding that the scale would necessitate a phase change in the way ground-based program has proceeded, started with the Snowmass planning exercise in 2013.  

Through several workshops and numerous telecons, the CMB experimental groups, the broader cosmology community and the DOE HEP cosmic frontier community came together to produce two influential planning papers, title, title, which included the outline of the CMB-S4 target. 


These papers along with input from the experimental groups, to P5 and led to their including CMB-S4 in their recommended program. 

After the P5 report was released, the CMB community began a series of biannual workshops to advance CMB-S4. The first of these was a dedicated session at the  {\it Cosmology with the CMB 
and its Polarization} workshop at the University of Minnesota January 14-16, 2015 attended by over 90 scientists.
% 94 registrants
Discussions focused on the unique and vital role of the future ground-based CMB program and it was decided that the community needed to draft the CMB-S4 science book.  The second and third workshops {\it Cosmology with CMB-S4}, held at the University of Michigan September 21-22, 2015 with over 100 participants
%106 registrants
, and at the Lawrence Berkeley National Laboratory March 7-9, 2016 with over 160 participants
% 1xx registrants
, respectively, were dedicate to developing the Science Book.  Working groups for each CMB-S4 science thrust were responsible for preparing and leading dedicated sessions at the workshop and for drafting the corresponding chapters in this book. In addition a small writing group was responsible for integrating the science book. Through the workshops, numerous teleconferences, postings on the CMB-S4 wiki,  contributions to the github science book repository, and feedback on drafts,  over 200 scientists have contributed to this first edition of the CMB-S4 science book.





  
\tableofcontents

%%%% Author comment macros here

\def\as#1{[{\bf AS:} {\it #1}] }


%%%%%%%%%%%%%%%%%%%%%%%%%%%%%%%%%%%%%%%%%%%%%%%%%%%
%%%%%%%%%%%%%%%%%%%%%%%%%%%%%%%%%%%%%%%%%%%%%%%%%%%
%%%     All of your files should be in a subdirectory.  Here the
%%%     subdirectory is called CF0. The title of your
%%%     report should be wgreportCF0.tex in that subdirectory.  Input
%%%     that file here
%%%%%%%%%%%%%%%%%%%%%%%%%%%%%%%%%%%%%%%%%%%%%%%%%%%%
%%%%%%%%%%%%%%%%%%%%%%%%%%%%%%%%%%%%%%%%%%%%%%%%%%%

\eject
\pagenumbering{arabic} 
\setcounter{page}{1}
\input Intro/intro_cmbs4.tex
\input Inflation/inflation_cmbs4.tex 
\input Neutrinos/neutrinos_cmbs4.tex 
\input Neff/neff_cmbs4.tex
\input DarkMatter/darkmatter_cmbs4.tex
\input DarkEnergy/darkenergy_cmbs4.tex 
\input CMBLensing/cmblensing_cmbs4.tex 
\input Analysis/analysis_cmbs4.tex 
%\input Instrumentation/instrumentation_cmbs4.tex 

\bibliography{cmbs4}

%\input conclusions_cmbs4.tex

%%%%%%%%%%%%%%%%%%%%%%%%%%%%%%%%%%%%%%%%%%%%%%%%%%
%%%%%%%%%%%%%%%%%%%%%%%%%%%%%%%%%%%%%%%%%%%%%%%%%%
%%%   Your subdirectory (here CF0) should include
%%%    the files:
%%%           wgreportCF0.tex
%%%           authorlistCF0.tex
%%%           bibCF0.bib
%%%         and all needed figures in pdf format
%%%%%%%%%%%%%%%%%%%%%%%%%%%%%%%%%%%%%%%%%%%%%%%%%%%%
%%%%%%%%%%%%%%%%%%%%%%%%%%%%%%%%%%%%%%%%%%%%%%%%%%%%

\end{document}


