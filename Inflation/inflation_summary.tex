Inflation, a period of accelerated expansion of the early Universe, is the leading paradigm for explaining the origin of the primordial density perturbations that grew into the CMB anisotropies and eventually into the stars and galaxies we see around us. In addition to primordial density perturbations, the rapid expansion creates primordial gravitational waves that imprint a characteristic polarization pattern onto the CMB. If our Universe is described by a typical model of inflation that naturally explains the statistical properties of the density perturbations, CMB-S4 will detect this signature of inflation. A detection of this particular polarization pattern would open a completely new window onto the physics of the early Universe and provide us with an additional relic left over from the hot big bang. This relic would constitute our most direct probe of the very early Universe and transform our understanding of several aspects of fundamental physics. Because the polarization pattern is due to quantum fluctuations in the gravitational field during inflation, it would provide insights into the quantum nature of gravity. The strength of the signal, encoded in the tensor-to-scalar ratio $r$, would provide a direct measurement of the expansion rate of the Universe during inflation. A detection with CMB-S4 would point to inflationary physics near the energy scale associated with grand unified theories and would provide additional evidence in favor of the idea of the unification of forces. Knowledge of the scale of inflation would also have broad implications for many other aspects of fundamental physics, including ubiquitous ingredients of string theory like axions and moduli. 
 
Even an upper limit of $r<0.002$ at $95\%$ CL achievable by CMB-S4, over an order of magnitude stronger than current limits, would significantly advance our understanding of inflation. It would rule out the most popular and most widely studied classes of models and dramatically impact how we think about the theory. To some, the remaining class of models would be contrived enough to give up on inflation altogether. Furthermore, CMB-S4 is in a unique position to probe the statistical properties of primordial density perturbations through measurements of primary anisotropies in the temperature and polarization of the CMB with unprecedented precision, providing us with invaluable information about the early Universe. 
