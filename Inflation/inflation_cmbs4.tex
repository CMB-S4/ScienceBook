%%%%%% CMB-S4 Inflation Chapter  %%%%%%%%%%%%%%%%
 
\chapter{Inflation Physics from the Cosmic Microwave Background}
\renewcommand*\thesection{\arabic{section}}

\def\gtrsim{\raise-.75ex\hbox{$\buildrel>\over\sim$}}
%%%%%%%%%%%%%%%%%%%%%%%%%%%%%%%%%%%%%%%%%%%%%%%%%%%%%%%%%%%
%%%%%%%%%%%%%%%%%%%%%%%%%%%%%%%%%%%%%%%%%%%%%%%%%%%%%%%%%%%
%%%%%%%%%%%%%%%%%%%%%%%%%%%%%%%%%%%%%%%%%%%%%%%%%%%%%%%%%%%
%%%%%%%%%%%%%%%%%%%%%%%%%%%%%%%%%%%%%%%%%%%%%%%%%%%%%%%%%%%

\section{Introduction}
%The study of the polarization of the cosmic microwave background will bring additional information about both the gravitational and matter sectors of the primordial universe. However, for primordial gravitational waves there are clear theoretical thresholds that can be reached with this next generation instrument. We will use the tensor-to-scalar ratio as the primary inflationary science driver for the design.

The small anisotropies in the cosmic microwave background radiation contain invaluable information about the primordial universe. On the one hand, they encode the properties of the primordial scalar (or density) perturbations and hence the matter sector of the primordial universe. 
%Models of the primordial universe make quantitative predictions for the statistical properties of primordial scalar perturbations, for example for their scale dependence. 
CMB-S4 will significantly improve current constraints on primordial observables in the scalar sector, typically by a factor of at least five. On the other hand, on degree scales a polarization pattern known as B-mode polarization would reveal the existence of primordial tensor modes or gravitational waves. In the tensor sector, CMB-S4 will improve current constraints by almost two orders of magnitude. This is especially interesting because it allows this next generation instrument to reach theoretically well-motivated thresholds for the tensor-to-scalar ratio (the ratio of power in tensor modes to power in scalar modes), which serves as the primary inflationary science driver for the design. A detection of primordial gravitational waves in the range accessible to this instrument would:
\begin{itemize}
 \item Reveal a new scale of particle physics far above those accessible with terrestrial particle colliders.  
\item  Rule out the only currently developed competitor to inflation (ekpyrotic scenario)
\end{itemize}
If, in addition, we can conclusively determine that any detected signal is dominated by vacuum fluctuations then a detection would also
\begin{itemize}
 \item Identify the energy scale of inflation. 
 \item Provide strong evidence that the complete theory of quantum gravity must accommodate a Planckian field range for the inflaton.
 \item Provide strong evidence that gravity is quantized, at least at the linear level.
\item  Constrain the mass of the graviton
\end{itemize}
In the absence of a detection, each factor of ten improvement beyond the current constraint will reach new and significant thresholds, ruling out major classes of inflation models. 

In Section \ref{sec:detection} we review in detail what a detection of primordial gravitational waves would mean and what follow-up measurements should or could be done to further characterize any signal. Section \ref{sec:upperLimits} explains why, in the absence of a detection, a robust upper limit of $r<0.001$ is the interesting number to push for. Section \ref{sec:needs} lays out what is required to achieve that goal. The final two sections describe the significant gains CMB S4 will allow in constraining other aspects of the primordial universe, both standard and more speculative. These include characterizing the scalar power spectrum, constraining curvature, non-Gaussianity, and isocurvature modes, further probes of CMB `anomalies' and test/constraints of cosmic strings, the mass of the graviton.....
 
Recent reviews \cite{Kamionkowski:2015yta}.

\section{Implications of a detection of primordial gravitational waves with CMB-S4}
\label{sec:detection}
The overall evolution of the universe is well modeled by a Friedmann-Robertson-LeMaitre-Walker line element
\begin{equation}
ds^2=-dt^2+a^2(t)\left[\frac{dr^2}{1-kr^2}+r^2d\Omega^2\right]\,,
\end{equation}
where $k=\pm1$ allows for spatial curvature and the time evolution is specified by the scale factor, $a(t)$. The Hubble parameter, $H=\dot{a}/a$, gives the rate of expansion of the universe. 
%Current data is consistent with a spatially flat universe, and we will assume spatial flatness ($k=0$) for most of the discussion. However, we will return to constraints on the curvature in Section xxx.  

The existence of primordial Helium and the cosmic microwave background radiation provide strong evidence for a hot big bang, a period during which the universe was dominated by radiation before it became dominated by matter and eventually dark energy. In the context of general relativity, observations of the cosmic microwave background furthermore provide strong evidence for a period preceding the hot big bang during which the co-moving Hubble radius, $(a|H|)^{-1}$, was decreasing with time: the measured average CMB temperature and the statistics of the measured anisotropies are the same over regions that otherwise share no causal history. 

%Arguably the most significant discrepancy between the predictions of a generic hot big bang model and our observed sky is the horizon problem: the measured average CMB temperature and the statistics of the measured anisotropies are the same over regions that share no causal history. Models for the primordial universe attempt to give a causal (and ideally also a `natural') explanation for the observed homogeneity on scales greater than a few degrees by postulating an early era where the co-moving Hubble radius, $(a|H|)^{-1}$ is decreasing with time. In an expanding universe this requires an era of accelerated expansion, $\ddot{a}>0$. The matter field that sources this expansion should have equation of state $w\approx -1$, which is minimally provided by a single scalar field whose energy density is predominately determined by its potential.

In an expanding universe a decreasing co-moving Hubble radius requires an era of accelerated expansion, $\ddot{a}>0$, cosmic inflation. Such a period will drive the spatial curvature close to zero, in good agreement with current observations. Thus, we will assume spatial flatness and set $k=0$ for most of the discussion, but will return to constraints on the curvature in Section xxx. Since the period of cosmic inflation must end, there must exist a clock, or scalar degree of freedom. According to the uncertainty principle this clock must fluctuate, generating density perturbations that are adiabatic. In the most economic scenarios, these density perturbations are the seeds that grow into the anisotropies observed in the cosmic microwave background radiation and the stars and galaxies around us. Other degrees of freedom could, of course, also be present during this phase and might even be responsible for the generation of density perturbations we observe. {\bf (Mention isocurvature modes here?)}

Alternatively, the phase of decreasing co-moving Hubble radius could have occurred during a period of decelerating contraction which must then be followed by a bounce as in the ekpyrotic or matter bounce scenarios. 
% The matter field that sources this expansion should have an equation of state $w\approx -1$, which is minimally provided by a single scalar field whose energy density is predominately determined by its potential.

%The remarkable feature of inflation is that once an evolving scalar field is invoked to source the background accelerated expansion, quantum fluctuations during inflation inevitably generate post-inflationary metric perturbations. 
For these early times, the ADM formalism provides a convenient parametrization of the line element
\begin{eqnarray}
\label{eq:metric}
ds^2&=&-N^2dt^2 +h_{ij}(dx^i+N^idt)(dx^j+N^jdt)\,\nonumber\\
h_{ij}&=&a^2(t)[e^{2\zeta}\delta_{ij}+\gamma_{ij}]\,.
\end{eqnarray}

The equations of motion for $N$ (the lapse) and $N^i$ (the shift) are the Hamiltonian and momentum constraints, while $\zeta$ ($=-\mathcal{R}$ of {\it Planck}) and $\gamma_{ij}$ contain the dynamical scalar and tensor degrees of freedom. In scenarios with matter sources other than a scalar field there may also be vector perturbations. These rapidly decay and can be neglected unless they are actively sourced in the post-inflationary universe, e.g. by cosmic strings.

The statistical properties of the scalar and tensor fluctuations, $\zeta$ and $\gamma_{ij}$, at times sufficiently late so that they have frozen out provide the link between the primordial era and the CMB as well as other probes of the structure of the late universe. For a universe that is statistically homogeneous and isotropic and in which the primordial fluctuations are Gaussian, the information about the statistical properties is contained in the two-point correlation functions

%The link between the primordial era and the CMB and other probes of the structure of the late universe are the statistical properties of the scalar and tensor fluctuations, $\zeta$ and $\gamma_{ij}$, at times sufficiently late so that they have frozen out. For a universe that is statistically homogeneous and isotropic and in which the primordial fluctuations are Gaussian, the information about the statistical properties is contained in the two-point correlation functions

%Observations today constrain these parameters at early times, which we can compare to calculations from any theory for the primordial era (and reheating). To an excellent approximation, the observed scalar fluctuations are statistically homogeneous, isotropic, and Gaussian so they are well characterized by a power spectrum that is only a function of wavenumber $|\vec{k}|$. Assuming that the same is true for any tensor fluctuations (as is predicted by inflation), we characterize the amplitude and scale dependence of fluctuations by


\begin{eqnarray}
\langle\zeta(\vec{k})\zeta(\vec{k}^{\prime})\rangle&=&(2\pi)^3\delta^3(\vec{k}+\vec{k}^{\prime})\frac{2\pi^2}{k^3}\mathcal{P}_{\zeta}(k)\nonumber\\
\langle\gamma_s(\vec{k})\gamma_{s^{\prime}}(\vec{k}^{\prime})\rangle&=&(2\pi)^3\delta_{ss^{\prime}}\delta^3(\vec{k}+\vec{k}^{\prime})\frac{2\pi^2}{k^3}\frac{1}{2}\mathcal{P}_{t}(k)\nonumber\\
\end{eqnarray}
where the factor of $1/2$ in the second to last line accounts for the fact that the measured power includes contributions from each of the two graviton polarizations. In single field slow-roll inflation, the gauge invariant combination of metric and scalar field fluctuations that is conserved outside the horizon has the power spectrum
\begin{equation}
\label{eq:inf_Pzeta}
\mathcal{P}_{\zeta}(k)=\frac{1}{2\epsilon M_p^2}\left.\left(\frac{H}{2\pi}\right)^2\right|_{k=aH}
\end{equation}
where $\epsilon=-\dot{H}/H^2$ is the first slow-roll parameter, and $M_p=1/\sqrt{8\pi G}$ is the reduced Planck mass. As indicated, the Hubble parameter and $\epsilon$ are to be evaluated at the time the wavenumber $k$ is equal to the inverse comoving Hubble radius. In the absence of additional sources, the tensor power spectrum generated by inflation is
\begin{equation}
\label{eq:inf_Pt}
\mathcal{P}_{t}(k)=\frac{8}{M_p^2}\left.\left(\frac{H}{2\pi}\right)^2\right|_{k=aH}
\end{equation}

The power spectra are conventionally parameterized as
%\begin{eqnarray}
%\mathcal{P}_{\zeta}(k)&\equiv& A_s\left(\frac{k}{k_\star}\right)^{n_s-1+\frac{1}{2}(dn_s/d\,\ln k)\ln(k/k_\star)+\dots}\,,\nonumber\\
%\mathcal{P}_{t}(k)&\equiv& A_t \left(\frac{k}{k_\star}\right)^{n_t+\frac{1}{2}(dn_t/d\,\ln k)\ln(k/k_\star)+\dots}\,.
%\end{eqnarray}
\begin{equation}\label{eq:specind}
\mathcal{P}_{\zeta}(k)\equiv A_s\left(\frac{k}{k_\star}\right)^{n_s(k)-1}\qquad{\rm and}\qquad\mathcal{P}_{t}(k)\equiv A_t \left(\frac{k}{k_\star}\right)^{n_t(k)}\,,
\end{equation}
where $A_s$, $A_t$ are the scalar and tensor amplitudes, and the logarithmic derivatives of the power spectra, conventionally denoted by $n_s(k)-1$ and $n_t(k)$, are in turn parameterized in terms of the scalar and tensor spectral indices and their logarithmic derivatives
\begin{equation}
n_s(k)-1=n_s-1+\left.\frac{dn_s(k)}{d\ln k}\right|_{k_\star}\ln(k/k_\star)+\dots\qquad{\rm and}\qquad n_t(k)=n_t+\left.\frac{dn_t(k)}{d\ln k}\right|_{k_\star}\ln(k/k_\star)+\dots \,.
\end{equation}
The tensor-to-scalar ratio, $r$ is the relative power in the two types of fluctuations at a chosen pivot scale $k_\star$ accessible by the CMB:
\begin{equation}
r=\frac{A_t}{A_s}
\end{equation}


As the co-moving horizon has decreased sufficiently so that $k/aH\ll 1$ the perturbations freeze out. The power spectra of $\zeta$ and $\gamma_s$ become time-independent, and only begin to evolve once the modes of interest re-enter the horizon at late times. In particular, they set the initial conditions for the system of equations governing the time evolution of the universe from around $10^9$ K when electrons and positrons have annihilated to the present. To exhibit the link between the primordial perturbations and late time observables explicitly, note that in a spatially flat universe, the contributions of primordial scalar perturbations to the angular power spectra of temperature or E-mode anisotropies are given by
\begin{equation}
C^{(S)}_{XX,\ell}=\int \frac{dk}{k}\mathcal{P}_\zeta(k)\left|\int\limits_0^{\tau_0} d\tau S_X^{(S)}(k,\tau)j_\ell(k(\tau_0-\tau))\right|^2\,,
\end{equation}
where $S_X^{(S)}(k,\tau)$ with $X=T,E$ are source functions that encode the physics of recombination and $j_\ell$ is a spherical Bessel function that encodes the (spatially flat) geometry of the universe.
% and $\mathcal{P}_\zeta(k)$ is the power spectrum of initial conditions of primordial scalar perturbations as a function of the (comoving) momentum of the modes. 
At linear order, scalar perturbations only contribute to angular power spectra of temperature and E-mode polarization and the cross-spectrum of temperature and E-mode polarization, while the tensor perturbations in addition generate B-mode polarization. The primordial contribution of the tensor perturbations to the angular power spectrum of B-modes is 
\begin{equation}
C_{BB,\ell}=\int \frac{dk}{k}\mathcal{P}_t(k)\left|\int\limits_0^{\tau_0} d\tau S_B^{(T)}(k,\tau)j_\ell(k(\tau_0-\tau))\right|^2\,.
\end{equation}
where $S_B^{(T)}(k,\tau)$ is the appropriate source function. 

At present, bounds on the tensor contribution to the temperature and E-mode anisotropies are comparable to constraints on the tensor-to-scalar ratio from B-mode observations. The former constraints are now cosmic variance limited. There is no limit on the latter from cosmic variance and improvements and a potential detection with CMB-S4 will rely on measurements of B-mode polarization on degree scales.

Constraints on the amplitude of primordial tensor modes already strongly disfavor once popular inflationary models like minimally coupled chaotic inflation with a quadratic potential.  We will discuss in detail what a detection of primordial gravitational waves would imply for inflation in the next several sections, but it is also important to note that a detection would rule out contracting universe scenarios. A contracting universe can also put large scales in causal contact if the scale factor $a$ is nearly constant while the magnitude of the Hubble parameter increases. This means the spectrum of gravitational wave fluctuations will be very blue \cite{Khoury:2001wf}. In addition the Hubble parameter at the end of the contracting phase can be approximately bounded (minimally, $H<M_p$, or $H\sim T_{\rm reheat}$) and so the value of $H$ that sets the amplitude of tensor fluctuations on scales accessible through the CMB must be exponentially smaller. The vacuum fluctuations in a contracting universe are then far too small to be detected \cite{Boyle:2003km}.
%During inflation, all fields that are light compared to the Hubble scale pick up quantum fluctuations with rms amplitude proportional to $H/2\pi$. In particular, in the absence of any additional sources, the tensor power spectrum generated by inflation is 
%\begin{equation}
%\label{eq:inf_Pt}
%\mathcal{P}_{t}(k)=\frac{8}{M_p^2}\left(\frac{H}{2\pi}\right)^2\big|_{k=aH}
%\end{equation}
%where $M_p=1/\sqrt{8\pi G}$ is the reduced Planck mass and the Hubble parameter is evaluated at the time the mode $k$ is equal to the inverse comoving Hubble scale. In the scalar sector, the combination of metric and scalar field fluctuations that is conserved outside the horizon (in the gauge used in Eq.(\ref{eq:metric}) has power spectrum
%\begin{equation}
%\label{eq:inf_Pzeta}
%\mathcal{P}_{\zeta}(k)=\frac{1}{2\epsilon M_p^2}\left(\frac{H}{2\pi}\right)^2\big|_{k=aH}
%\end{equation}
%where $\epsilon=-\dot{H}/H^2$ is the first slow-roll parameter. 

%Current constraints on the amplitude of primordial tensor modes already rule out some inflationary models.  We will discuss in detail what a detection of primordial gravitational waves would imply for inflation in the next several sections.
%
%It is important to note that a detection of primordial gravitational waves would rule out contracting universe scenarios. A contracting universe can also put large scales in causal contact if the scale factor $a$ is nearly constant while the Hubble parameter increases. This means the spectrum of gravitational wave fluctuations will be very blue \cite{Khoury:2001wf}. In addition the Hubble parameter at the end of the contracting phase can be approximately bounded (minimally, $H<M_p$, or $H\sim T_{\rm reheat}$) and so the value of $H$ that sets the amplitude of tensor fluctuations on scales accessible through the CMB must be exponentially smaller. The vacuum fluctuations in a contracting universe are then far too small to be detected \cite{Boyle:2003km}.


\subsection{A new scale of particle physics}
\label{sec:scale-of-inflation}
According to the inflationary prediction for the amplitude of primordial gravitational waves, Eq.~(\ref{eq:inf_Pt}), a detection implies a direct measurement of the Hubble scale during inflation. In single field slow-roll models, Eq.~(\ref{eq:inf_Pzeta}), the amplitude of the power spectrum of scalar fluctuations as measured by the {\it Planck} satellite together with the Friedmann equation $3H^2\approx V$ determine energy scale of inflation in terms of $r$ (all at the pivot scale $k=0.05$ Mpc$^{-1}$)
\begin{equation}\label{eq:Vofr}
V^{1/4}=1.04\times 10^{16}{\rm GeV}\left(\frac{r_*}{0.01}\right)^{1/4}\,,
\end{equation}
so that a detection of primordial gravitational waves determines the energy scale of inflation to within a few per cent. 

In more general models of inflation the relation can be modified by changing the power in the scalar sector and then a detection only determines the order of magnitude of the new energy scale. Familiar examples are models in which the speed of sound of the inflaton quanta differs from unity or multi-field models. The extent to which the relation can be modified is bounded by constraints on non-Gaussianity. {\bf be more specific}
%How model-independent is the relationship (equation $r\leftrightarrow V$)? To answer this question, a significant number of papers has dealt with scenarios of low-scale inflation where $r$ is nevertheless large and observable. These scenarios provide an opportunity for escaping the Lyth bound \cite{Lyth:1996im}. 
In other examples string or particle production events source gravitational waves~\cite{Cook:2011hg,Senatore:2011sp}. To maximize the production of gravitational waves, the field $\chi$ creating and annihilating the quanta should be massless, since massive particles contribute a smaller quadrupole moment and are a weaker source of gravitational waves~\cite{Barnaby:2012xt}. Furthermore, $\chi$ should be a stronger source of tensor than of scalar modes~\cite{Barnaby:2012xt,Ferreira:2014zia,Mirbabayi:2014jqa,Ozsoy:2014sba}, since the properties of the latter are tightly constrained by current the bounds on non-Gaussianities~\cite{Barnaby:2010vf}, and in some models by constraints on the running of $n_s$~\cite{Meerburg:2012id}. 

The current bounds on an equilateral type bispectrum constrain scenarios where the inflaton is directly to the additional field sourcing gravitational waves to be at most of the same order as the vacuum signal \cite{Ozsoy:2014sba, others?}. In this case, a detection is still an excellent indicator of the scale of inflation. Improved constraints on non-Gaussianity (see Section \cite{sec:scalar}) will further restrict these scenarios. A bound $f_{\rm NL}^{\rm equil}<?$ achievable with CMB S4 would constrain these scenarios to contribute at most xx\% of any gravitational wave signal.

To completely evade any restrictions from scalar non-Gaussinaity, the dynamics generating $\chi$ should be decoupled from the inflaton sector~\cite{Barnaby:2012xt} (that is, only gravitationally coupled). Together with the considerations above, this suggests a scenario in which $\chi$ is a gauge field whose quanta are created by a parity violating interaction with a spectator field~\cite{Cook:2011hg,Barnaby:2012xt}, so that only modes with a definite handedness are produced \cite{Anber:2006xt}. The gauge fields in turn source gravity waves and scalar perturbations. Helicity conservation implies that gravitons of that same handedness are produced in much larger amount than gravitons of the opposite handedness \cite{Sorbo:2011rz}, and than scalar modes \cite{Barnaby:2012xt}. 
%A scenario where the above limitations are naturally evaded is the one developed in \cite{Cook:2011hg,Barnaby:2012xt}, where $\chi$ is a gauge field -- that is naturally massless -- generated by a parity violating mechanism, so that only modes of $\chi$ with a definite handedness are produced \cite{Anber:2006xt}. The gauge fields in turn source gravity waves and scalar perturbations. Helicity conservation implies that gravitons of that same handedness are produced in much larger amount than gravitons of the opposite handedness \cite{Sorbo:2011rz}, and than scalar modes \cite{Barnaby:2012xt}. 
In such a scenario current constraints on non-Gaussian correlations in the temperature anisotropies can be evaded if the production of $\chi$ quanta occurs only around the time the modes contributing to the multipoles relevant for the B-mode search leave the horizon \cite{Namba:2015gja} because constraints on non-Gaussianities are dominated by smaller scales. Reference~\cite{Namba:2015gja} provides a model in which gravitational waves from gauge field production could be measured at a level of $r=10^{-1}$ with a vacuum contribution of only $r=10^{-4}$. While in that case the determination of the scale of inflation is affected by less than one order of magnitude, adjusting the parameters of the scenario in may allow for more dramatic modifications of Eq.~(\ref{eq:Vofr}). However, in a regime in which the relation is strongly modified the B-mode anisotropies would be highly non-Gaussian. So in the case of a detection even the B-mode bispectrum would become observable with CMB-S4. In addition, since the signal would be parity violating, the angular bispectrum of B-modes would be dominated by $\ell_1+\ell_2+\ell_3=$even, which would vanish in any theory that respects parity. So, such a signal would not be confused with the vacuum fluctuations of the spacetime metric arising in single field slow-roll inflation. We return to this point in more detail in Section \ref{sec:beyond_r} below.

Finally, note that these secondary production mechanisms cannot be used to obtain a signal from a contracting primordial era... {\bf check this with Marco/Lorenzo}


{\bf Should we address these papers at all: 1410.8845, rebutted in 1508.01527, re-rebutted in 1510.06759, (different author) 1510.07956?}


\subsection{Large $r$, inflaton field range, symmetries}
The spectrum of tensor fluctuations depends only on the Hubble ($H$) parameter during inflation, while the scalar power depends on both $H$ and the evolution of the homogeneous field sourcing inflation. This means that the tensor-to-scalar ratio, $r$, can be related to the inflaton field range in Planck units \cite{Lyth:1996im}
\begin{equation}
\label{eq:Lyth}
\frac{\Delta\Phi}{M_p}=\int_0^{\mathcal{N}_{\rm end}}d\mathcal{N}\,\left(\frac{r}{8}\right)^{1/2}
\end{equation}
where $\mathcal{N}$ is the total number of e-folds needed to put the largest observed scales in causal contact during inflation.

Remarkably, the distance the inflaton moves in field space may be used to infer some properties of quantum gravity from the amplitude of primordial gravitational waves. As bottom-up effective theories, Lagrangians describing inflation need only be valid up to some scale $\Lambda_{\rm UV}$ where $H<\Lambda_{\rm UV}\lesssim M_P$. Scenarios with $\Delta\Phi\sim\mathcal{O}(\Lambda_{\rm UV})$ and $\Lambda_{\rm UV}\approx M_p$ (``large field inflation") can be constructed if the inflaton interactions are invariant under a shift symmetry, $\Phi\rightarrow\Phi+c$. The symmetry must be weakly broken to allow for slow-roll inflation but is nonetheless sufficient to ensure that quantum corrections from the inflaton and graviton will not introduce large corrections to the inflationary Lagrangian \cite{Linde:2005ht, Kaloper:2011jz, Csaki:2014bua,Kaplan:2015fuy,Choi:2015fiu}. However, it is non-trivial to arrive at large-field inflation from the top-down, beginning with a theory valid up to energies well above those needed for inflation. Generically, fields that are heavy enough to be integrated out in constructing the effective theory of inflation ( $m>\Lambda_{\rm UV}$) will introduce corrections to the potential that will limit the field range to $\Delta\Phi\ll\Lambda_{\rm UV}$ (``small-field inflation"). Beyond extrapolating our understanding of effective field theory from energies that are tested experimentally this difficulty is also well studied in string theory, which provides a scale $\Lambda_{\rm UV}=\Lambda_{\rm string}\lesssim M_p$. As the best developed candidate for the theory of quantum gravity, string theory is a useful framework for exploring mechanisms that may allow large-field inflation to be realized even in the presence of heavy degrees of freedom. Several promising constructions have been proposed \cite{Silverstein:2008sg, McAllister:2008hb, Kaloper:2008fb, Berg:2009tg, Palti:2014kza,McAllister:2014mpa, Marchesano:2014mla, Blumenhagen:2015xpa}. However, there is also interesting current discussion of whether at least some apparent means to achieve large field inflation (via multiple axions) may be incompatible with basic principles of quantum gravity \cite{Kim:2004rp, delaFuente:2014aca,Bachlechner:2015qja,Heidenreich:2015wga,Kooner:2015rza}. 

Equation (\ref{eq:Lyth}) can be used to determine a useful theoretical threshold for $r$ that would allow the most robust conclusion about the field range of the inflaton. Give the possibility that $\Lambda_{\rm UV}$ may be slightly below $M_p$, and the desire for true parametric control of corrections rather than accidental cancellations, we will define small-field inflation as requiring $\frac{\Delta\Phi}{M_p}\ll1$ (in contrast to perhaps half the literature where the line is drawn at $\frac{\Delta\Phi}{M_p}=1$). The total number of e-folds needed to put the largest scales in causal contact depends on the reheating temperature, but since the integrand in Eq.(\ref{eq:Lyth}) is always positive, we can use constraints on the $\approx 7$ e-folds constrained by current CMB observations to provide a conservative bound. The evolution of $r$ during inflation is model dependent, but as a simple starting point consider canonical single-field slow-roll inflation where the consistency relation between $r$ and $n_T$ to write $d\ln r/d\mathcal{N}=-(n_s-1)-\frac{r}{8}$. The variation of this quantity is second-order in slow-roll parameters, so as a first pass we may take it to be constant and use the {\it Planck} satellite constraints assuming constant $n_s$ (and allowing a $3\sigma$ interval) to find
\begin{equation}
\frac{\Delta\Phi}{M_p}\ll1\Rightarrow r\lesssim0.002.
\end{equation}
Allowing the spectral index to run, we may use the second-order consistency relation $d\ln r/d\mathcal{N}=-\frac{r}{8}(n_s-1+\frac{r}{8})$ and {\it Planck} constraints to find 
\begin{equation}
\frac{\Delta\Phi}{M_p}\ll1\Rightarrow r\lesssim0.0015.
\end{equation}
One might try to construct scenarios where the evolution of the tensor-to-scalar ratio violates slow-roll sufficiently to allow inflation scenarios with $\frac{\Delta\Phi}{M_p}\ll1$ to generate $r\sim\mathcal{O}(0.01)$ or higher \cite{BenDayan:2009kv,Hotchkiss:2011gz, Chatterjee:2014hna}. These attempts are under considerable pressure from {\it Planck observations}. {\bf Can we add a the stronger statement here?}

In summary, a conclusive detection with a CMB S4 instrument would provide strong evidence that the theory of quantum gravity must accommodate a shift symmetry for the inflaton. As discussed in Section \ref{sec:scale-of-inflation}, there may be some room for small-field inflation to generate a signal from non-vacuum fluctuations. Whether or not future constraints on non-Gaussianity completely close that window, the {\it absence} of a detection with CMB S4 will robustly rule out large-field inflation. 

{\it A detection of $r$, together with high confidence that the gravitational waves are sourced from vacuum fluctuations alone, would provide the only data point for the foreseeable future that weighs in on quantum gravity.}

\subsection{Bounding the mass of the graviton}

Theories of massive gravity come in many flavors (see e.g.~\cite{Dubovsky:2004sg,Hinterbichler:2011tt}), and their predictions in the scalar sector differ significantly. However, by definition, the dispersion relation for the graviton in all of them is
\begin{equation}
\omega^2=p^2+m_g^2\,,
\end{equation}
where $p$ is the physical momentum and $m_g$ the possibly time-dependent graviton mass. In particular, gravitational waves necessarily have frequencies $\omega>m_g$. A detection of primordial $B$-mode polarization on angular degree scales may be considered as a detection of gravitational waves with frequencies $\omega\sim H_{rec}$ through the quadrupole they produce in the primordial plasma, where $H_{rec}\approx 3\times 10^{-29}$~eV is the Hubble parameter at recombination. A detection then implies a model-independent bound $m_g<H_{rec}$ or 
\begin{equation}
m_g< 3\times 10^{-29}{\mbox{ eV}}\,.
\end{equation}
If the graviton mass is time-dependent, this should be interpreted as a constraint on the graviton mass around the time of recombination.

Because the perturbations in the primordial plasma before and around recombination are linear, the effect of the graviton mass is straightforward to incorporate by a simple modification of the field equation for tensor metric perturbations so that the above argument can be made more quantitative. The equation of motion for the transverse traceless metric perturbation $\gamma$ takes the same form as for a minimally coupled {\it massive} scalar field,
\begin{equation}
\label{massive}
\ddot{\gamma}_q(\tau)+2{\dot a\over a} \dot{\gamma}_q(\tau)+(q^2+m_g^2 a^2) \gamma_q(\tau)=0\,.
\end{equation}
Here $q$ is the comoving momentum of the metric perturbation, and we work in the conformal coordinates so that the background cosmological metric is
\begin{equation}
ds^2= a^2(\tau)(d\tau^2-d{\bf x}^2)\,.
\end{equation}
The consequences of this modification are discussed in detail in~\cite{Dubovsky:2009xk}. The most important consequence is that superhorizon modes start to oscillate around the time $\tau_m$ when $H(\tau_m)=m_g$, and their amplitude subsequently redshifts as $a^{-3/2}$. In contrast, in the massless case all modes remain frozen until they enter the horizon. This results in a suppression of the amplitude primordial $B$-mode for $m_g\gg H_{rec}$, and a detection of $B$-modes would rule out this possibility. For masses around $H_{rec}$, there is no suppression, but the angular power spectra are modified by the presence of a graviton mass, and a detection of primordial $B$-mode polarization would allow to measure the graviton mass. A detection of primordial gravitational waves with angular $B$-mode power spectrum consistent with that expected in general relativity would imply $m_g< 3\times 10^{-29}{\mbox{ eV}}$.
 
For comparison, the current model-independent bounds on the graviton mass arise from the indirect detection of $\sim 3\times 10^{-5}$~Hz gravitational waves through the timing of the Hulse-Taylor binary pulsar~\cite{PhysRevD.65.044022}, and the bound on the difference in arrival times for gravitational waves with different frequencies in the recent direct detection of astrophysical gravitational waves with LIGO~\cite{PhysRevLett.116.061102}. The resulting bounds are $m_g\lesssim 10^{-19}{\mbox{ eV}}$ and $m_g\lesssim 10^{-22}{\mbox{ eV}}$, respectively. A detection of $B$-mode polarization on angular degree scales consistent with the expectation in the context of general relativity would improve these bounds by nearly seven orders of magnitude. Measurements of $B$-mode polarization on the largest angular scales, possible only with a satellite, would further strengthen the bound. 

\subsection{Beyond $r$: Following up on a detection}
\label{sec:beyond_r}
The cosmic microwave background is not the only probe of primordial gravitational waves, and the tensor-to-scalar ratio is not the only parameter that describes the vacuum fluctuations. Additional details, including the scalar tilt and higher order correlations probe possible extensions beyond Einstein gravity and minimal coupling of the inflationary matter fields to gravity. Especially if the amplitude of gravitational waves is large enough to be measured in the CMB, it is worth considering whether there are additional features we might then hope to measure and what would be needed to do so. Although we do not find that these considerations affect the optimal design of CMB S4 ({\bf right?}), they may be important for distinguishing vacuum fluctuations from secondary sources of gravitational waves, and because of the complementarity with other future direct detection instruments.

\begin{itemize}
\item {\bf Distinguishing particle production sources from vacuum fluctuations}\\
{\bf suggest we move most of last paragraph about Lorenzo/Sorbo model here:}
In a regime in which the relation between the amplitude of tensor modes and the scale of inflation is strongly modified the B-mode anisotropies would be highly non-Gaussian. So in the case of a detection even the B-mode bispectrum would become observable with CMB-S4, and such a signal would not be confused with the vacuum fluctuations of the spacetime metric arising in single field slow-roll inflation. In particular, since the signal would be parity violating, the angular bispectrum of B-modes would be dominated by $\ell_1+\ell_2+\ell_3=$even, which would vanish in any theory that respects parity. \\

{\bf Also, text from Robert Caldwell}

\item {\bf Distinguishing inflation scenarios using the spectral tilt}\\
The simplest inflation scenarios all predict a red spectrum for gravitational waves. The single field consistency relation fixes $n_t=-8r$. For multiple fields, $n_t/r<-1/8$ instead \cite{Price:2014ufa}. However, some extensions can even produce a blue tilt \cite{}. A detection of primordial gravitational waves on CMB scales would allow precise predictions, especially relevant for the a blue index, for the amplitude expected on the much smaller scales accessible to direct detection. The recent detection of gravitational waves by LIGO, as well as the beginning of operation of the LISA pathfinder instrument, open an exciting new era of gravitational wave science. If CMB S4 also sees a signal, LIGO and future instruments may be particle physics detectors as well as astrophysical observatories. A recent analysis in \cite{Lasky:2015lej} shows the complementarity between observations over a wide range of scales in constraining the spectrum (although one must assume a constant tilt $n_T$ over many orders of magnitude). 

\item{\bf Probing gravity using non-Gaussianity}\\
The three-point function of gravitational waves would also be a sensitive probe of possible extensions of Einstein gravity \cite{}. In standard inflation, the three point correlations between tensors alone and tensors and scalars scale as $\langle\zeta\zeta\gamma\rangle\propto P_{\zeta}P_{\gamma}$, $\langle\zeta\gamma\gamma\rangle\sim\langle\gamma\gamma\gamma\rangle\propto P_{\gamma}P_{\gamma}$. In other words, $f^{\rm tensor}_{\rm NL}\sim 1$ \cite{}. Although direct detection experiments benefit from long integration times, one study suggested that even BBO would not be able to detect the standard three-point \cite{Adshead:2009bz} for $r<xxx$. Need a boost of at least xxx... Even with extensions it is difficult to see how such a signal could be obtained?

\end{itemize}


\section{Constraining the amplitude of primordial gravitational waves} 
\label{sec:upperLimits}
A detection of primordial gravitational waves has profound implications. Even excluding the presence of gravitational waves at a level observable by CMB-S4 has important consequences for the theory of inflation. Current constraints strongly disfavor models that were plausible candidates such as chaotic inflation with a quadratic potential~\cite{bicepkeckplanck15}. Upper limits from CMB-S4 would rule out entire classes of inflationary models. 

We first present a version of an argument developed in~\cite{Mukhanov:2013tua,Roest:2013fha,Creminelli:2014nqa} that does not rely on microscopic details of inflationary models. In the limit $\epsilon\ll1$, equations~(\ref{eq:inf_Pzeta}) and~(\ref{eq:specind}) lead to a differential equation
\begin{equation}\label{eq:epsdiffeq}
\frac{d\ln\epsilon}{d\mathcal{N}}-(n_s(\mathcal{N})-1)-2\epsilon=0\,,
\end{equation} 
where $\mathcal{N}$ is the number of e-folds until the end of inflation, and $n_s(\mathcal{N})-1$ denotes the spectral index evaluated for the mode which exits the horizon $\mathcal{N}$ e-folds before the end of inflation, {\it i.e.} for which $k=aH$ at that time. Note that $\epsilon$ is small (but positive) during inflation and $\epsilon\sim 1$ when inflation ends. If $\epsilon$ is a monotonic function of $\mathcal{N}$ this implies $n_s(\mathcal{N})-1\leq 0$ in agreement with observations. 

Denoting the number of e-folds before the end of inflation at which the CMB pivot scale exits the horizon as $\mathcal{N}_\star$, the constraint on the spectral index from the {\it Planck} satellite is $\mathcal{O}(1/\mathcal{N}_\star)$. While this could be a coincidence, it would find a natural explanation if 
\begin{equation}\label{eq:nsassump}
n_s(\mathcal{N})-1=-\frac{p+1}{\mathcal{N}}\,,
\end{equation}
up to subleading corrections in an expansion in large $\mathcal{N}$ for some real $p$. Under this assumption, the general solution to equation~(\ref{eq:epsdiffeq}) is
\begin{equation}\label{eq:epssol}
\epsilon(\mathcal{N})=\frac{p}{2\mathcal{N}}\frac{1}{1\pm\left(\mathcal{N}/\mathcal{N}_{\rm eq}\right)^{p}}\,,
\end{equation}
where we have chosen to parameterize the integration constant by $\mathcal{N}_{\rm eq}$ so that the magnitudes of the first and second term in the denominator become equal when $\mathcal{N}=\mathcal{N}_{\rm eq}$ . We take $\mathcal{N}_{\rm eq}>0$ and indicate the choice of sign for the integration constant by `$\pm$'. 

Assuming the epoch during which the modes we observe in the CMB exit is not special so that $\mathcal{N}_\star\gg\mathcal{N}_{\rm eq}$ or $\mathcal{N}_\star\ll\mathcal{N}_{\rm eq}$, equation~(\ref{eq:epsdiffeq}) leads to four classes of solutions
\begin{eqnarray}
{\rm I.}&&\epsilon(\mathcal{N})=\frac{p}{2\mathcal{N}}\,,\\
{\rm II.}&&\epsilon(\mathcal{N})=\frac{p}{2\mathcal{N}}\left(\frac{\mathcal{N}_{\rm eq}}{\mathcal{N}}\right)^p\hspace{1.92cm}\qquad{\rm with}\qquad \hspace{7.3mm}p>0 \hspace{1.0cm}\qquad{\rm and}\qquad\mathcal{N}_{\rm eq}\ll\mathcal{N}_\star\,,\\
{\rm III.}&&\epsilon(\mathcal{N})=\frac{|p|}{2\mathcal{N}}\left(\frac{\mathcal{N}}{\mathcal{N}_{\rm eq}}\right)^{|p|}\hspace{1.77cm}\qquad{\rm with}\qquad \hspace{7.3mm}p<0 \hspace{1.0cm}\qquad{\rm and}\qquad\mathcal{N}_{\rm eq}\gg\mathcal{N}_\star\,,\\
{\rm IV.}&&\epsilon(\mathcal{N})=\frac{1}{2\mathcal{N}\ln\mathcal{N}_{\rm eq}/\mathcal{N}}+\frac{p}{4\mathcal{N}}+\dots\qquad{\rm with}\qquad |p|\ll\frac{1}{\ln\mathcal{N}_{\rm eq}/\mathcal{N}_\star}\qquad{\rm and}\qquad \mathcal{N}_{\rm eq}\gg\mathcal{N}_\star\,.
\end{eqnarray}

For a given $\mathcal{N}_\star$ classes I and IV naturally lead to larger $r= 16\epsilon(\mathcal{N}_\star)$ than classes II and III. Upper limits on the amount of primordial gravitational would then favor classes II and III and strongly constrain or exclude I and IV. The value of $\mathcal{N}_\star$ depends on the post-inflationary history of the universe. Equation~(\ref{eq:nsassump}) implies that a measurement of the spectral index and its running would determine $p$ and hence $\mathcal{N}_\star$, but unfortunately such a measurement is out of reach for CMB-S4. A given reheating scenario predicts $\mathcal{N}_\star$, but the space of reheating scenarios is large. Instantaneous reheating leads to $\mathcal{N}_\star\approx 57$ for $k_\star=0.05 {\rm Mpc}^{-1}$, smaller values correspond to less efficient reheating. We will assume $47<\mathcal{N}_\star<57$ for the following discussion. Current constraints on $n_s$ and $r$ from~\cite{bicepkeckplanck15} disfavor class III at just over $2\sigma$ relative to class II. Furthermore, the best-fit of class III occurs for $p\approx 0$ where classes I, II, and III degenerate so that class III need not be discussed seperately. Class IV is disfavored at $2-3\sigma$ relative to class II, and we focus on the first two classes in what follows.

For the first class, constraints on $r$ directly translate into constraints on $p$.

Discussion of second class

Corresponding potentials and implications

\section{CMB data products and simulations required to achieve goals for PGW}
\label{sec:needs}

\section{Improved constraints on the particle content of the primordial universe}
\label{sec:scalar}

\begin{itemize}
\item {\bf The scalar power spectrum: running, wiggles}\\
{\bf To add: we don't see any of this driving the science, so this addition will be minimal. }
 
\item {\bf Non-Gaussianity}\\
The present best constraints on local non-Gaussianity come from the {\it Planck} 2015 temperature and polarization analysis and give $f_{\rm NL}^{\rm local} = 0.8 \pm 5.0$ and $f_{\rm NL}^{\rm equil} = -4 \pm 43$(68\% CL) \cite{Ade:2015ava}. A noise-free cosmic variance limited CMB experiment is expected to produce constraints on $f_{\rm NL}$ with 1$\sigma$ error bars of about 3 \cite{Komatsu:2001rj}. Therefore the best that can be expected of CMB Stage-IV is slightly less than a factor of two improvement on the current best limits. This is not sufficient to reach the interesting theoretical threshold around $|f_{\rm NL}^{\rm any}|<\mathcal{O}(1)$ \cite{Alvarez:2014vva}, but will still reduce the space of viable models or hint at a detection.

Even such a factor of two improvement on $f_{\rm NL}^{\rm equil}$ would further constrain scenarios with secondary production of gravitational waves of comparable strength to the vacuum signal from fields coupled to the inflaton....{\bf be specific}

A detection of local non-Gaussianity would have far reaching theoretical implications. First, any significant detection of $f_{\rm NL}^{\rm local}$ would rule out all models of single clock inflation \cite{Creminelli:2004yq}. In addition, such a signal would open the door to significant cosmic variance on all scales from coupling of fluctuations within our observed volume to any super-Hubble modes \cite{}. Although any scenario that predicts local non-Gaussianity together with fluctuations on scales much larger than our observed volume predicts a probability distribution for our observed $f_{\rm NL}^{\rm local}$, there are well-motivated models which predict a small mean value at a level where CMB Stage-IV could either hint toward, or slightly disfavor at around the level of 2$\sigma$. These models include the simplest modulated reheating scenario \cite{Zaldarriaga:2003my} and ekpyrotic cosmology \cite{Lehners:2009ja}, both of which predict mean values of $f_{\rm NL}^{\rm local}\sim5$. The simplest curvaton scenario, which predicts $f_{\rm NL} = -5/4$ \cite{Lyth:2001nq}, will unfortunately be out of reach of CMB Stage-IV.

%In the modulated reheating scenario, the field which drives inflation $\Phi$ decays to the particles of the standard model with a rate $\gamma$ which is determined by the value of a second field $\sigma$ which remains light throughout inflation. The quantum fluctuations in $\sigma$ result in a spatially modulated reheating surface resulting in the curvature perturbations that we observe in the CMB and large scale structure. The process by which the fluctuations in the light field are converted into curvature fluctuations naturally results in local non-Gaussianity given by $f_{\rm NL} = 5(1-\Gamma \Gamma^{\prime\prime}/\Gamma^{\prime2})$, where this formula holds in the case that $\Phi$ oscillates about a quadratic minimum after inflation and the fluctuations in $\Phi$ make a negligible contribution to the observed power spectrum.

%This can be contrasted with the simplest curvaton scenario, where a scalar field $\sigma$ which remains light during inflation comes to dominate the energy density of the universe after the field which drives inflation $\Phi$ decays. The fluctuations in the energy density of $\sigma$ then determine the curvature perturbations that are observed today. The local non-Gaussianity in this simple model is predicted to be $f_{\rm NL} = -5/4$ \cite{Lyth:2001nq}, which is unfortunately a few times smaller than the expected error bar from CMB Stage-IV.

%In the absence of a detection, however, it is important to ask what can be learned from improved constraints on $f_{\rm NL}$. Though not firm, nor entirely robustly defined, it can be argued that a natural theoretical threshold where qualitatively new general conclusions about the physics of the early universe can be drawn would come from constraints on $f_{\rm NL}<\mathcal{O}(1)$, see for example [1412.4671] for a detailed discussion. In order to achieve this level of constraint, it seems necessary to move beyond the cosmic microwave background to study other data sets, such as large scale structure. Despite the fact that CMB Stage-IV is not expected to reach this threshold, it is worth asking what can be gleaned from an improved constraint on $f_{\rm NL}$ from the CMB.


\item {\bf Isocurvature}
\end{itemize}



\section{Improved constraints on spatial curvature, anomalies, birefringence, cosmic strings, axions\dots}
\label{sec:speculative}
{\bf Expect text from Vera Gluscevic, Glenn Starkman, Dominik Schwarz, ....}\\
{\bf Add a paragraph on cosmic strings}\\
{\bf Add connections to axion physics. eg 1601.03049}
%%
%% Populate the .bib file with entries from SPIRES Bibtex (preferred)
%% or ADS Bibtex (if no SPIRES entry).
%%  SPIRES will also supply the CITATION line information; please include it.
%%


