%%%%%% CMB-S4 Dark Energy Template  %%%%%%%%%%%%%%%%
 \documentclass[11pt]{article}
\usepackage{graphicx}
\usepackage{amssymb}
\usepackage{epstopdf}
\DeclareGraphicsRule{.tif}{png}{.png}{`convert #1 `dirname #1`/`basename #1 .tif`.png}

\textwidth = 6.5 in
\textheight = 9 in
\oddsidemargin = 0.0 in
\evensidemargin = 0.0 in
\topmargin = 0.0 in
\headheight = 0.0 in
\headsep = 0.0 in
\parskip = 0.2in
\parindent = 0.0in

\newtheorem{theorem}{Theorem}
\newtheorem{corollary}[theorem]{Corollary}
\newtheorem{definition}{Definition}



\begin{document}

\def\bibname{References}

\bibliographystyle{utphys}  %%%% MODIFIED FOR CF %%%%
%\bibliographystyle{plain}

\raggedbottom

\pagenumbering{roman}

%\parindent=0pt
%\parskip=8pt
%\setlength{\evensidemargin}{0pt}
%\setlength{\oddsidemargin}{0pt}
%\setlength{\marginparsep}{0.0in}
%\setlength{\marginparwidth}{0.0in}
%\marginparpush=0pt

% The content begins here

\pagenumbering{arabic}
\textwidth = 7.5 in
\textheight = 9 in
\oddsidemargin = 0.0 in
\evensidemargin = 0.0 in
\topmargin = 0.0 in
\headheight = 0.0 in
\headsep = 0.0 in
\parskip = 0.2in
\parindent = 0.0in

%\renewcommand{\chapname}{chap:intro_}
%\renewcommand{\chapterdir}{.}
%\renewcommand{\arraystretch}{1.25}
%\addtolength{\arraycolsep}{-3pt}
% 
% define chapter in the template separately so that we don't need to pass style files
\newcommand{\chapter}[1]{{\begin{center}
{\Huge\bf #1}\end{center}}}
%\newcommand{\chapter}[1]{{\title{#1}}}

%%%% Author comment macros here

\def\as#1{[{\bf AS:} {\it #1}] }
\def\gtrsim{\raise-.75ex\hbox{$\buildrel>\over\sim$}}
\renewcommand*\thesection{\arabic{section}}



%%%%%%%%%%%%%%%%%%%%%%%%%%%%%%%%%%%%%%%%%%%%%%%%%%%
%%%%%%%%%%%%%%%%%%%%%%%%%%%%%%%%%%%%%%%%%%%%%%%%%%%
%%%     All of your files should be in a subdirectory.  Here the
%%%     subdirectory is called CF0. The title of your
%%%     report should be wgreportCF0.tex in that subdirectory.  Input
%%%     that file here
%%%%%%%%%%%%%%%%%%%%%%%%%%%%%%%%%%%%%%%%%%%%%%%%%%%%
%%%%%%%%%%%%%%%%%%%%%%%%%%%%%%%%%%%%%%%%%%%%%%%%%%%

%\input Intro/intro_cmbs4.tex
%\input Inflation/inflation_cmbs4.tex 
%\input Neutrinos/neutrinos_cmbs4.tex 
%\input DarkEnergy/darkenergy_cmbs4.tex 
%\input CMBLensing/cmblensing_cmbs4.tex 
%\input Instrumentation/instrumentation_cmbs4.tex 
%\input Analysis/analysis_cmbs4.tex 


%\input conclusions_cmbs4.tex

%%%%%%%%%%%%%%%%%%%%%%%%%%%%%%%%%%%%%%%%%%%%%%%%%%
%%%%%%%%%%%%%%%%%%%%%%%%%%%%%%%%%%%%%%%%%%%%%%%%%%
%%%   Your subdirectory (here CF0) should include
%%%    the files:
%%%           wgreportCF0.tex
%%%           authorlistCF0.tex
%%%           bibCF0.bib
%%%         and all needed figures in pdf format
%%%%%%%%%%%%%%%%%%%%%%%%%%%%%%%%%%%%%%%%%%%%%%%%%%%%
%%%%%%%%%%%%%%%%%%%%%%%%%%%%%%%%%%%%%%%%%%%%%%%%%%%%

\chapter{3. Dark Energy and Modified Gravity}

%%%%%%%%%%%%%%%%%%%%%%%%%%%%%%%%%%%%%%%%%%%%%%%%%%%%%%%%%%%
%%%%%%%%%%%%%%%%%%%%%%%%%%%%%%%%%%%%%%%%%%%%%%%%%%%%%%%%%%%
%%%%%%%%%%%%%%%%%%%%%%%%%%%%%%%%%%%%%%%%%%%%%%%%%%%%%%%%%%%
%%%%%%%%%%%%%%%%%%%%%%%%%%%%%%%%%%%%%%%%%%%%%%%%%%%%%%%%%%%

\section{Introduction}

Boilerplate material on cosmic acceleration and dark energy vs modified gravity.  Then expand on points in the global introduction, reproduced here.

 The CMB can be used to investigate Dark Energy through growth of structure tests, i.e., CMB lensing and SZ clusters, and through testing Gravity on large scales, i.e., though exploiting the kinematic SZ effect to measure the momentum field and large scale flows. The power of these probes is amplified by combining CMB-S4 data with galaxy surveys and Lyman alpha surveys, such as DESI, LSST, Euclid and WFIRST.

 a) CMB lensing maps from CMB-S4 will provide hi-fidelity projected mass maps that will be cross-correlated with optical survey maps. This will increase the reach and precision of the dark energy constraints, as well as provide independent checks. Papers in the literature have quantified the DE FOM improvement of various projects with the addition of CMB lensing. Simulations need to be done to quantify the projected improvements with CMB-S4.

 b) The DE task force pointed out that galaxy cluster evolution had the highest 
 sensitivity of the DE probes considered. However, it also had the largest systematic. The issue is the uncertainty in understanding the mass scaling of the cluster observable. The thermal SZ effect has now been demonstrated to be a low scatter observable  with the extraordinary feature of its brightness being redshift independent; an SZ survey probes all redshifts to a limiting mass. However, there still remain large uncertainties in the SZ observable mass scaling. CMB-S4 will be revolutionary in that it is expected to be able to calibrate the mass scaling to better than 1\% through CMB lensing. This coupled with a low mass threshold will enable CMB-S4 to identify of order 100,000 clusters, probe the growth of structure to redshifts beyond $z \sim 2.5$, and will allow CMB-S4 to realize the full potential of galaxy clusters as a probe of dark energy.  In combination with other Stage-IV baryon acoustic oscillation, supernova, and weak lensing surveys, a Stage-IV cluster survey similar to CMB-S4 would improve the overall dark energy figure of merit to approximately 1250, nearly a factor of two improvement than achieved without clusters. 

 c) Testing GR on large scales is important for our understanding of dark energy and the underlying workings of space and matter in general.  The kinematic SZ effect allows measurement on the peculiar velocity (departure from Hubble flow) of structures. By measuring the differences in kSZ between pairs of clusters with known redshifts (a synergy of CMB-S4 and optical surveys), gravity can be tested on scales of 100~Mpc and larger. In this way, CMB-S4 paired with a Stage-IV spectroscopic survey would improve constraints on the growth rate predicted by general relativity by a factor of two.
 
 
\section{Growth of structure}

\subsection{Cluster abundance and mass}

Brad Benson?


With sufficient angular resolution, CMB-S4 opens tremendous possibilities for measuring cluster, more generally halo masses through CMB lensing.  This is essential in particular for using cluster counts as a probe of dark energy and modified gravity.  Figure XX shows an estimate of the mass sensitivity, expressed as the one-sigma mass uncertainty as a function of redshift, for two possible S4 configurations and compares them to planned CMB experiments. 
	The estimation is made assuming foreground subtraction to reach the quoted CMB map noise level at the given angular resolution.  The method (Melin & Bartlett 2015) employs an optimal filter matched to the NFW profile and applied to reconstructions of the lensing potential with a quadratic estimator (Hu & Okamoto 2002).  The method has already been successfully applied to the Planck cluster cosmology sample of more than 400 objects (Planck Collaboration XXIV 2015).  This figure shows the sensitivity obtained with just CMB temperature lensing reconstruction.  Including polarization will significantly improve it.  
	We see that the mass sensitivity remains flat with redshift, a remarkable property that enables mass estimation out to redshifts unreachable with galaxy shear measurements.  This is a powerful and unique capability of CMB lensing.  The figure also demonstrates the important gains attained with high angular resolution.  At one arcmin resolution, S4 achieves a mass sensitivity of 2e13 solar masses with temperature alone.  This unprecedented sensitivity not only ensures robust mass estimation for cluster cosmology over the large redshift interval where S4 will detect clusters through the SZ effect, but also paves the way to numerous cluster and large-scale structure studies.  The rapidly increasing science reach enabled by high angular resolution, approaching one arcmin, is an important consideration in S4 objectives.  



\subsection{Lensing}

Suzanne Staggs? 

\subsection{Kinetic SZ}

de Bernardis

\subsection{ISW ?}

omit? 

\section{Forecasts}

\subsection{Parameterized Dark Energy}

Paragraphs from Marco on EFT parameters.   Probably need to forecast some DETF FoM.  Do we forecast slip, growth rate or any other phenomenological parameter?

\subsection{Forecasts}

Set up fisher forecast methodology for completion with results 

%\bibliography{cmbs4}

%%
%% Populate the .bib file with entries from SPIRES Bibtex (preferred)
%% or ADS Bibtex (if no SPIRES entry).
%%  SPIRES will also supply the CITATION line information; please include it.
%%

\chapter{4. Dark Matter}

Cora Dvorkin  and Francis-Yan Cyr-Racine


 \end{document}