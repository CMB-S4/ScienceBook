%%%%%% CMB-S4 Neutrinos Chapter  %%%%%%%%%%%%%%%%
 
\chapter{Dark Energy / Gravity / Dark Matter}
\renewcommand*\thesection{\arabic{section}}

\def\gtrsim{\raise-.75ex\hbox{$\buildrel>\over\sim$}}
%%%%%%%%%%%%%%%%%%%%%%%%%%%%%%%%%%%%%%%%%%%%%%%%%%%%%%%%%%%
%%%%%%%%%%%%%%%%%%%%%%%%%%%%%%%%%%%%%%%%%%%%%%%%%%%%%%%%%%%
%%%%%%%%%%%%%%%%%%%%%%%%%%%%%%%%%%%%%%%%%%%%%%%%%%%%%%%%%%%
%%%%%%%%%%%%%%%%%%%%%%%%%%%%%%%%%%%%%%%%%%%%%%%%%%%%%%%%%%%

\section{Introduction}

Here is an intro.  Test commit

\section{Dark Matter}
\subsection{Dark Matter Annihilation}

{\bf Cora is currently writing this section.}

\subsection{Non-Standard Dark Matter Physics}

Near the epoch of CMB last scattering, dark matter (DM) accounts for about $65\%$ of the energy budget of the Universe, hence making the CMB a particularly good probe of potential new physics in the DM sector. Of particular relevance to CMB studies, the presence of new DM interactions with light degrees of freedom \cite{Goldberg:1986nk,1992ApJ...398...43C,1992ApJ...398..407G,1994ApJ...431...41M,1995ApJ...452..495D,AtrioBarandela:1996ur,Green:2005fa,Profumo:2006bv,Bringmann:2009vf,Boehm:2001hm,Mangano:2006mp,Serra:2009uu,McDermott:2010pa,Aarssen:2012fx,Wilkinson:2013kia,Wilkinson:2014ksa,Dvorkin:2013cea,Boehm:2014vja,Escudero:2015yka,Foot:2004pa,Ackerman:2008gi, ArkaniHamed:2008qn,Feng:2009mn,Kaplan:2009de,Behbahani:2010xa,Kaplan:2011yj,Aarssen:2012fx,Cline:2012is,Hooper:2012cw,Das:2012aa,Cyr-Racine:2013ab,Diamanti:2012tg,Baldi:2012ua,Fan:2013yva,Fan:2013tia,McCullough:2013jma,Cyr-Racine:2013fsa,Cline:2013pca,Cline:2013zca,Bringmann:2013vra,Chu:2014lja,Archidiacono:2014nda,Randall:2014kta,Buen-Abad:2015ova,Lesgourgues:2015wza,Choquette:2015mca} can leave subtle imprints on the temperature and polarization CMB power spectra. The introduction of such non-minimal DM models has been primarily (but not exclusively) motivated in the literature by potential shortcomings of the standard cold DM scenario at small sub-galactic scales \cite{deBlok:1997zlw,Klypin:1999uc,Moore:1999aa,Zavala:2009ms,Oh:2010ea,BoylanKolchin:2011de,Papastergis:2011xe,Walker:2011zu,Pawlowski01112013,Klypin:2014ira,Oman:2015xda,Papastergis2015de}. While these issues are far from settled, they motivate the search for other non-minimal DM signatures in complementary data sets (such as the CMB) that could indicate whether or not DM can be part of the solution. In addition, DM interacting with light (or massless) dark radiation (DR) has been put forward \cite{Buen-Abad:2015ova,Lesgourgues:2015wza} has a potential solution to the small discrepancy between the amplitude of matter fluctuations inferred from CMB measurements and those inferred from cluster number counts and weak lensing measurements. As we discuss below, CMB-S4 measurements of the CMB lensing and the lensed B-mode power spectra have the potential to significantly improve constraints on DM interacting with light degrees of freedom in the early Universe.

The key equations governing the evolution of cosmological fluctuations for this broad class of non-minimal DM models are presented in Ref.~\cite{Cyr-Racine:2015ihg}. Essentially, the new DM physics enters entirely through the introduction of DM and DR opacities, which, similarly to the photon-baryon case, prohibit DR free-streaming at early times and provides a pressure term that opposes the gravitational growth of DM density fluctuations. The impact of this new physics on CMB fluctuations has been studied in detail in Ref.~\cite{Cyr-Racine:2013fsa} and we briefly review it here. First, the presence of extra DR mimics the presence of extra neutrino species and affects the expansion history of the Universe, possibly modifying the epoch of matter-radiation equality, the CMB Silk damping tail, and the early Integrated Sachs-Wolfe effect. However, unlike standard free-streaming neutrinos, the DR forms a tightly-coupled fluid at early times, leading to distinct signatures on CMB fluctuations which include a phase and amplitude shift of the acoustic peaks (see e.g. Ref.~\cite{Bashinsky:2003tk,Cyr-Racine:2013jua,Follin:2015hya}). Second, the DR pressure prohibits the growth of interacting DM fluctuations on length scales entering the causal horizon before the epoch of DM kinematic decoupling. This weakens the depth of gravitational potential fluctuations on these scales, hence affecting the source term of CMB temperature fluctuations. Finally, the modified matter clustering in the Universe due to nonstandard DM properties will affect the lensing of the CMB as it travel from the last-scattering surface to us.

{\bf CD: More comments:

- At what ell does the DR mainly affect the CMB?

- How important is the effect on the B-mode lensing? Could you be a bit more quantitative here?

- How relevant would it be for CMB-S4? You should think about a full-sky experiment with 1-4 arcmins beam and a detector noise of DeltaP~0.5-1 muK-arcmin.

- How big is the degeneracy expected to be with neutrino mass?

- Mention that these DM constraints are a lot more sensitive to $\ell_{\rm max}$ than $\ell_{\rm min}$.}

{\bf Briefly discuss the possible degeneracy between the impact of interacting DM and massive neutrinos on the matter power spectrum.}

\subsection{Ultralight axions}

Ultralight axions (ULAs) with masses in the range $10^{-33}~{\rm eV}\leq m_{a}\leq 10^{-20}~{\rm eV}$ are well motivated by string theory, can contribute to either the dark matter or dark energy components of the Universe, depending on their masses, and are distinguishable from DE and CDM in cosmological observables. The current best constraints from the primary CMB TT power, and WiggleZ galaxy redshift survey were made in Hlozek et al (2015).

The degeneracies of the axions with other cosmological parameters, such as $N_{\mathrm{eff}}$ or neutrino mass $\Sigma$ $m_{\nu}$ vary depending on the axion mass assumed. Dark energy-like axions with masses around $10^{-33}~{\rm eV}$ change the late-time expansion rate and therefore the sound horizon, changing the location of the acoustic peaks. This has degeneracies with the matter and curvature content.
 Axions that behave more like dark matter (for masses around $m_a \gtrsim 10^{-26}~{\rm eV}$) start oscillating in the radiation era and reduce the angular scale of the diffusion distance, leading to a boost in the higher acoustic peaks, which has a degeneracy with $N_\mathrm{eff}$.
 In both of these cases, improved errors on the temperature and polarization power spectrum, coupled with constraints on the Hubble constant (for the lightest axions) from Baryon Acoustic Oscillations, lead to improvements in the error on allowed axion energy density of a factor of three.
In the matter power spectrum, and thus CMB lensing power, light axions suppress clustering power, suggesting a degeneracy with effects of massive neutrinos that must be broken to make an unambiguous measurement of neutrino mass using the CMB. The above mentioned effects in the expansion rate likely break this degeneracy.
 
 Adding in information from the lensing reconstruction using S4 will improve constraints on axion DM significantly, and allow one to break the degeneracy between dark-matter like axions and massive neutrinos. A percent-level measurement of the lensing deflection power at multipoles L $>$ 1000 leads to an improvement in the error on the axion energy density of a factor of eight relative to the current Planck constraints, for an axion mass of $m_a=10^{-26}~{\rm eV}$. Furthermore, S4 will improve the maximum axion mass constrainable by the CMB by up to two orders of magnitude from $m_a=10^{-24}~{\rm eV}$ to $m_a=10^{-22}~{\rm eV}$. Achieving this improvement in the mass constraint will require improved understanding of non-linear clustering of axions. If realized it will begin to make contact to the preferred axion model to solve the CDM small scale crises (Hu et al, 2000; Marsh \& Silk, 2013).


%\bibliography{cmbs4}

%%
%% Populate the .bib file with entries from SPIRES Bibtex (preferred)
%% or ADS Bibtex (if no SPIRES entry).
%%  SPIRES will also supply the CITATION line information; please include it.
%%


