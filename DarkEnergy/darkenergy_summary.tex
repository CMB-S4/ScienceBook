The discovery almost 20 years ago that the expansion of the universe is accelerating presented a profound challenge to our laws of physics, one that we have yet to conquer. Our current framework can explain these observations only by invoking a new substance with bizarre properties ({\it dark energy}) or by changing the century-old, well-tested theory of general relativity invented by Einstein. The current epoch of acceleration is much later than the epoch from which the photons in the CMB originate, and the behavior of dark energy or modifications of gravity do not significantly influence the properties of the primordial CMB. However, during their long journey to our telescopes, CMB photons occasionally interact with the intervening matter and can have their trajectories and their energies slightly distorted. These distortions---gravitational lensing by intervening mass and energy gain by scattering off hot electrons---are small, but powerful experiments currently online have already detected them, and CMB-S4 will exploit them to the fullest extent, enabling us to learn about the mechanism driving the current epoch of acceleration.

The canonical model is that acceleration is driven by a cosmological constant. Although theoretically implausible, this model does satisfy current constraints, so a simple target for CMB-S4 is to test the many predictions this model makes at late times. Using the gravitational lensing of the CMB, the abundance of galaxy clusters, and cosmic velocities, CMB-S4 will measure both the expansion rate $H$ and the amount of clustering, quantified by the parameter $\sigma_8$, as a function of time. The constraints from CMB-S4 alone will be at the sub-percent level on each, and when combined with other experiments will reach below a tenth of a percent, particularly when the power of CMB-S4 is also harnessed to calibrate these other probes. These constraints will be among the most powerful tests of the cosmological constant; more crucially, this simultaneous sensitivity to expansion and growth will allow us to distinguish the dark energy paradigm from a failure of general relativity. Models for acceleration in this latter class abound, and CMB-S4 will constrain the parameters of these as well. 
