Dark matter is required to explain a host of cosmological observations such as the velocities of galaxies in galaxy clusters, galaxy rotation curves, strong and weak lensing measurements, and the acoustic peak structure of the CMB. While most of these observations could be explained by non-luminous baryonic matter, the CMB provides overwhelming evidence that $85\%$ of the matter in the Universe is non-baryonic, presumably a new particle never observed in terrestrial experiments. Because dark matter has only been observed through its gravitational effects, its microscopic properties remain a mystery. Identifying its nature and its connection to the rest of physics is one of the prime challenges of high energy physics.

Weakly interacting massive particles (WIMPs) are one well-motivated candidate that naturally appears in many extensions of the standard model. A host of experiments are hoping to detect them: deep underground ton-scale detectors, gamma-ray observatories, and the Large Hadron Collider. The CMB provides a complementary probe through annihilation of dark matter into Standard Model particles.   In the WIMP paradigm, the processes that allow dark matter to be created often allow the particles to annihilate with one another. The rate for this process governs how many of the particles remain today and, for a given WIMP mass, is well-constrained by the known dark matter abundance. The same annihilation process injects a small amount of energy into the CMB, slightly distorting its anisotropy power spectrum. CMB-S4 will probe dark matter masses a few times larger than those probed by current CMB experiments.% and will be sensitive to the sweet spot of WIMP masses around $100$ GeV.

Dark matter need not be heavy or thermally produced. Axions provide one compelling example that appears in many extensions of the standard model and is often invoked as solution to some of the most challenging problems in particle physics. Although axions are often extremely light, they can naturally furnish some or all of the dark matter non-thermally. Their effects on the expansion rate of the Universe, on the clustering, and on the local composition of the Universe through quantum fluctuations in the axion field all lead to subtle modifications of the CMB and lensing power spectra. CMB-S4 will improve current limits by as much as an order of magnitude and for some range of masses would be sensitive to axions contributing as little as $1\%$ to the energy density of dark matter.  As for WIMPs, there is an active program of direct and indirect experimental searches for axions that will complement the CMB, and the interplay can reveal important insights into both axions and cosmology.

