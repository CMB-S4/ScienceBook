The distribution of matter in the Universe contains a wealth of information about the primordial density perturbations and the forces that have shaped our cosmological evolution. Mapping this distribution is one of the central goals of modern cosmology. Gravitational lensing provides a unique method to map the matter between us and distant light sources, and lensing of the CMB, the most distant light source available, allows us to map the matter between us and the surface of last scattering.

Gravitational lensing of the CMB can be measured because the statistical properties of the primordial CMB are exquisitely well-known. As CMB photons travel to Earth from the last scattering surface, they are deflected by intervening matter which distorts the observed pattern of CMB anisotropies and modifies their statistical properties. This can be used to create a map of the gravitational potential that altered the photons' paths. The gravitational potential encodes information about the formation of structure in the Universe and, indirectly, cosmological parameters like the sum of the neutrino masses.  CMB-S4 is expected to produce high fidelity maps over large fractions of the sky, improving on the signal-to-noise of  the \planck\ lensing maps by more than an order of magnitude.  These maps will inform many of the science targets discussed throughout the book and can also be used to calibrate and enhance results of upcoming galaxy redshift surveys or any other maps of the matter distribution.  Unfortunately, lensing also obscures our view of the CMB.  By measuring and removing the effects of lensing from the CMB maps, we sharpen our view of primordial gravitational waves and our understanding of the very early Universe more generally.  
