\begin{center}
  {\Large \bf Executive Summary}
\end{center}

We describe the scientific case for the next generation ground-based cosmic microwave background experiment, CMB-S4, consisting of dedicated telescopes at the South Pole, the high Chilean Atacama plateau and possibly a northern hemisphere site, all equipped with new superconducting cameras that will provide a dramatic leap forward in cosmological studies, crossing critical thresholds in testing inflation, the number and masses of the neutrinos or the existence of other `dark radiation', providing precise constraints on the nature of dark energy, and testing general relativity on large scales. 

CMB-S4 will be the definitive ground-based CMB project. It will deliver the data set with which any model for the origin of the primordial fluctuations, be it inflationary or an alternative theory, must be consistent with to be viable.  CMB-S4 will allow an investigation of Inflation by making ultra-sensitive polarization measurements of the recombination bump at degree angular scales in search of the B-mode signature imprinted by primordial gravitational waves. It will allow the determine the neutrino masses by making unprecedentedly precise reconstruction of the matter power spectrum through from the correlations induced on the CMB from the gravitational lensing of large scale structure, and it will allow the determine the effective number of neutrino-like species to high accuracy through detailed constraints on the energy density in the early Universe as revealed in the temperature and polarization angular power spectra.
%, allowing stringent constraints on additional neutrino species, energy injection of decaying particles, and a thorough test of the evolution from neutrino decoupling at one second, at Big Bang Nucleosynthesis at a few minutes, to photon decouping of the 370,000 years.
It will be used to investigate dark energy through growth of structure tests, i.e., CMB lensing and SZ clusters, and through testing Gravity on large scales, i.e., though exploiting the kinematic SZ effect to measure the momentum field and large scale flows. The power of these probes is amplified by combining CMB-S4 data with galaxy surveys and Lyman alpha surveys, such as DESI, LSST, Euclid and WFIRST.

There is only one CMB sky. It holds a wealth of information on fundamental physics and the origin and evolution of the Universe. While we have learned a great deal from CMB measurements, including discoveries that have pointed the way to new physics, we have only begun to tap the information contained in CMB polarization, CMB lensing and secondary effects. CMB-S4 will  maximize discovery space by producing high fidelity maps of long lasting legacy value.

Through the efforts of the CMB experimental groups over the last decade, the technologies needed for CMB-S4 are now in place. There are, however, considerable technical challenges presented by the required scaling up of the instrumentation as well as by the scope and complexity of the data analysis and interpretation.  CMB-S4 will require: scaled up superconducting detector arrays with well understood and robust material properties and processing techniques; high throughput mm-wave telescopes and optics with unprecedented precision and rejection of systematic contamination; full internal characterization of astronomical foreground emission; large cosmological simulations and theoretical modeling with accuracies yet to be achieved; and computational methods for extracting minute correlations in massive, multi-frequency data sets contaminated by noise and a host of known and unknown signals. 


CMB-S4 is the logical successor to the ongoing Stage 3 CMB projects and should be implemented on a time scale that allows a transition from Stage 3 to Stage 4 that is as seamless and timely as possible,  thereby ensuring a continued  stream of CMB science results and the maximum synergistic enhancement of the science return from other contemporaneous cosmic surveys (e.g., LSST, DESI).  This timing also has the advantage of capitalizing on the momentum of the field and community.  This timing is possible as CMB-S4 will use existing technology optimized for scaling up, but with no major downselects expected.  Information learned from ongoing Stage 3 experiments (e.g., foregrounds, level of the primordial tensor to scalar ratio) can be easily incorporated into the CMB-S4 survey strategy with little or no impact on its design. In fact, the sensitivity and fidelity of the foreground measurements needed to realize the goals of CMB-S4 will only be provided by CMB-S4 itself, at frequencies just below and above the main CMB channels.
\eject

 
