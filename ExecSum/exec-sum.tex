\begin{center}
  {\Large \bf Executive Summary}
\end{center}

The next generation ``Stage-4" ground-based cosmic microwave background (CMB) experiment, CMB-S4, consisting of dedicated telescopes equipped with highly sensitive superconducting cameras operating at the South Pole, the high Chilean Atacama plateau, and possibly northern hemisphere sites, will provide a dramatic leap forward in our understanding of the fundamental nature of space and time and the evolution of the Universe. CMB-S4 will be designed to cross critical thresholds in testing inflation, determining the number and masses of the neutrinos, constraining possible new light relic particles, providing precise constraints on the nature of dark energy, and testing general relativity on large scales. 

CMB-S4 is intended to be the definitive ground-based CMB project. It will deliver the data set with which any model for the origin of the primordial fluctuations---be it inflation or an alternative theory---and their evolution to the structure seen in the Universe today must be consistent. %with to be viable.  
While we have learned a great deal from CMB measurements, including discoveries that have pointed the way to new physics, we have only begun to tap the information encoded in CMB polarization, CMB lensing and other secondary effects.  The discovery space from these and other yet to be imagined effects will be  maximized by designing CMB-S4 to produce high-fidelity maps, which will also ensure enormous legacy value for CMB-S4.

%CMB-S4 will allow an investigation of inflation by making ultra-sensitive polarization measurements of the recombination bump at degree angular scales and the requisite measurements at arc minutes scales for de-lensing to search for the B-mode signature imprinted by primordial gravitational waves. It will allow the determination of the neutrino masses by making unprecedentedly precise reconstruction of the matter power spectrum by exploiting the correlations induced on the CMB from the gravitational lensing of large scale structure, and it will allow the determination of the effective number of neutrino-like species to high accuracy through detailed constraints on the energy density in the early Universe as revealed in the temperature and polarization angular power spectra.It will be used to investigate dark energy through growth of structure tests, i.e., CMB lensing and SZ clusters, and through testing gravity on large scales, i.e., though exploiting the kinematic SZ effect to measure the momentum field and large scale flows. The power of these probes is amplified by combining CMB-S4 data with galaxy surveys and Lyman alpha surveys, such as DESI, LSST, Euclid and WFIRST.

%There is only one CMB sky. It holds a wealth of information on fundamental physics and the origin and evolution of the Universe. While we have learned a great deal from CMB measurements, including discoveries that have pointed the way to new physics, we have only begun to tap the information contained in CMB polarization, CMB lensing and secondary effects. CMB-S4 will  maximize discovery space by producing high fidelity maps of enormous legacy value.
%
%Through the efforts of the CMB experimental groups over the last decade, the technologies needed for CMB-S4 are now in place. There are, however, considerable technical challenges presented by the required scaling-up of the instrumentation and by the scope and complexity of the data analysis and interpretation.  CMB-S4 will require: scaled-up superconducting detector arrays with well understood and robust material properties and processing techniques; high throughput mm-wave telescopes and optics with unprecedented precision and rejection of systematic contamination; full internal characterization of astronomical foreground emission; large cosmological simulations and theoretical modeling with accuracies yet to be achieved; and computational methods for extracting minute correlations in massive, multi-frequency data sets contaminated by noise and a host of known and unknown signals. 


CMB-S4 is the logical successor to the Stage-3 CMB projects which will operate over the next few years. For maximum impact, CMB-S4 should be implemented on a schedule that allows a transition from Stage~3 to Stage~4 that is as seamless and as timely as possible, preserving the expertise in the community and ensuring a continued stream of CMB science results. This timing is also necessary to ensure the optimum synergistic enhancement of the science return from contemporaneous optical surveys (e.g., LSST, DESI, Euclid and WFIRST).   Information learned from the ongoing Stage-3 experiments, such as the properties of Galactic foregrounds, can be easily incorporated into the CMB-S4 survey strategy, with little or no impact on its design. In fact, the sensitivity and fidelity of the foreground measurements needed to realize the goals of CMB-S4 will only be provided by CMB-S4 itself, at frequencies just below and above those of the main CMB channels.

This timeline is possible because CMB-S4 will use proven existing technology that has been developed and demonstrated by the CMB experimental groups over the last decade. There are, to be sure, considerable technical challenges presented by the required scaling-up of the instrumentation and by the scope and complexity of the data analysis and interpretation.  CMB-S4 will require: scaled-up superconducting detector arrays with well-understood and robust material properties and processing techniques; high-throughput mm-wave telescopes and optics with unprecedented precision and rejection of systematic contamination; full internal characterization of astronomical foreground emission; large cosmological simulations and theoretical modeling with accuracies yet to be achieved; and computational methods for extracting minute correlations in massive, multi-frequency data sets contaminated by noise and a host of known and unknown signals. 


This Science Book sets the scientific goals for CMB-S4 and the measurements required to achieve them. It thereby provides the basis for proceeding with the detailed experimental design. We now provide summaries of the primary science drivers, observables, and analysis/computing issues, each of which is developed in depth in a dedicated chapter of the book.

\subsection*{Inflation: investigating the origin of primordial perturbations and the beginning of time}

Inflation, a period of accelerated expansion of the early Universe, is the leading paradigm for explaining the origin of the primordial density perturbations that grew into the CMB anisotropies and eventually into the stars and galaxies we see around us. In addition to primordial density perturbations, the rapid expansion creates primordial gravitational waves that imprint a characteristic polarization pattern onto the CMB. If our Universe is described by a typical model of inflation that naturally explains the statistical properties of the density perturbations, CMB-S4 will detect this signature of inflation. A detection of this particular polarization pattern would open a completely new window onto the physics of the early Universe and provide us with an additional relic left over from the hot big bang. This relic would constitute our most direct probe of the very early Universe and transform our understanding of several aspects of fundamental physics. Because the polarization pattern is due to quantum fluctuations in the gravitational field during inflation, it would provide insights into the quantum nature of gravity. The strength of the signal, encoded in the tensor-to-scalar ratio $r$, would provide a direct measurement of the expansion rate of the Universe during inflation. A detection with CMB-S4 would point to inflationary physics near the energy scale associated with grand unified theories and would provide additional evidence in favor of the idea of the unification of forces. Knowledge of the scale of inflation would also have broad implications for many other aspects of fundamental physics, including ubiquitous ingredients of string theory like axions and moduli. 
 
Even an upper limit of $r<0.002$ at $95\%$ CL achievable by CMB-S4, over an order of magnitude stronger than current limits, would significantly advance our understanding of inflation. It would rule out the most popular and most widely studied classes of models and dramatically impact how we think about the theory. To some, the remaining class of models would be contrived enough to give up on inflation altogether. Furthermore, CMB-S4 is in a unique position to probe the statistical properties of density perturbations over the entire range of scales that can be probed through measurements of primary anisotropies in the temperature and polarization of the CMB and with unprecedented precision, providing us with invaluable information about the early Universe. 


\subsection*{Neutrinos: setting the neutrino mass scale and testing the 3-neutrino paradigm}

Neutrinos are the least explored corner of the Standard Model of particle physics.  The 2015 Nobel Prize recognized the discovery of neutrino oscillations, which shows that they have mass. However, the overall scale of the masses and the full suite of mixing parameters are still not measured.  Cosmology offers a unique view of neutrinos; they were produced in large numbers in the high temperatures of the early universe and left a distinctive imprint in the cosmic microwave background and on the large-scale structure of the universe. Therefore, CMB-S4 and large-scale structure surveys together will have the power to detect properties of neutrinos that supplement those probed by large terrestrial experiments such as short- and long-baseline as well as neutrino-less double beta decay experiments.


Specifically, while long baseline experiments are sensitive to the differences in the masses of the different types of neutrinos, CMB-S4  will probe the sum of all the neutrino masses. The current lower limit on the sum of neutrino masses imposed by oscillation experiments is \mbox{$\sum m_\nu=58\,\rm{meV}$} for the normal hierarchy and \mbox{$\sum m_\nu=105\,\rm{meV}$} for the inverted hierarchy. CMB-S4, in conjunction with other upcoming cosmic surveys, will measure this sum with high significance. Once determined, the sum of neutrino masses will inform the prospects for future neutrino-less double beta decay experiments that aim to determine whether neutrinos are their own anti-particle. Furthermore, an upper limit below  \mbox{$\sum m_\nu=105\,\rm{meV}$} would determine the mass hierarchy. Finally, CMB-S4 is particularly sensitive to the possible existence of additional neutrinos that interact even more weakly than the neutrinos in the Standard Model. These so-called {\it sterile} neutrinos are also being vigorously pursued with short baseline experiments around the world. So the combination of CMB-S4, large scale structure surveys, and terrestrial probes adds up to a comprehensive assault on the three-neutrino paradigm.



\subsection*{Light Relics: searching for new light particles}

New light particles appear in many attempts to understand both the observed laws of physics and extensions to higher energies.  These light particles are often deeply tied to the underlying symmetries of nature and can play crucial roles in understanding some of the great outstanding problems in physics.  In most cases, these particles interact too weakly to be produced at an appreciable level in Earth-based experiments, making them experimentally elusive.  At the very high temperatures believed to be present in the early Universe, however, even extremely weakly coupled particles can be produced prolifically and can reach thermal equilibrium with the Standard Model particles. Light particles (masses less than $0.1$ eV) produced at early times survive until the time when the CMB is emitted and direct observations become possible.  Neutrinos are one example of such a relic found in the Standard Model.  Extensions of the Standard Model also include a wide variety of possible light relics including axions, sterile neutrinos, hidden photons, and gravitinos.  As a result, the search for light relics from the early Universe with CMB-S4 can shed light on some of the most important questions in fundamental physics, complementing existing collider searches and efforts to detect these light particles in the lab.  
 
Light relics contribute to the total energy density in radiation in the Universe during the radiation era and significantly alter the appearance of the CMB at small angular scales (high multipole number $\ell$). The energy density in radiation controls both the expansion rate of the Universe at that time and the fluctuations in the gravitational potential in which the baryons and photons evolve.  Through these effects, CMB-S4 can provide an exquisite measurement of the total energy density in light weakly-coupled particles, often parametrized by the quantity $\Neff$.  Any additional light particle that decoupled from thermal equilibrium with the Standard Model produces a change to the density equivalent to $\Delta \Neff \geq 0.027$ per effective degree of freedom of the particle.  This is a relatively large contribution to the radiation density that arises from the democratic population of all species during thermal equilibrium.  Realistic configurations of CMB-S4 can reach $\sigma(\Neff) \sim 0.02-0.03$, which will test the minimal contribution of any light relic with spin at $2\sigma$ and at 1$\sigma$ for any particle with zero spin.  $\Neff$ is a unique measurement to cosmology, and it is likely that these thresholds can only be reached by observing the CMB with the angular resolution and sensitivity attainable by a CMB-S4 experiment.   


 
\subsection*{Dark Matter: searching for heavy WIMPS and extremely light axions}

Dark matter is required to explain a host of cosmological observations such as the velocities of galaxies in galaxy clusters, galaxy rotation curves, strong and weak lensing measurements, and the acoustic peak structure of the CMB. While most of these observations could be explained by non-luminous baryonic matter, the CMB provides overwhelming evidence that $85\%$ of the matter in the Universe is non-baryonic, presumably a new particle never observed in terrestrial experiments. Because dark matter has only been observed through its gravitational effects, its microscopic properties remain a mystery. Identifying its nature and its connection to the rest of physics is one of the prime challenges of high energy physics.

Weakly interacting massive particles (WIMPs) are one well-motivated candidate that naturally appears in many extensions of the standard model. A host of experiments are hoping to detect them: deep underground ton-scale detectors, gamma-ray observatories, and the Large Hadron Collider. The CMB provides a complementary probe through annihilation of dark matter into Standard Model particles.   In the WIMP paradigm, the processes that allow dark matter to be created often allow the particles to annihilate with one another. The rate for this process governs how many of the particles remain today and, for a given WIMP mass, is well-constrained by the known dark matter abundance. The same annihilation process injects a small amount of energy into the CMB, slightly distorting its anisotropy power spectrum. CMB-S4 will probe dark matter masses a few times larger than those probed by current CMB experiments.% and will be sensitive to the sweet spot of WIMP masses around $100$ GeV.

Dark matter need not be heavy or thermally produced. Axions provide one compelling example that appears in many extensions of the standard model and is often invoked as solution to some of the most challenging problems in particles physics. Although axions are often extremely light, they can naturally furnish some or all of the dark matter non-thermally. Their effects on the expansion rate of the Universe, on the clustering, and on the local composition of the Universe through quantum fluctuations in the axion field all lead to subtle modifications of the CMB and lensing power spectra. CMB-S4 will improve current limits by as much as an order of magnitude and for some range of masses would be sensitive to axions contributing as little as $1\%$ to the energy density of dark matter.  As with WIMPs, there is an active program of direct and indirect experimental searches for axions that will complement the CMB, and the interplay can reveal important insights into both axions and cosmology.



\subsection*{Dark Energy: growth of structure tests of dark energy and general relativity}

The discovery almost 20 years ago that the expansion of the universe is accelerating presented a profound challenge to our laws of physics, one that we have yet to conquer. Our current framework can explain these observations only by invoking a new substance with bizarre properties ({\it dark energy}) or by changing the century-old, well-tested theory of general relativity invented by Einstein. The current epoch of acceleration is much later than the epoch from which the photons in the CMB originate, and the behavior of dark energy or modifications of gravity do not significantly influence the properties of the primordial CMB. However, during their long journey to our telescopes, CMB photons occasionally interact with the intervening matter and can have their trajectories and their energies slightly distorted. These distortions---gravitational lensing by intervening mass and energy gain by scattering off hot electrons---are small, but powerful experiments currently online have already detected them, and CMB-S4 will exploit them to the fullest extent, enabling us to learn about the mechanism driving the current epoch of acceleration.

The canonical model is that acceleration is driven by a cosmological constant. Although theoretically implausible, this model does satisfy current constraints, so a simple target for CMB-S4 is to test the many predictions this model makes at late times. Using the gravitational lensing of the CMB, the abundance of galaxy clusters, and cosmic velocities, CMB-S4 will measure both the expansion rate $H$ and the amount of clustering, quantified by the parameter $\sigma_8$, as a function of time. The constraints from CMB-S4 alone will be at the sub-percent level on each and, when combined with other experiments, will reach below a tenth of a percent, particularly when the power of CMB-S4 is also harnessed to calibrate these other probes. These constraints will be among the most powerful tests of the cosmological constant; more crucially, this simultaneous sensitivity to expansion and growth will allow us to distinguish the dark energy paradigm from a failure of general relativity. Models for acceleration in this latter class abound, and CMB-S4 will constrain the parameters of these as well. 


\subsection*{CMB lensing: mapping all the mass in the Universe}

The distribution of matter in the Universe contains a wealth of information about the primordial density perturbations and the forces that have shaped our cosmological evolution. Mapping this distribution is one of the central goals of modern cosmology. Gravitational lensing provides a unique method to map the matter between us and distant light sources, and lensing of the CMB, the most distant light source available, allows us to map the matter between us and the surface of last scattering.

Gravitational lensing of the CMB can be measured because the statistical properties of the primordial CMB are exquisitely well-known. As CMB photons travel to Earth from the last scattering surface, they are deflected by intervening matter which distorts the observed pattern of CMB anisotropies and modifies their statistical properties. This can be used to create a map of the gravitational potential that altered the photons' paths. The gravitational potential encodes information about the formation of structure in the Universe and, indirectly, cosmological parameters like the sum of the neutrino masses.  CMB-S4 is expected to produce high fidelity maps over large fractions of the sky, improving on the signal-to-noise of  the Planck lensing maps by more than an order of magnitude.  These maps will inform many of the science targets discussed throughout the book and can also be used to calibrate and enhance results of upcoming galaxy redshift surveys or any other maps of the matter distribution.  Unfortunately, lensing also obscures our view of the CMB.  By measuring and removing the effects of lensing from the CMB maps, we sharpen our view of primordial gravitational waves and our understanding of the very early Universe more generally.  


\subsection*{Data Analysis, Simulations \& Forecasting}

Extracting science from a CMB dataset is a complex, iterative process requiring expertise in both physical and computational sciences. An integral part of the analysis process is played by high-fidelity simulations of the millimeter-wave sky and the experiment's response to the various sources of emission. Fast-turnaround versions of these sky and instrument simulations play a key role at the instrument design stage, allowing exploration of instrument configuration parameter space and projections for science yield. In all three of these areas (analysis, simulations, forecasting), the large leap in detector count and complexity of CMB-S4 over fielded experiments presents challenges to current methods. Some of these challenges are purely computational---for example, performing full time-ordered-data simulations for CMB-S4 will require computing resources and distributed computing tools significantly beyond what was required for \planck. Other challenges are algorithmic, including finding the optimal way to separate the CMB signal of interest from foregrounds and how to optimally combine data from different experimental platforms. To meet these challenges, we will bring the full intellectual and technical resources of the CMB community to bear, in an effort analogous to the unified effort among hardware groups to build the CMB-S4 instrument. A wide cross-section of the CMB theory, phenomenology, and analysis communities has already come together to produce the forecasts shown elsewhere in this document, including detailed code comparisons and agreement on unified frameworks for forecasting.


\eject






 
