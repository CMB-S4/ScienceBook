Extracting science from a CMB dataset is a complex, iterative process requiring expertise in both physical and computational sciences. An integral part of the analysis process is played by high-fidelity simulations of the millimeter-wave sky and the experiment's response to the various sources of emission. Fast-turnaround versions of these sky and instrument simulations play a key role at the instrument design stage, allowing exploration of instrument configuration parameter space and projections for science yield. In all three of these areas (analysis, simulations, forecasting), the large leap in detector count and complexity of CMB-S4 over fielded experiments presents challenges to current methods. Some of these challenges are purely computational---for example, performing full time-ordered-data simulations for CMB-S4 will require computing resources and distributed computing tools significantly beyond what was required for Planck. Other challenges are algorithmic, including finding the optimal way to separate the CMB signal of interest from foregrounds and how to optimally combine data from different experimental platforms. To meet these challenges, we will bring the full intellectual and technical resources of the CMB community to bear, in an effort analogous to the unified effort among hardware groups to build the CMB-S4 instrument. A wide cross-section of the CMB theory, phenomenology, and analysis communities has already come together to produce the forecasts shown elsewhere in this document, including detailed code comparisons and agreement on unified frameworks for forecasting.
