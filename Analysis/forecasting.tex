%%%%%% CMB-S4 Simulations and Data Analysis Chapter, Forecasting Section  %%%%%%%%%%%%%%%%

\section{Forecasting}

We emphasize the importance of accurate forecasting for CMB-S4.
Forecasting efforts for Stage 1 and 2 experiments were hampered by lack of experience with previous deep polarization maps and little knowledge of high latitude Galactic foregrounds.
Forecasting for CMB-S4 will be built on the solid foundation of map-derived evaluations of instrumental noise performance and astrophysical foreground levels from the Stage 2 experiments and the Planck satellite.

The forecasting approach will combine Fisher matrix-derived estimates of power spectrum errors with detailed map-level simulations.
The spectral-domain projections are computationally easy, making them useful to explore the large parameter space of instrument and survey configurations.
Map-domain simulations are used to ground the spectral-domain projections in reality and to challenge them with cases of real world astrophysical complexity.
To gain the benefit of this complementarity, it is important that we maintain compatibility between these two forecasting approaches and establish agreement between them for simple questions before proceeding to more difficult tests.

A key input to the forecasting process are full-season noise maps from existing Stage 2 experiments, which encode actual noise performance and have been verified by null tests on real datasets.
Performance of CMB-S4 can be estimated by rescaling these noise maps, which already contain reality factors such as detector yield, weather, observing efficiency, and filtering of sky modes.
Systematic errors should be included in the projections, with unknown systematics allowed at a level that scales with the map noise used for jackknife null tests.
Forecasting should also include our best knowledge of the astrophysical foregrounds and account for the impact of component separation on CMB-S4 science goals.
The forecasting inputs will improve as we acquire data from Stage 3 experiments and possible complementary balloon-borne experiments, which will produce deeper noise maps and better assessments of foregrounds, as well as demonstrations of new techniques and technologies in development for Stage 4.

Here we describe the main approaches used by our community for forecasting the expected performance of CMB-S4. The central considerations for assessing the expected performance for large-scale B-modes are Galactic foregrounds, ability to delens the data, and a realistic assessment of instrument noise at large scales. 
For the smaller-scale polarization two-point functions (TE, EE) and the lensing four-point function ($\kappa \kappa$), extragalactic foregrounds and instrumental noise are the key considerations.
To forecast the return of the thermal Sunyaev-Zel'dovich effects, an estimate of the expected cluster counts as a function of mass and redshift is the core statistic, combined with an estimate of how well the masses can be calibrated using overlaps with weak lensing surveys. For the kinetic Sunyaev-Zeldovich effect, extragalactic foregrounds and overlap with spectroscopic surveys must all be considered. 

\subsection{Limits on the tensor-to-scalar ratio}

\subsubsection{Spectrum-based domain forecasting}
\label{sec_specforecast}

Power spectra are the primary tool used for CMB analysis.
Forecasting the power spectrum uncertainty and resulting parameter constraints for CMB-S4 is an efficient and powerful tool to explore trade-offs in experiment design.

The bandpower covariance matrix describes the raw sensitivity of all auto and cross-spectra obtained between maps of T, E, and/or B modes at multiple observing frequencies, as well as the signal and noise correlations that exist between these spectra.
This covariance matrix includes contributions from the sample variance of signal fields (CMB and foreground) and instrumental noise, including signal$\times$noise terms.
The signal variance depends on the assumed sky model, which can be modified to explore optimistic or pessimistic scenarios.
As discussed above, estimates of the noise variance should be obtained by rescaling of noise levels that have actually been obtained by Stage 2 experiments (or Stage 3, when available).
Only these scaled noise levels will include all the small ``reality factors'' that are incurred in operating a CMB experiment.

We will explicitly account for the impact of systematic errors by including them in the bandpower covariance matrix.
For constraining tensor-to-scalar ratio, we are particularly concerned with effects that add B-mode power to maps.
Even if the bandpowers are debiased using accurate simulations of such a systematic, it will still leave behind a noise floor due to its sample variance.
Unknown and unforseen sources of spurious signal in CMB-S4 will ultimately be constrained by jackknife null tests, which analyze a map constructed from the difference of two data subsets.
The statistical power of the null test is set by the noise level of the maps, since any signal contributions should difference away.
We can acknowledge this limitation by including in our projections an unknown systematic that adds B modes at the level of the null test uncertainty.
Errors that are multiplicative in the signal, such as an absolute calibration error in the map, are best handled by adding nuisance parameters to the signal model.

Once we have a projection for the bandpower covariance matrix of CMB-S4, we can derive constraints on a parametrized model of cosmological and foreground signals via the Fisher information matrix.
While we are most interested in parameter $r$, it is necessary to also consider the amplitude, spectrum, and spatial distribution of the dust and synchrotron foregrounds (see \cite{bicepkeckplanck15} for an example).
The Fisher matrix formalism allows us to calculate the marginalized error on each parameter, with priors if desired, or to explore the degeneracies between parameters.

By compressing the data down to power spectra, it is feasible to use this technique to evaluate a wide range of survey designs.
The parametrized signal model is also quite flexible and can include complications such as dust--synchrotron correlation or spatially varying foreground spectral indices.
The limitation is that by considering the power spectrum only we are treating all signals as Gaussian, an approximation which must break down at some point for foregrounds.
For this reason, it is important to have the ability to spot check the spectrum-based forecasts against map-based forecasts at specific choices of signal model.

\subsubsection{Map-based domain forecasting}

Foregrounds are intrinsically non-Gaussian, so it is beneficial to consider approaches directly in map space, to check the robustness of spectrum-based approaches, in particular in the case of pessimistic foregrounds where the spectral indices or dust emissivities have non-trivial spatial variation. Here one of the approaches our community uses is a Bayesian model fitting method, where the foregrounds are described parametrically using a physical model for each component.

Using this method, maps of the CMB plus Galactic foreground sky and expected noise are simulated at each of the CMB-S4 frequencies and integrated across the expected bandpasses, using Galactic models as described for example in Section \ref{sec:skymodel}. Simulations at ancillary frequencies that might be provided by other experiments, for example the Planck data, can also be included in the same way. A parameterized model is then fit to the simulated maps, for example fitting the CMB, thermal dust, and synchrotron in small pixels, and typically the synchrotron spectral index and dust emissivity and temperature in larger pixels of order degree-scale or larger. The BB power spectrum of the foreground-marginalized CMB map is then estimated using e.g., the MASTER \cite{hivon02} algorithm or a pixel-based likelihood, and converted into an estimate of $r$ and its uncertainty. Our community has at least two such codes that can perform this procedure (Commander and BFoRe).

This method allows for an assessment of the expected bias on $r$ if the model does not match the simulation, and shows how much the expected uncertainty on $r$ would increase if more complicated foreground models are explored e.g. \cite{armitage-caplan12,remazeilles16}. It is more computationally expensive than spectral-domain forecasts though, so we limit this approach to a smaller subset of explorations. 

\subsection{Limits on parameters from TT/TE/EE/$\kappa\kappa$}

To forecast the expected constraints on cosmological parameters from TT/TE/EE and $\kappa\kappa$ for CMB-S4, many of our community's codes use a Fisher matrix method, which assumes that the resulting parameter distributions are close to Gaussian.  Some forecasts are performed using full MCMC simulations if the parameters are known to be highly non-Gaussian. 

For CMB-S4, the method we follow is to combine CMB-S4 specifications with other available datasets, for example the data from Planck and expected measurements of Baryon Acoustic Oscillations and other low redshift data. 

For the noise levels of Planck, we assume that a data release including reliable polarization data will have happened before CMB-S4 data is taken, and forecast results that include TE and EE data and also large-scale polarization from HFI at multipoles lower than CMB-S4 is expected to reach. This follows approaches in e.g. \cite{allison15}.

Our community uses two approaches to these Fisher forecasts, either considering the unlensed maps and the lensing convergence map as the basic statistics, or the lensed power spectra of those maps together with the reconstructed $\kappa \kappa$ spectrum. In the case of the power spectrum approach, to compute the Fisher matrix for the CMB we use the lensed power spectrum between each pair of fields $X, Y$:
%
\begin{equation}
\label{eqEstimator}
\hat{C}^{XY}_\ell = \frac{1}{2\ell+1}\sum_{m=-\ell}^{m=\ell} x^{*}_{\ell m} y_{\ell m}.
\end{equation}
%
The estimated power spectrum is Gaussian distribution to good approximation at small scales. In this case a full-sky survey has
%
\begin{equation}
-2\ln\mathcal{L}(\boldsymbol{\theta}) = -2\sum_\ell \ln p( \hat{C}_\ell | \boldsymbol{\theta}) \\
=  \sum_\ell  \Big[ (\hat{C}_\ell - C_\ell(\boldsymbol{\theta}) )^\top  \mathbb{C}^{-1}_\ell(\boldsymbol{\theta}) \big(\hat{C}_\ell - C_\ell(\boldsymbol{\theta})) + \ln \det(2 \pi \mathbb{C}_\ell(\boldsymbol{\theta})) \Big]
\end{equation}
%
where $ \hat{C}_\ell = (\hat{C}_\ell^{TT}, \hat{C}_\ell^{TE}, ...) $ contains auto- and cross-spectra and $\mathbb{C}_\ell$ is their covariance matrix. Discarding any parameter dependence in the power spectrum covariance matrix gives
%
\begin{equation}
F_{ij} = \sum_\ell \frac{\partial C^\top_l}{\partial \theta_i} \mathbb{C}^{-1}_\ell \frac{\partial C_l}{\partial \theta_j}.
\end{equation}
%
Here the covariance matrix for the power spectra has elements
%
\begin{equation}
\mathbb{C}(\hat{C}_l^{\alpha \beta}, \hat{C}_l^{\gamma \delta}) = \frac{1}{(2l+1)f_{\rm sky}} \big[ (C_l^{\alpha \gamma} + N_l^{\alpha \gamma}) (C_l^{\beta \delta} + N_l^{\beta \delta})  \\
+ (C_l^{\alpha \delta} + N_l^{\alpha \delta}) (C_l^{\beta \gamma} + N_l^{\beta \gamma}) \big],
\end{equation}
%
where $\alpha, \beta, \gamma, \delta \in \{T, E, B, \kappa_c\}$ and $f_{\rm sky}$ is the effective fractional area of sky used. 

The second approach we consider is to construct the Fisher matrix using the unlensed temperature and polarization fields, and the lensing convergence field, rather than the suite of lensed two-point spectra and the lensing four-point function. Both approaches give consistent estimates.

The CMB lensing reconstruction noise is calculated using the \cite{hu02a} quadratic-estimator formalism. Our nominal approach is to neglect non-Gaussian terms in the power spectrum covariance. We also avoid including information from both lensed BB and the four-point $\kappa \kappa$, as they are covariant. The BB spectrum will not contribute as significantly to S4 constraints, compared to $\kappa \kappa$, and has a highly non-Gaussian covariance \cite{BenoitSmithHu1205}. 

To address the issue of possible extragalactic foregrounds, we set a maximum multipole for the recoverable information of $\ell^T_{\rm max} = 3000$ and $\ell^P_{\rm max} = 4000$ for CMB-S4, as foregrounds are expected to be limiting at smaller scales. We also set a minimum multipole due to the challenge of recovering large-scales from the ground, and consider in general $\ell=50$. We include Planck data at the scales $\ell<\ell_{\rm min}$. 

When relevant, we can also add information from Baryon Acoustic Oscillation (BAO) experiments. This can be done by adding the BAO Fisher matrix
%
\begin{equation}
F_{ij}^{\rm BAO} = \sum_{k} \frac{1}{\sigma_{f,k}^2}\frac{\partial f_k}{\partial \theta_i}\frac{\partial f_k}{\partial \theta_j}
\end{equation}
%
where $f_k = r_s/d_V(z_k)$ is the sound horizon at photon-baryon decoupling $r_s$ over the volume distance $d_V$ to the source galaxies at redshift $z_k$. We also follow standard approaches to including other low redshift probes.
%, 

\subsubsection{Instrument and atmospheric noise}
For initial forecasts, noise spectra are generated assuming the sum of white noise and atmospheric noise. The white noise part is given by
%
\begin{equation}
N^{\alpha \alpha}_\ell = (\Delta T)^2 \exp \left( \frac{\ell(\ell + 1) \theta^2_{\rm FWHM}}{8 \ln 2} \right)
\end{equation}
%
for $\alpha \in \{T, E, B\}$, where $\Delta T$ ($\Delta P$ for polarization) is the map sensitivity in $\mu$K-arcmin and $\theta_{\rm FWHM}$ is the beam width. 

To estimate the atmospheric noise level in intensity, we consider levels at the South Pole and in Chile. For Chile, we base this estimate on current data from ACT and POLARBEAR. For the South Pole, we use data from SPT and BICEP as a guide. We then make the assumption that the atmospheric noise will scale down with observation time for detector arrays within a telescope, and will be uncorrelated for physically separated telescopes. For polarization our nominal estimate is white noise, assuming that the tiny intrinsic polarization of the atmosphere, potentially combined with the use of polarization modulators, minimizes atmospheric contamination.

In the longer term, it is expected that the TT/TE/EE/$\kappa \kappa$ forecasts will move toward the more
experience-based approach laid out in Section \ref{sec_specforecast}. The analytic approximation to 
$N_\ell$ can be replaced by scaled versions of noise spectra achieved in the field by experiments at the
appropriate site. Eventually, full bandpower covariance matrices scaled from fielded experiments can 
also be used.

%

\subsection{Limits on parameters from tSZ/kSZ}

To estimate the expected tSZ cluster counts from S4, we use.
