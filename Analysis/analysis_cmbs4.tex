%%%%%% CMB-S4 Simulations and Data Analysis Chapter  %%%%%%%%%%%%%%%%
 
\chapter{Simulations and Data Analysis}
\renewcommand*\thesection{\arabic{section}}

%%%%%%%%%%%%%%%%%%%%%%%%%%%%%%%%%%%%%%%%%%%%%%%%%%%%%%%%%%%
%%%%%%%%%%%%%%%%%%%%%%%%%%%%%%%%%%%%%%%%%%%%%%%%%%%%%%%%%%%
%%%%%%%%%%%%%%%%%%%%%%%%%%%%%%%%%%%%%%%%%%%%%%%%%%%%%%%%%%%
%%%%%%%%%%%%%%%%%%%%%%%%%%%%%%%%%%%%%%%%%%%%%%%%%%%%%%%%%%%

\section{Introduction}

Extracting science from a CMB dataset is a complex, iterative process requiring expertise in both physical and computational sciences. In this chapter we start by outlining the nominal data analysis pipeline and describing how this is complicated by various real-world issues ranging from instrument systematics to computational tractability. We then discuss the motivations for and corresponding structure of the simulation pipeline. Next we couple these into an overall simulation and data analysis (SimDA) pipeline, and motivate the division of this into 5 SimDA blocks based on the challenges they pose and the expertise they require. Finally we consider each of these blocks in turn, detailing their current state of the art and sketching future prospects.

\newpage

\subsection{Data Analysis}

\begin{figure}[htbp]
\begin{minipage}[h]{0.6\linewidth}
The reduction of a CMB data set nominally proceeds in a sequence of steps.
\begin{description}
\item[ Pre-processing:] The raw time-ordered detector data are calibrated and gross time-domain systematics are either removed, typically by template subtraction or filtering, or flagged.
\item[Map-making:] Estimates of the intensity and the Stokes q- and u-polarizations of the sky signal are extracted from the cleaned time-ordered data at each observing frequency based on their spatial stationarity, and typically using some degree of knowledge of the noise properties. Typically maps are made both using all of the data at a given frequency and splitting the data by time interval, detector subsets, etc to identify and residual systematic effects.
\item[Foreground removal:] The CMB is separated from the various foreground sky components using the set of frequency maps and relying on its unique spectral invariance (in CMB units).
\item[Power spectrum estimation:] The six auto- and cross-angular power spectra of the CMB intensity and E- and B-mode polarizations are estimated from the autocorrelation of the CMB maps or the cross-correlation of the frequency maps, and corrected for E- to B-mode lensing.
\item[Parameter estimation:] The best-fit parameters for any cosmological model are derived by comparing the theoretical CMB power spectra they would induce with the data and their uncertainties.
\end{description}
\end{minipage}
\begin{minipage}[h]{0.4\linewidth}
\begin{center}
\includegraphics[width=0.4\textwidth]{Analysis/nda}
\caption{The nominal CMB data analysis pipeline; ovals represent data objects and rectangles processing steps.}
\label{fig_nda}
\end{center}
\end{minipage}
\end{figure}

This reduction essentially consists of a series of changes of basis of the data, from time samples (shaded red in Figure \ref{fig_nda} and beyond) to map pixels (blue) to spectral multipoles (green) to cosmological parameters (yellow), with each basis-change reducing the data volume, increasing the signal-to-noise, and exposing a different class of systematic effects for mitigation. Note however that the data set can only remain a sufficient statistic for the cosmological parameters if we also propagate its full covariance matrix between the various bases. 

In addition to this basic linear data reduction pipeline there are various side-processes (Figure \ref{fig_da}). The most critical of these is the characterization of the mission itself (comprising both the instrument and the observation) from the time-ordered data - for example determining beam properties from planet crossings; estimating noise properties (including cross-correlations) from calibrated, signal-subtracted, time-ordered data; or reconstructing the telescope pointing solution from ancilliary time-domain data such as star-trackers. The resulting mission model then feeds back into all of the ensuing data reduction and interpretation.

In addition, given sufficient observing frequencies we can not simply perform foreground removal, but rather component separation, where the individual foreground components are isolated based on their different (possibly spatially varying) spectral dependencies; such component maps then provide important inputs to astronomy and astrophysics. Similarly a wide range of analyses of the CMB maps, power spectra and parameters collectively inform our understanding of cosmology and fundamental physics, including measurements of the energy scale of inflation and the sum of neutrino masses from the power spectra and of cosmological non-Gaussianity from the higher-order statistics of the CMB map.

\begin{figure}[htbp]
\hspace*{3in}\includegraphics[width=0.5\textwidth]{Analysis/da}
\caption{The full CMB data analysis pipeline}
\label{fig_da}
\end{figure}

\newpage

\subsection{Simulation}

Needed for 
\begin{itemize}
\item Mission design and development (both instrument and observation)
\item Validation and verification of analysis tools
\item Absent a full covariance matrix, Monte Carlo based uncertainty quantification and debiasing.
\end{itemize}

From top to bottom, there is an inevitable trade-off between the computational cost of generating the simulation and its realism and accuracy.

\begin{figure}[htbp]
\includegraphics[width=0.5\textwidth]{Analysis/sim}
\caption{The CMB simulation pipeline}
\label{default}

\end{figure}

\newpage

\subsection{Coupling Simulation and Data Analysis}

The overall simulation and data analysis pipeline runs both as a top-down data reduction and a wrap-around refinement of mission and sky characterization and modeling.

It can be subdivided into 5 blocks based on the challenges faced and expertise required to address them.

\begin{figure}[htbp]
\centering
\includegraphics[width=1\textwidth]{Analysis/simda}
\caption{The full CMB simulation/data analysis pipeline}
\label{default}

\end{figure}

\newpage

\input Analysis/forecasting.tex

\newpage

\input Analysis/sky_modeling.tex

\newpage

\input Analysis/tod_processing.tex

\newpage

\input Analysis/component_separation.tex

\newpage

\input Analysis/statistics_parameters.tex

\newpage

%\bibliography{cmbs4}

%%
%% Populate the .bib file with entries from SPIRES Bibtex (preferred)
%% or ADS Bibtex (if no SPIRES entry).
%%  SPIRES will also supply the CITATION line information; please include it.
%%


